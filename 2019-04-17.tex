\begin{ex}[]
    Suppose we have a finite Galois extension that's separable. What if we have a finite separable extension? A Galois separable extension has only finitely many subfields---as they're in correspondence with the subgroups of a finite group. Thus, if we have $K \ge E \ge F$ with $K/F$ Galois and $E/F$ separable, then $E$ has only finitely many subfields over $F$.

    (Recall that for finite field extensions $\Fpn$, we may find an $\alpha \in \Fpn$ such that $\ang{\alpha} = \Fpn^\times$.)
\end{ex}

\begin{prop}[Generating a simple extension]
    Suppose 
    \[
        \xymatrix{K \ar@{-}[d]^{\text{finite}}\\ F}
    \]
    Is $K$ of the form $F(\theta)$? Yes, if and only if there are finitely many subfields of $K$ containing $F$.
\end{prop}

\begin{defn}[Simple extensions]
    Say that $K/F$ is a field extension. If $K = F(\theta)$ for some $\theta \in \K$, then $K/F$ is called a \term{simple extension}. 
\end{defn}

\begin{proof}
Let $f$ be the minimal polynomial for $\theta$ over $F$, and $g$ the minimal polynomial for $\theta$ over $E$.
    \[
        \xymatrix{F(\theta) \ar@{=}[r] 
            & K \ar@{-}[d]^{\text{finite}}\\
            & E \ar@{-}[d]^{\text{finite}}\\
            & F}
    \]
    Then suppose $E'$ is the field generated by $F$ and the coefficients of $g(x)$. Then
    \[
        \xymatrix{F(\theta) \ar@{=}[r] 
            & K \ar@{-}[dl] \ar@{-}[d] \\
        E \ar[r]^{\text{\subset}} & E'\ar@{-}[d]\\
            & F}
    \]
    Since the minimal polynomial of $\theta$ over $E'$ is still $g(x)$, we have 
    \[
        [K:E] = \deg g = [K: E'].
    \]
    Because $g(x)$ is a monic factor of $f \in K[x]$, we only have finitely many choices for $g(x)$. Thence only finitely many choices for $E' \subset E$.

    Conversely, we need to show that a field generated by any number of elements adjoined can be generated by a single element. Well, it suffices to show if I have $F(\alpha, \beta)$, then I can find $\gamma$ that generates $F(\alpha, \beta) = F(\gamma)$. (E.g., suppose I have an extension generated by two elements $\Q(\sqrt{2}, \sqrt{3})$. How can I prove that $\sqrt{2} + \sqrt{3}$ generates $\Q(\sqrt{2}, \sqrt{3})$?)

    If $F$ is finite, we know the result. So suppose that $F$ is infinite. Consider 
    \[
        \set{F(\alpha + c \beta): c \in F} \qq{which is a \emph{finite} set by hypothesis.}
    \]
    Thence for some $c \neq c' \in F$, we have $F(\alpha + c \beta) = E = F(\alpha + c' \beta)$.

    Thus $c -c'$ is nonzero in $F \subset E$, so 
    \begin{align}
        \alpha + c \beta \in E \qand \alpha + c' \beta &\in E\\
        &\implies (c - c') \beta \in E\\
        &\implies \beta \in E\\
        &\implies (\alpha + c\beta) - c\beta \in E\\
        &\implies \alpha \in E.
    \end{align}
    We've shown $E \supset F$, and since $E \subset F$ is \TODO, we conclude $E = F(\alpha, \beta)$. So $F(\alpha, \beta) = F(\alpha + c\beta)$.
\end{proof}

What are the main results from chapter 14? 
\begin{itemize}
    \item That Galois closures exist, and 
    \item that any finite separable extension is simple.
\end{itemize}

\begin{thm}[Primitive $n$th roots of unity]
   Consider that $\zeta_n$ is a primitive $n$th root of unity. 
   \[
       \xymatrix{\Q(\zeta_n) \ar@{-}[d]\\
       \Q}
   \]
   \TODO. Prove that $\Phi_n(x)$, the $n$th cyclotomic polynomial, is irreducible over $\Q$. Then $\Q(\zeta_n)$ is the splitting field for $\Phi_n(x)$ over $\Q$.
\end{thm}

\begin{ex}[Galois group of $\Q(\zeta_n)$]
    Consider $n = 12$.
   \[
       \xymatrix{\Q(\zeta_{12}) \ar@{-}[d]^{4}\\
       \Q}
   \]
   Since $\Phi_{12}(x) = \prod_{k = 1, 5, 7, 11} (x - \zeta^k)$, and the $\zeta^k$ are conjugates, $G = \Gal(\Q(\zeta))$ acts transitively on the linear factors of this product. E.g., if $\sigma \in G$ acts on $\zeta$ by $\sigma(\zeta) = \zeta^5$, then $\sigma(\zeta^7) = \zeta^{35} = \zeta^{11}$. So $\sigma(\zeta^a)$ is determined for all $a \in \Z/12\Z$. Since for any $\sigma, \tau$ mapping 
   \[
       \sigma \colon \zeta \mapsto \zeta^a \qand 
       \tau \colon \zeta \mapsto \zeta^b,
   \]
   Then we have $\sigma\tau \colon \zeta \mapsto \zeta^{ab \pmod{12}}$. Whence we can define an isomorphism of groups $G \to (\Z/n \Z)^\times$ such that for $\sigma(\zeta) = \zeta^a$, we map $\sigma \mapsto a \pmod{n}$.
\end{ex}

\begin{note}[]
    Recall that $12 \mid 24$, so $k^2 \equiv 1 \pmod{12}$ for all $k = 1, 5, 7, 11$. So $\Gal(\Q(\zeta_{12}) \cong V_4$, the Klein four group.
\end{note}

\begin{ex}[]
    Let $\zeta_5$ be a primitive $5$th root of unity. Since adjoining $\zeta$ is a Galois extension (the intermediate extensions have degree $2$). We compute $\Gal(\Q(\zeta_5)) \cong \paren{\Z/5\Z}^\times \cong \Z/4\Z$. 
   \[
       \xymatrix{\Q(\zeta_{4}) \ar@{-}[d]^{2}
           & 1 \ar@{-}[d]\\
       \Q(?) \ar@{-}[d]^{2}
           & H \ar@{-}[d]\\
       \Q & \Z_4 = \ang{\sigma}  \ar@{-}[d]\\
   \]

   Choose an automorphism $\sigma$ of order $4$. Since $\sigma(\zeta) = \zeta^k$ has order $4$ if and only if $k = 2,3$ (since $k =1,4$ would produce an automorphism of order $2$). Then $H = \ang{\id, \sigma^2}$. 

   We can sum the elements in the group (recall from group algebras that $h \in G$ and $h \paren{\sum_{g \in G} g} = \sum_{g \in G} g$) $(1 + \sigma^2)(\zeta) = \zeta + \zeta^4 \in \R$.

   Then we can compute $\Q(\zeta^4 + \zeta) = \Q(\sqrt{5})$ \TODO. (Consider that it's a field fixed by conjugation.)
\end{ex}
