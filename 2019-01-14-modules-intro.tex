\documentclass[11pt]{amsart}
\usepackage[utf8]{inputenc}
\usepackage{fullpage}
\usepackage{hyperref}
\usepackage{ccg-macros}

\setcounter{secnumdepth}{0}
\providecommand{\tightlist}{%
  \setlength{\itemsep}{0pt}\setlength{\parskip}{0pt}}

\title{Introduction to Module Theory: Basic Definitions and Examples}
\author{Colton Grainger (MATH 6140 Algebra 2)}
\date{2019-01-14}

\begin{document}

\maketitle

\setcounter{section}{0}

\subsection{Assignment due 2019-01-13}

\subsubsection{\texorpdfstring{{[}1, No.
10.1.1{]}}{, number 10.1.1{[}1, No. 10.1.1{]}}}

\gvn \(R\) is a unital ring and \(M\) is a left \(R\)-module.

\wts \(0m = 0\) and \((-1)m = -m\) for all \(m \in M\).

\subsubsection{\texorpdfstring{{[}1, No.
10.1.3{]}}{, number 10.1.3{[}1, No. 10.1.3{]}}}

\gvn Say \(rm = 0\) for some \(r \in R\) and some \(m \in M\) with
\(m \neq 0\).

\wts There is no \(s \in R\) such that \(sr =1\).

\subsubsection{\texorpdfstring{{[}1, No.
10.1.4{]}}{, number 10.1.4{[}1, No. 10.1.4{]}}}

\gvn Let \(M\) be the modules \(R^n\) and let
\(\fa_1, \fa_2, \ldots, \fa_n\) be left ideals of \(R\).

\wts Both of the following are submodules of \(M\):

\begin{enumerate}
\def\labelenumi{\alph{enumi}.}
\tightlist
\item
  \(P = \{(x_1, x_2, \ldots, x_n) : x_i \in \fa_i\}\),
\item
  \(N = \left\{(x_1, x_2, \ldots, x_n) : x_i \in R \text{ and } \sum_i x_i = 0\right\}\).
\end{enumerate}

\subsubsection{\texorpdfstring{{[}1, No.
10.1.5{]}}{, number 10.1.5{[}1, No. 10.1.5{]}}}

\gvn Consider a left ideal \(\fa\) of \(R\). Let
\[\fa M = \left\{\sum_\text{finite} a_i m_i : a_i \in \fa, m_i \in M\right\}.\]

\wts We have \(\fa M\) as a submodule of \(M\).

\subsubsection{\texorpdfstring{{[}1, No.
10.1.6{]}}{, number 10.1.6{[}1, No. 10.1.6{]}}}

\gvn Let \(M\) be a module over \(R\) and \(\{N_i\}\) be a nonempty
collection of submodules.

\wts The intersection \(\bigcap_i N_i\) is a submodule of \(M\).

\subsubsection{\texorpdfstring{{[}1, No.
10.1.8{]}}{, number 10.1.8{[}1, No. 10.1.8{]}}}

\newcommand{\Tor}[1]{\mathrm{Tor}\left( #1 \right)}

\gvn An element \(m\) of the \(R\)-module \(M\) is called a
\emph{torsion element} if \(rm = 0\) for some nonzero element
\(r \in R\). The set of torsion elements is denoted
\[\mathrm{Tor}\left( M \right) = \{ m \in M : rm = 0 \text{ for some nonzero } r \in R\}.\]

\wts

\begin{enumerate}
\def\labelenumi{\alph{enumi}.}
\item
  If \(R\) is an integral domain, then \(\mathrm{Tor}\left( M \right)\)
  is a submodule of \(M\) (called the \emph{torsion submodule}).
\item
  If \(R\) has zero divisors, then every nonzero \(R\)-module has
  nonzero torsion elements.
\end{enumerate}

\subsubsection{\texorpdfstring{{[}1, No.
10.1.9{]}}{, number 10.1.9{[}1, No. 10.1.9{]}}}

\gvn If \(N\) is a submodule of \(M\), the \emph{annihilator of \(N\) in
\(R\)} is defined to be
\[\{r \in R : rn = 0 \text{ for all } n \in N\}.\]

\wts The annihilator of \(N\) in \(R\) is a \(2\)-sided ideal of \(R\).

\subsubsection{\texorpdfstring{{[}1, No.
10.1.15{]}}{, number 10.1.15{[}1, No. 10.1.15{]}}}

\gvn Say \(M\) is a finite abelian group. \(M\) is naturally a
\(\ZZ\)-module.

\wts This action cannot be extended to make \(M\) into a \(\QQ\)-module.

\subsubsection{\texorpdfstring{{[}1, No.
10.1.18{]}}{, number 10.1.18{[}1, No. 10.1.18{]}}}

\gvn Let \(F = \RR\), let \(V = \RR^2\), and let \(T\) be the linear
transformation from \(V\) to \(V\) that is rotation clockwise about the
origin by \(\pi / 2\) radians.

\wts \(V\) and \(0\) are the only \(F[x]\)-submodules for this \(T\).

\subsubsection{\texorpdfstring{{[}1, No.
10.1.19{]}}{, number 10.1.19{[}1, No. 10.1.19{]}}}

\gvn Let \(F = \RR\), let \(V = \RR^2\), and let \(T\) be the linear
transformation from \(V\) to \(V\) that is projection onto the
\(y\)-axis.

\wts \(V\), \(0\), the \(x\)-axis and the \(y\)-axis are the only
\(F[x]\)-submodules for this \(T\).

\subsubsection{\texorpdfstring{{[}1, No.
10.1.20{]}}{, number 10.1.20{[}1, No. 10.1.20{]}}}

\gvn Let \(F = \RR\), let \(V = \RR^2\), and let \(T\) be the linear
transformation from \(V\) to \(V\) that is rotation clockwise about the
origin by \(\pi\) radians.

\wts Every subspace of \(V\) is an \(F[x]\)-submodule for this \(T\).

\subsubsection{\texorpdfstring{{[}1, No.
10.1.21{]}}{, number 10.1.21{[}1, No. 10.1.21{]}}}

\gvn Let \(n \in \ZZ^+\), \(n >1\), and \(R\) be the ring \(\sM_n(F)\)
of \(n \times n\) matrices from the field \(F\). Let
\(M\subset \sM_n(F)\) be
\[M = \left\{(a_i^j) : a_i^j = 0 \text{ if } j > 1\right\},\] that is,
the set of matrices with arbitrary elements of \(F\) in the first column
and zeros elsewhere.

\wts

\begin{itemize}
\tightlist
\item
  \(M\) is a submodule of \(R\) when \(R\) is considered as a left
  module over itself.
\item
  \(M\) is \emph{not} a submodule of \(R\) when \(R\) is considered as a
  right module.
\end{itemize}

\subsection*{References}
\addcontentsline{toc}{subsection}{References}

\hypertarget{refs}{}
\leavevmode\hypertarget{ref-DF04}{}%
{[}1{]} D. Dummit and R. Foote, \emph{Abstract algebra}. Prentice Hall,
2004.

\end{document}
