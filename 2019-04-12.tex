\hypertarget{ideas-behind-the-fundamental-theorem-of-galois-theory}{%
\subsection{Ideas behind the fundamental theorem of Galois
theory}\label{ideas-behind-the-fundamental-theorem-of-galois-theory}}

Recall: in favorable conditions, the degree
\(\abs{\Aut{K/F}} \le [K: F]\). If \(K/F\) is Galois, then we have
equality.

Let \(K/F\) be a Galois extension over a field \(F\). \TFAE.

\begin{itemize}
\item
  The automorphism group \(\Aut{K/F}\) is ``sufficiently big''.
\item
  The action of \(\Aut{K/F}\) on \(K\) has stabilizer subgroup
  \(\Stab{\Aut{K/F}}{F} = \Aut{K/F}\).
\item
  \(\abs{\Aut{K/F}} = [K:F]\).
\end{itemize}

The fundamental theorem of Galois theory claims that there's a
\emph{bijective} and \emph{order-reversing} correspondence between the
subfields of \(K/F\) and the subgroups of \(\Gal{K/F}\).

For example, the groups lattice of subgroups \(1 \le H \le G\)
corresponds to the lattice of field extensions \(K \ge E \ge F\). That
is: \$\$\xymatrix{
    K\ar{-}[rr]& & 1\ar[d]\\
    E\ar{-}[rr] \ar[u]& & H\ar[d]\\
    F\ar{-}[rr] \ar[u]& & G}\$\$ 

The degree of \(K/E\) is \(H\), with

Moreover, \([E:F] = [G:H]\). \(K/E\) is also Galois, with
\(\Gal{K/E} = H\).

\[E \qq{ is Galois over $F$ iff} H \qq{is normal in} G.\]

\hypertarget{example-qqsqrt32-omega.}{%
\subsection{\texorpdfstring{Example
\(\Q(\sqrt[3]{2}, \omega)\).}{Example \textbackslash Q(\textbackslash sqrt{[}3{]}\{2\}, \textbackslash omega).}}\label{example-qqsqrt32-omega.}}

Recall that the minimal polynomial of \(\Q(\sqrt[3]{2}, \omega)\) over
\(\Q\) is \(x^3 - 2\).

We have the tower of extensions \TODO. Consider that
\(\Q(\sqrt[3]{2}) \le \Q(\sqrt[3]{2}, \omega)\), but there's
\(\sigma \in \Aut{\Q(\sqrt[3]{2}, \omega)/\Q}\) that takes
\(\sigma(\Q(\sqrt[3]{2})) = \Q(\sqrt[3]{2}\omega)\). (In other words,
the fields \(\Q(\sqrt[3]{2}\omega^k)\) for \(k = 0,1,2\) correspond to
the \(2\)-cycles in \(S_3\).

\hypertarget{lattice-isomorphisms}{%
\subsection{Lattice isomorphisms}\label{lattice-isomorphisms}}

What's the largest subfield contained in the subfields \(E_1, E_2\) of
\(K\)? It's the intersection. How about the largest subfield containing
both \(E_1\) and \(E_2\)? It's the composite in \(K\). Correspondingly,
for the subgroups \(G_1\) and \(G_2\) of \(\Aut{K/F}\) fixing \(E_1\)
and \(E_2\), the subgroup \(G_1 \cap G_2\) fixes the composite, and the
subgroup \(\langle G_1, G_2\rangle\) fixes the intersection. \TODO

\hypertarget{finite-fields}{%
\subsection{Finite fields}\label{finite-fields}}

Exercise: prove that the algebraic closure of a field is an infinite
degree extension.

Fact: consider the algebraic closure \(\F_p^a\) of \(\F_p\). Since
\(\F_{p^n}/\F_p\) is an algebraic extension, we have
\[\F_{p^n} \qq{ contained in } \F_p^a.\]

Idea: there \emph{is} a \emph{non-algebraic} extension of \(\CC\), e.g.,
\(\CC(t)\), the polynomial ring.

Consider that \(\F_{p^n}/\F_p\) is an algebraic extension, since
\(\F_{p^n}\) is the splitting field of \(x^{p^n} - x\) over \(\F_p\).
Whence \(\F_{p^n}/\F_p\) is a \emph{Galois} extension. So also
\[\abs{\Gal{\F_{p^n}/\F_p}} = n.\]

Consider the Froebenius automorphism \(\Phi\colon \F_{p^n} \to \F_p\)
defined by \(\Phi(a) = a^p\). Note (by Fermat's little theorem) \(\Phi\)
fixes \(\F_p\) (this in general holds for prime subfields). Now we have
\[\Phi \in \Gal{\F_{p^n}/\F_p} \qq{ and } \Phi^k = \id \qq{iff} n \mid k.\]

\pf If \(\Phi^k = \id\), then \(\alpha^{p^k} -\alpha = 0\) for all
\(\alpha \in \F_{p^n}\). But there are not enough roots! \TODO~tighten
up. \qedsymbol

Thus \(\Gal{\F_{p^n}/\F_p}\) is cyclic, finitely generated by
\(\Phi\).

Consider the tower \(\F_{p^n} \ge \F_{p^k} \ge \F_p\). The degrees of
the extensions are \(n/k\) and \(k\) respectively, where we \emph{must
have} \(k \mid n\). The corresponding subgroups are
\(1 \le \text{cyclic of order $n/k$} \le \text{cyclic of order $n$}\).
The generators are thence
\(\langle \Phi^n\rangle \le \langle \Phi^k \rangle \le \langle \Phi \rangle\).

\pf (Consider Lagrange's theorem.)

Moral. What Galois groups can appear as automorphism groups of
extensions of finite fields? Only cyclic groups.

\hypertarget{example-qqsqrt2-sqrt3-qqsqrt2-sqrt3.}{%
\subsection{\texorpdfstring{Example
\(\Q(\sqrt{2}, \sqrt{3}) = \Q(\sqrt{2} + \sqrt{3})\).}{Example \textbackslash Q(\textbackslash sqrt\{2\}, \textbackslash sqrt\{3\}) = \textbackslash Q(\textbackslash sqrt\{2\} + \textbackslash sqrt\{3\}).}}\label{example-qqsqrt2-sqrt3-qqsqrt2-sqrt3.}}

\hypertarget{consider-ff_pn-a-simple-extension-of-ff_p.}{%
\subsection{\texorpdfstring{Consider \(\F_{p^n}\), a \emph{simple
}extension of
\(\F_p\).}{Consider \textbackslash F\_\{p\^{}n\}, a simple extension of \textbackslash F\_p.}}\label{consider-ff_pn-a-simple-extension-of-ff_p.}}

\pf Let \(\alpha\) be a generator of \(\F_{p^n}^\times\). Since the
finite subgroups of the group of units of a field is cyclic, and the
group \(\F_{p^n}^\times\) \emph{is finite}. Then
\(\F_{p^n} = \F_p(\alpha)\).

But how to find \(\alpha\)?

\begin{itemize}
\item
  Randomly? (c.f. stackexchange.)
\item
  Observe that the minimal polynomial of \(\alpha\) (such that
  \(\F_p(\alpha)\) is a degree \(n\) extension) has degree \(n\).
\item
  Proposition. \TODO~For any \(n\), there's a irreducible polynomial of
  degree \(n\) over \(\F_p\).
\end{itemize}

Consider the tower \[\xymatrix{
    \F_{p^n} = \F_p(\alpha) \ar[d] \\
    E = \F_p(\beta) \ar[d] \\
    \F_p}\]

With the degree of \(\alpha\) equal to \(n\), and the degree of
\(\beta\) equal to \(d\), we have \(d \mid n\). But also the minimal
polynomials \(m_\alpha(x)\) and \(m_\beta(x)\) both \emph{divide}
\(x^{p^n} - x \in \F_{p}[x]\).

\pf Suppose \(m(x)\) is an irreducible factor of \(x^{p^n} -x\). Let
\(\gamma\) be a root of \(m(x)\) in \(\F_{p^n}\) of degree \(d\).
Consider: \[\xymatrix{
    \F_{p^n} \ar[dd]^n \ar[dr]^{n/d} & \\
    & \F_p(\gamma) \ar[dl]^d \\
    \F_p &}\] Thence \(d\mid n\).

Conversely, if we have an irreducible polynomial of degree \(d\) over
\(\F_p\) and \(d\mid n\), then every element of \(\F_p(\gamma)\) has
degree dividing \(d\).
\[\gamma^{p^d} - \gamma = 0 \quad \gamma \in \F_{p^d}.\]

\hypertarget{example.-factorize-x8---x-over-ff_2.}{%
\subsection{\texorpdfstring{Example. Factorize \(x^8 - x\) over
\(\F_2\).}{Example. Factorize x\^{}8 - x over \textbackslash F\_2.}}\label{example.-factorize-x8---x-over-ff_2.}}

Consider \(x^{2^3} -x\). We've a degree \(8\) polynomial. So consider
\begin{align*}
x^8 - x &= x(x^ -1)\\
    &= x (x -1) (x^6 + x^5 + \cdots + 1).
\end{align*}

What a chore?! Instead with \(p =2\) and \(n= 3\). We've have a table

\begin{longtable}[]{@{}ll@{}}
\toprule
poly & irreduc?\tabularnewline
\midrule
\endhead
\(x^3 + 1\) & no (\(-1\) is a root)\tabularnewline
\(x^3 +x^2 + 1\) & yes\tabularnewline
\(x^3 + x + 1\) & yes\tabularnewline
\(x^3 + x^2 + x + 1\) & no (\(-1\) is a root)\tabularnewline
\bottomrule
\end{longtable}

Therefore \(x(x+1)(x^3 + x + 1)(x^3+ x^2 + 1) = x^8 -x\). \qedsymbol

\hypertarget{example-x4-1-in-zzx.}{%
\subsection{\texorpdfstring{Example
\(x^4 + 1 \in \Z[x]\).}{Example x\^{}4 + 1 \textbackslash in \textbackslash Z{[}x{]}.}}\label{example-x4-1-in-zzx.}}

\TODO~Show that this polynomial is reducible, but modulo any prime
\(p\), irreducible.
