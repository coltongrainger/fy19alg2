\documentclass{article}
\usepackage[T1]{fontenc}
\usepackage[english]{babel}
\usepackage[utf8]{inputenc}
\usepackage{fancyhdr}
\usepackage{amsmath}
\usepackage{amsfonts}
\usepackage{amssymb}
\usepackage{amsthm} 
\usepackage{thmtools}
\usepackage{lipsum}
\usepackage{geometry}
\usepackage{mathtools}
\usepackage{bold-extra}
\usepackage{mathrsfs}
\usepackage{tikz}
\usepackage{tikz-cd}
\usepackage[makeroom]{cancel}
\usepackage{hanging}
\usepackage{stmaryrd}
\usepackage{enumerate}
\usepackage{color, soul}
\usepackage{fancyhdr}
\usepackage{titlesec}
\usepackage{parskip}
\usepackage{soul}
\usepackage{graphicx}
\usepackage{mathdots}

\DeclareSymbolFont{extraup}{U}{zavm}{m}{n}
\DeclareMathSymbol{\varheart}{\mathalpha}{extraup}{86}
\DeclareMathSymbol{\vardiamond}{\mathalpha}{extraup}{87}

\geometry{a4paper,total={168mm,248mm},left=21mm,top=22mm}

%theorems, etc.
\theoremstyle{definition}
\newtheorem{thm}{Theorem}[section]
%\numberwithin{thm}{subsection}
\newtheorem{lem}[thm]{Lemma}
%\numberwithin{lem}{subsection}
\newtheorem{prop}[thm]{Proposition}
%\numberwithin{prop}{subsection}
\newtheorem{cor}[thm]{Corollary}
%\numberwithin{cor}{subsection}
\newtheorem{defn}[thm]{Definition}
%\numberwithin{defn}{subsection}
\newtheorem{conj}[thm]{Conjecture}
%\numberwithin{conj}{subsection}
\newtheorem{exam}[thm]{Example}
%\numberwithin{exam}{subsection}
\newtheorem{alg}[thm]{Algorithm}
%\numberwithin{alg}{subsection}
\newtheorem{hw}[thm]{Exercise}
\newtheorem{note}[thm]{Note}
%\numberwithin{note}{subsection}
\newtheorem{rem}[thm]{Remark}
%\numberwithin{rem}{subsection}

\newcommand{\dfn}{\textbf{Definition. }}


%\numberwithin{equation}{section}

\usepackage[colorlinks]{hyperref}
\usepackage[nameinlink,capitalize]{cleveref}


%\titleformat{<command>}[<shape>]{<format>}{<label>}{<sep>}{<before-code>}[<after-code>]

\setlength\parindent{0pt}


%========= Spacing and format
\newcommand{\cen}{\centerline}
\newcommand{\hang}{\hangindent=0.8cm}
\newcommand{\nhang}{\hangindent=0cm}
\newcommand{\nf}{\normalfont}
\newcommand{\fl}{\noindent}
\newcommand{\vs}{\vspace{0.7em}}
\newcommand{\vv}{\\\vs}
\newcommand{\nl}{\textcolor{white}{nothing}}



%========= Common Math commands
\newcommand{\ex}{\exists}
\newcommand{\nin}{\not\in}
\newcommand{\ra}{\rightarrow}
\newcommand{\Ra}{\Rightarrow}
\newcommand{\La}{\Leftarrow}
\newcommand{\oa}{\overrightarrow}
\newcommand{\lbb}{\llbracket}
\newcommand{\rbb}{\rrbracket}
\newcommand{\p}{\prime}
\newcommand{\wh}{\widehat}
\newcommand{\os}{\overset}
\newcommand{\us}{\underset}
\newcommand{\mf}{\mathfrak}
\newcommand{\ol}{\overline}
\newcommand{\td}{\widetilde}
\newcommand{\seq}{\subseteq}
\newcommand{\lp}{\left(}
\newcommand{\rp}{\right)}
\newcommand{\im}{\text{im}}
\newcommand{\inv}{^{-1}}




%========= Math letters
\newcommand{\R}{\mathbb{R}}
\newcommand{\C}{\mathbb{C}}
\newcommand{\Q}{\mathbb{Q}}
\newcommand{\Z}{\mathbb{Z}}
\newcommand{\F}{\mathbb{F}}
\newcommand{\K}{\mathbb{K}}
\newcommand{\N}{\mathbb{N}}
\newcommand{\E}{\mathbb{E}}
\newcommand{\fs}{\mathscr{S}} %fancy S
\newcommand{\ff}{\mathscr{F}} %fancy F
\newcommand{\OO}{\mathcal{O}}
\newcommand{\FF}{\mathcal{F}}
\newcommand{\bs}{\mathbb{S}}


\newcommand{\fg}{\mathfrak{g}}
\newcommand{\LL}{\mathcal{L}}
\newcommand{\fke}{\mathfrak{e}}
\newcommand{\al}{\alpha}
\newcommand{\ga}{\gamma}
\newcommand{\de}{\delta}
\newcommand{\Ga}{\Gamma}
\newcommand{\be}{\beta}
\newcommand{\Lm}{\Lambda}
\newcommand{\lm}{\lambda}
\newcommand{\Sig}{\Sigma}
\newcommand{\sig}{\sigma}
\newcommand{\Tht}{\Theta}
\newcommand{\tht}{\theta}
\newcommand{\vphi}{\varphi}
\newcommand{\vep}{\varepsilon}



%========= Linear Algebra
\newcommand{\bpm}{\begin{pmatrix}}
\newcommand{\epm}{\end{pmatrix}}
\newcommand{\bsm}{\left( \begin{smallmatrix}}
\newcommand{\esm}{\end{smallmatrix}\right)}
\newcommand{\hh}{\hspace{2em}}
\newcommand{\vect}{\overset{\rightharpoonup}}


%========= Algebra notation
\newcommand{\Aut}{\text{Aut}}
\newcommand{\Inn}{\text{Inn}}
\newcommand{\Tor}{\text{Tor}}
\newcommand{\Hom}{\text{Hom}}
\newcommand{\End}{\text{End}}
\newcommand{\Gal}{\text{Gal}}
\newcommand{\Fix}{\text{Fix}}


%========= Analysis notation
\newcommand{\M}{\mathcal{M}} 



%========= Topology notation
\newcommand{\YY}{\mathcal{Y}}
\newcommand{\PP}{\mathcal{P}}
\newcommand{\BB}{\mathcal{B}}
\newcommand{\CS}{\mathcal{S}}
\newcommand{\CC}{\mathcal{C}}
\newcommand{\FB}{\mathfrak{B}}
\newcommand{\EE}{\mathcal{E}}
\newcommand{\wt}{\widetilde}
\newcommand{\es}{\varnothing}



%========= Diff. Geo Shorthand
\newcommand{\bv}{\textbf{v}}
\newcommand{\bw}{\textbf{w}}
\newcommand{\A}{\mathcal{A}}
\newcommand{\BS}{\mathbb{S}}
\newcommand{\BSS}{\mathbb{S}^1}
\newcommand{\BSN}{\mathbb{S}^n}
\newcommand{\CP}{\mathbb{CP}}
\newcommand{\B}{\mathbb{B}}
\newcommand{\RP}{\mathbb{RP}}
\newcommand{\Cin}{C^\infty}
\newcommand{\px}{\widehat{x}}
\newcommand{\lh}  %left hook
{\mathbin{\mathpalette\blh\relax}}
\newcommand{\blh}[2]{\raisebox{\depth}{\scalebox{1}[-1]{$#1\lnot$}}} 





\titleformat{\section}[hang]{\centering\large\bfseries}{\thesection.}{1em}{}
\titleformat{\subsection}[runin]{\large\itshape}{- }{0em}{}[ -]

\fancyhf{} %these three lines put the page number at the bottom right
\rfoot{\thepage}
\renewcommand{\headrulewidth}{0pt}



\pagestyle{fancy}

\renewcommand{\qedsymbol}{$\clubsuit$}


\title{Algebra prelim theorems}




\begin{document}
\begin{center}
\textbf{\Large Algebra Preliminary Exam Notes}
\end{center}

This is a list of most of the definitions, theorems, and propositions contained within \textit{Abstract Algebra $3^{rd}$ edition} by Dummit and Foote as well as some extra useful ones. References made in red in this series of notes refer to the actual number of the theorem in the book.\\

\tableofcontents
\newpage


\section{Introduction to Groups}
\begin{thm}
Let $(G,\cdot)$ be a group, $H\seq G$. Then $(H,\cdot)$ is a subgroup of $G$ if
\begin{enumerate}
\item $H\neq \es$.
\item For all $a,b\in H$, $ab^{-1}\in H$.
\end{enumerate}
\end{thm}

\nl
\begin{defn}
Let $g\in G$ a group, then $g^{-1}$ is the \underline{\textit{unique}} element of $G$ such that $gg^{-1}=g^{-1}g=id$
\end{defn}

\nl
\begin{defn}
A \textbf{\textit{group action}} of a group $G$ on a set $A$ is a map, $G\times A$ to $A$ satisfying the following properties:
\begin{enumerate}
\item $g_1\cdot(g_2\cdot a) = (g_1g_2)\cdot a$, for all $g_1,g_2\in G$ and $a\in A$.
\item $id_G\cdot a = a$ for all $a\in A$.
\end{enumerate}
\end{defn}

%################################################################################

\section{Subgroups}

\setcounter{thm}{0}
\nl
\begin{defn}
Let $A\seq G$, $A\neq\es$. \hl{Define $C_G(A) = \{g\in G\ |\ gag^{-1} = a \text{ for all } a\in A\}$}. This subset of $G$ is called the \textbf{\textit{centralizer}} of $A$ in $G$. Since $gag^{-1} = a$ if and only if $ga = ag$, $C_G(A)$ is the set of elements of $G$ which commute with every element of $A$. When $A = G$ this set is denoted by $Z(G)$ and is called the \textit{\textbf{center}} of $G$.
\end{defn}
\nl

\textbf{Note:} $Z(G) \leq C_G(A)$ for all $A\seq G$.

\nl

\begin{defn}
Let $A\seq G$, $A\neq\es$. Define $gAg^{-1} = \{gag^{-1}\ |\ a\in A\}$. We define the \textit{\textbf{normalizer}} of $A$ in $G$ to be the set $N_G(A) = \{g\in G\ |\ gAg^{-1} = A\}$.
\end{defn}
\nl

\begin{defn}
Let $G$ be a group acting on a set $S$. The \textit{\textbf{stabilizer}} $G_s$ for some fixed $s\in S$ is the set 
\[G_s = \{g\in G\ |\ g\cdot s = s\}.\]
\end{defn}

\begin{prop}
Let $A$ be some set and $G$ be a group. Then $C_G(A)\leq N_G(A)$.
\end{prop}
\begin{proof}
$C_G(A)$ is the kernel of $N_G(A)$ acting on $A$ under the conjugation map $a\mapsto gag^{-1}$.
\end{proof}

\nl

\begin{prop}
Let $G$ be a group and $S\seq G$, $s\neq \es$. \hl{Then $N_G(S) \leq G$.}
\end{prop}

\begin{proof}
Let $G$ be a group and $S\seq G$. We know that $N_G(S) = \{g \in G\ |\ gSg\inv = S\}$. If we take $a, b\in N_G(S)$ we have that 
\begin{align*}
abSb\inv a\inv = aSa\inv = S
\end{align*}
so $ab \in N_G(S)$. Similarly we have that for $a\in N_G(S)$
\[a\inv Sa = a\inv(aSa\inv)a = S\]
so $a\inv \in N_G(S)$ and we have that $N_G(S) \leq G$.
\end{proof}

\nl
\begin{thm}
There is only one cyclic group of each order.
\end{thm}

\nl
\begin{prop}
Let $G$ be a group, let $x\in G$ and let $a\in \Z^\times$.
\begin{enumerate}
\item If $|x|=\infty$, then $|x^a| = \infty$.
\item If $|x| = n< \infty$, then $|x^a| = \frac{n}{\gcd(n,a)}$.
\end{enumerate}
\end{prop}

\nl
\begin{defn}
Let $A\seq G$ and define 
\[\langle A\rangle = \bigcap_{\substack{ A\seq H\\H\leq G}} H. \]
This is called the \textit{\textbf{subgroup of G generated by A}} and is simply the intersection of all the subgroups containing the set $A$.
\end{defn}


\nl

\hl{\textbf{Zorn's Lemma.}} If $A$ is a nonempty partially ordered set in which every chain has an upper bound then $A$ has a maximal element.

%################################################################################

\section{Quotient Groups and Homomorphisms}

\setcounter{thm}{0}
\begin{prop}
Let $G$ and $H$ be groups and let $\vphi: G\ra H$ be a homomorphism.
\begin{enumerate}
\item $\vphi(id_G) = id_H$
\item $\vphi(g^{-1}) = \vphi(g)^{-1}$
\item $\vphi(g^n)= \vphi(g)^n$
\item $\ker(\vphi)$ is a subgroup of $G$
\item $\im(\vphi)$ is a subgroup of $H$
\end{enumerate}
\end{prop}

\nl
\begin{prop}\label{coset_op}
Let $G$ be a group and let $N$ be a subgroup of $G$
\begin{enumerate}
\item The operation on the set of left cosets of $N$ in $G$ described by
\[uN\cdot vN = (uv)N\]
is sell defined if and only if $gng^{-1}\in N$ for all $g\in G$ and all $n\in N$.
\item If the above operation is well defined then it makes the set of left cosets of $N$ in $G$ into a group. In particular the identity of this group is the coset $id_G N$and the inverse of $gN$ is $g^{-1}N$.
\end{enumerate}
\end{prop}

\nl


\begin{defn}
The element $gng^{-1}$ is called the conjugate of $n\in N$ by $g$. The set $gNg^{-1}$ is also called the conjugate of $N$ by $g$. The element $g$ is said to \textbf{\textit{normalize}} $N$ if $gNg^{-1} = N$. A subgroup $N$ of $G$ is a \textbf{\textit{normal subgroup}} if every $g\in G$ normalizes $N$. We will write this as $N \unlhd G$.
\end{defn}

\nl

\begin{thm}
Let $N$ be a subgroup of $G$. The following are equivalent.
\begin{enumerate}
\item $N\unlhd G$
\item $N_G(N) = G$
\item $gN = Ng \quad \forall g\in G$
\item The operation on left cosets of $N$ in $G$ described by \autoref{coset_op} makes the set of left cosets into a group
\item $gNg^{-1} \seq N$ for all $g\in G$.
\end{enumerate}
\end{thm}

\hl{\textbf{Lagrange's Theorem.}} If $G$ is a finite group and $H$ is a subgroup of $G$, then the order of $H$ divides the order of $G$ and the number of left cosets of $H$ in $G$ equals $\frac{|G|}{|H|}$.

\hl{\textbf{Cauchy's Theorem.}} If $G$ s a finite group and $p$ is a prime dividing $|G|$ then $G$ has an element of order $p$.

\nl

\begin{defn}\hl{(Dedekind and Hamiltonian Groups)}
For any group $G$, if all the subgroups of $G$ are normal then $G$ is called a \textit{Dedekind} group. If $G$ is non-abelian then $G$ is called a \textit{Hamiltonian} group.
\end{defn}

\nl
\begin{thm}
If $G$ is a finite group of order $p^\al m$, where $p$ is a prime and $p$ does not divide $m$, then $G$ has a subgroup of order $p^\al$ (Proof will be done with the big Sylow theorem).
\end{thm}

\nl
\begin{defn}
Let $H$ and $K$ be subgroups of a group and define
\[HK = \{hk |\ h\in H,\ k\in K\}.\]
\end{defn}

\begin{prop}
If $H$ and $K$ are subgroups of a group then
\[|HK| = \frac{|H||K|}{|H\cap K|}.\]
\end{prop}

\begin{cor}
If $H$ and $K$ are subgroups of $G$ then $HK$ is a subgroup if $H$ normalizes $K$ (i.e. if $H\seq N_G(K)$).
\end{cor}


===============

\textbf{Isomorphism Theorems}

===============

\nl

\begin{thm}\textit{(First Isomorphism Theorem)}
If $\vphi:G \ra H$ is a homomorphism of groups, then $\ker(\vphi)\unlhd G$ and $G/\ker(\vphi) \cong \im(\vphi)$.
\end{thm}

\nl

\begin{cor}
Let $\vphi: G\ra H$ be a homomorphism of groups.
\begin{enumerate}
\item $\vphi$ is injective if and only if $\ker(\vphi) = id_G$.
\item $|G:\ker(\vphi)| = |\im(\vphi)|$.
\end{enumerate}
\end{cor}

\nl

\begin{thm}\textit{(The Second or Diamond Isomorphism Theorem)}
Let $G$ be a group, let $A$ and $B$ be subgroups of $G$ and assume $A\leq N_G(B)$. Then $AB$ is a subgroup of $G$, $B\ \unlhd\  AB,\ \ A\cap B\ \unlhd\  A$ and $AB/B\cong A/A\cap B$.
\end{thm}

\begin{center}
\begin{tikzcd}
& G \arrow[dash, d] & \\
& AB\arrow[dash, dr, "\unrhd"] \arrow[dash, dl] & \\
A\arrow[dash, dr, "\unrhd"] & & B\arrow[dash, dl] \\
& A\cap B\arrow[dash, d] & \\
& \{id_G\} &  \\
\end{tikzcd}
\end{center}


\begin{thm}\textit{(The Third Isomorphism Theorem)}
Let $G$ be a group and let $H$ and $K$ be normal subgroups of $G$ with $H\leq K$. Then $K/H\unlhd G/H$ and 
\[(G/H)/(K/H)\cong G/K\]
\end{thm}

\nl

\begin{thm}\textit{(The Fourth Isomorphism Theorem)}
Let $G$ be a group and let $N$ be a normal subgroup of $G$. The there is a bijection from the set $\CS$ of subgroups $A$ of $G$ which contain $N$ onto the set $\mathcal{T}$ of subgroups of the quotient group $G/N$. Specifically, there is a bijective map $\vphi:\CS \ra \mathcal{T}:A\mapsto A/N$ and we have the following:
\begin{enumerate}
\item $A\leq B$ if and only if $A/N\leq B/N$,
\item if $A\leq B$, then $|B:A| = |B/N:A/N|$,
\item $\langle A, B\rangle/N = \langle A/N, B/N\rangle$,
\item $(A\cap B)/N = A/N\cap B/N$, and 
\item $A\unlhd G$ if and only if $A/N \unlhd G/N$.
\end{enumerate}
\end{thm}


===============

===============

\begin{thm}\textit{(Feit-Thompson)}
If $G$ is a simple group of odd order, then $G\cong\Z/p\Z$ for some prime $p$.
\end{thm}

\nl

\begin{defn}
\hl{A group $G$ is \textbf{\textit{solvable}}} if there is a chain of subgroups 
\[1 = G_0\unlhd G_1\unlhd\cdots\unlhd G_n = G\]
such that $G_{i+1}/G_i$ is abelian for $i= 0,1,\ldots,n-1$.
\end{defn}

\nl

\begin{thm}
The finite group $G$ is solvable if and only if for every divisor $n$ of $|G|$ such that $\gcd\lp n,\frac{|G|}{n}\rp = 1$, $G$ has a subgroup of order $n$.
\end{thm}

\nl

\begin{defn}
The \textit{alternating group of degree n}, denoted by $A_n$, is the kernel of the sign homomorphism acting on $S_n$.
\end{defn}

\nl

\begin{prop}
The permutation $\sigma$ is odd if and only if the number of cycles of even length in its cycle decomposition is odd.
\end{prop}

%################################################################################

\section{Group Actions}

\setcounter{thm}{0}

\begin{defn}
Let $G$ be a group acting on a nonempty set $A$. For each $g\in G$ the map 
\[\sigma_g:A\ra A:a\mapsto g\cdot a\]
is a permutation of $A$. The homomorphism associated to an action of $G$ on $A$
\[\vphi: G\ra S_A:\vphi(g)\mapsto\sigma_g\]
is called the \textit{permutation representation} associated to the given action.
\end{defn}

\nl

\begin{defn}
Let $G$ be a group acting on a set $A$
\begin{enumerate}
\item The \textbf{\textit{kernel}} of the action is the set of elements of $G$ that act trivially on every element of $A$: $\{g\in G\ |\ g\cdot a = a\text{ for all }a\in A\}$.
\item Fro each $a\in A$ the \textbf{\textit{stabilizer}} of $a$ in $G$ is the set of elements of $G$ that fix the element $a$: $\{g\in G\ |\ g\cdot a =a\}$ and is denoted by $G_a$.
\item An action is \hl{\textbf{\textit{faithful}}} if its kernel is the identity.
\end{enumerate}
\end{defn}

\nl

\begin{cor}
Let $G$ be a group acting on a set $A$. Two elements of $G$ induce the same permutation on $A$ if and only if they are in the same coset.
\end{cor}

\nl

\begin{prop}
Let $G$ be a group acting on the nonempty set $A$. The relation on $A$ defined by
\[a\sim b\quad\text{if and only if}\quad a = g\cdot b\text{ for some } g\in G\]
is an equivalence relation. For each $a\in A$, the number of elements in the equivalence class containing $a$ is $|G:G_a|$, the index of the stabilizer of $a$.
\end{prop}

\nl

\begin{defn}
Let $G$ be a group acting on the nonempty set $A$.
\begin{enumerate}
\item The equivalence class $\{g\cdot a\ |\ g\in G\}$ is called the \textbf{\textit{orbit}} of $G$ containing $a$.
\item The action of $G$ on $A$ is called \hl{\textbf{\textit{transitive}}} if there is only one orbit, i.e., given any two elements $a,b\in A$ there is some $g\in G$ such that $a= g\cdot b$.
\end{enumerate}
\end{defn}

\nl


\begin{thm}
Let $G$ be a group, let $H$ be a subgroup of $G$ and let $G$ act by left multiplication on the set $A$ of left cosets of $H$ in $G$. Let $\pi_H$ be the associated permutation representation afforded by this action. Then
\begin{enumerate}
\item $G$ acts transitively on $A$
\item the stabilizer in $G$ of the point $1H\in A$ is the subgroup $H$
\item the kernel of the action (i.e., the kernel of $\pi_H$) is $\cap_{x\in G}\ xHx\inv$, and $\ker(\pi_H)$ \hl{is the largest normal subgroup of $G$ contained in $H$}.
\end{enumerate}
\end{thm}

\nl

\begin{cor}\hl{\textit{(Cayley's Theorem)}}
Every group is isomorphic to a subgroup of some symmetric group. If $G$ is of order $n$, then $G$ is isomorphic to a subgroup of $S_n$.
\end{cor}

\nl

\begin{cor}
Let $G$ be a simple, non-abelian group and let $H\leq G$. Then $G$ is isomorphic to a subgroup of the symmetric group on $G/H$, $Sym(G/H)$.
\end{cor}

\begin{proof}
Let $G$ be a simple, non-abelian group and let $H\leq G$. Suppose that $G$ acts on the coset space $G/H$ by left multiplication. Obviously, this action is transitive, so we have that there is a homomorphism
\[\vphi:G \ra Sym(G/H): g\mapsto \sig_g\]
where
\[\sig_g: G/H \ra G/H: xH\mapsto (g\cdot x)H.\]
Now, $H$ is a proper subgroup, so $|G/H| > 1$, and since $G$ acts transitively, we have that $\vphi$ is nontrivial. This gives us that $\ker(\vphi) \neq G$, and since $G$ is simple we get that $\vphi$ is injective.
\end{proof}

\nl

\begin{cor}
If $G$ is a finite group of order $n$ and $p$ is the smallest prime dividing $|G|$, then any subgroup of index $p$ is normal. (Note: this is used mostly with subgroups of index 2)
\end{cor}

\nl

\begin{defn}
Two elements $a$ and $b$ of $G$ are said to be \textbf{\textit{conjugate}} in $G$ if there is some $g\in G$ such that $b = gag\inv$. The orbits of $G$ acting on itself by conjugation are called \hl{\textit{conjugacy classes} of $G$.}
\end{defn}

\nl

\begin{defn}
Two subsets $S$ and $T$ of $G$ are said to be \textbf{\textit{conjugate in G}} if there is some $g\in G$ such that $T= gSg\inv$.
\end{defn}

\nl

\begin{prop}
\hl{The number of conjugates of a subset $S$ in a group $G$ is the index of the normalizer of $S$, $|G:N_G(S)|$. In particular, the number of conjugates of an element $s$ of $G$ is the index of the centralizer of $s$, $|G:C_G(s)|$.}
\end{prop}

\nl

\begin{thm}\hl{\textit{(The Class Equation)}}
Let $G$ be a finite group and let $g_1,g_2,\ldots,g_r$ be representatives of the distinct conjugacy classes of $G$ not contained in the center $Z(G)$ of $G$. Then
\[|G| = |Z(G)| + \sum_{i = 1}^r |G:C_G(g_i)|.\]
\end{thm}

\nl

\begin{thm}\hl{\textit{(Orbit Stabilizer Theorem)}}
Let $G$ be a group acting on a set $A$ and consider some $a\in A$. Then
\[|Orb(a)| = |G:Stab(a)|.\]
\end{thm}

\nl

\begin{thm}
\hl{Every normal subgroup is the union of conjugacy classes.}
\end{thm}

\nl 

\begin{defn}
Let $G$ be a group. An isomorphism from $G$ onto itself is called an \textbf{\textit{automorphism}}. The set of all automorphisms of $G$ is denoted by $\Aut(G)$.
\end{defn}

\nl

\begin{prop}
Let $H$ be a normal subgroup of $G$. Then $G$ acts by conjugation on $H$ as automorphisms of $H$. More specifically, the action of $G$ on $H$ by conjugation is defined for each $g\in G$ by 
\[h\mapsto ghg\inv\qquad\text{for each } h\in H.\]
For each $g\in G$, conjugation by $g$ is an automorphism of $H$. The permutation representation afforded by this action is a homomorphism of $G$ into $\Aut(H)$ with kernel $C_G(H)$. In particular, $G/C_G(H)$ is isomorphic to a subgroup of $\Aut(H)$.
\end{prop}

\nl

\begin{cor}
If $K$ is any subgroup of the group $G$ and $g\in G$, then $K\cong gKg\inv$. Conjugate elements and conjugate subgroups have the same order.
\end{cor}

\nl

\begin{cor}
For any subgroup $H$ of a group $G$ the quotient group $N_G(H)/C_G(H)$ is isomorphic to a subgroup of $\Aut(H)$. In particular, $G/Z(G)$ is isomorphic to a subgroup of $\Aut(G)$.
\end{cor}

\nl

\begin{defn}
Let $G$ be a group and let $g\in G$. Conjugation by $g$ is called an \textbf{\textit{inner automorphism}} of $G$ and the subgroup of $\Aut(G)$ consisting of all inner automorphisms is denoted by $\Inn(G)$.
\end{defn}

\nl

\textbf{Note:} For any group G we have that 
\[\Inn(G)\cong G/Z(G).\]
This is really useful when proving that $\Aut(G)$ is nontrivial.

\nl

\begin{defn}
A subgroup $H$ of a group $G$ is called \textbf{\textit{characteristic}} in G, denoted $H$ char $G$, if every automorphism of $G$ maps $H$ to itself, i.e., $\sigma(H) = H$ for all $\sigma\in Aut(G)$.
\end{defn}

\nl

\begin{prop}\textit{(Properties of Characteristic Subgroups)}
\begin{enumerate}
\item characteristic subgroups are normal
\item \hl{if $H$ is the unique subgroup of $G$ of a given order, then $H$ is characteristic in $G$}, and 
\item if $K$ char $H$ and $H\unlhd G$, then $K\unlhd G$.
\end{enumerate}
\end{prop}

\nl

\begin{prop}
The automorphism group of the cyclic group of order $n$ is isomorphic to $(\Z/n\Z)^\times$, and abelian group of order $\vphi(n)$  (where $\vphi$ is Euler's function).
\end{prop}

===============

\textbf{Sylow Theorems}

===============
\nl
\begin{defn}
Let $G$ be a group and let $p$ be a prime.
\begin{enumerate}
\item A group of order $p^\al$ for some $\al\geq 0$ is called a $p$\textbf{\textit{-group}}. Subgroups of $G$ which are $p$-groups are called $p$\textbf{\textit{-subgroups}}.
\item If $G$ is a group of order $p^\al m$, where $p\not |m$, then a subgroup of order $p^\al$ is called a \textbf{\textit{Sylow p-subgroup}} of $G$.
\item The set of Sylow $p$-subgroups of $G$ will be denoted by $Syl_p(G)$ and the number of Sylow $p$-subgroups of $G$ will be denoted by $n_p(G)$.
\end{enumerate}
\end{defn}

\nl

\begin{thm}\hl{\textit{(Sylow's Theorem)}}
Let $G$ be a group of order $p^\al m$, where $p$ is a prime not dividing $m$.
\begin{enumerate}
\item Sylow $p$-subgroups of $G$ exist.
\item If $P$ is a Sylow $p$-subgroup of $G$ and $Q$ is any $p$-subgroup of $G$, then there exists $g\in G$ such that $Q\leq gPg\inv$, i.e., $Q$ is contained in some conjugate of $P$. In particular, any two Sylow $p$-subgroups of $G$ are conjugate in $G$.
\item The number of Sylow $p$-subgroups in $G$ is of the form $1+kp$, i.e., 
\[n_p\equiv 1\mod p.\]
Further, $n_p$ is the index in $G$ of the normalizer $N_G(P)$ for any Sylow $p$-subgroup $P$, \hl{hence $n_p$ divides $m$}.
\end{enumerate}
\end{thm}

\nl

\begin{lem}
Let $P\in Syl_p(G)$. If $Q$ is any $p$-subgroup of $G$, then $Q\cap N_G(P)=  Q\cap P$.
\end{lem}

\nl

\begin{thm}
\hl{A nontrivial $p$-group has a nontrivial center.}
\end{thm}

\begin{proof}
Let $G$ be a nontrivial $p$-group, and $P$ the set of order-$p$ elements of $G$. We have seen that $P$ is nonempty, and indeed that $|P|$ is congruent to $-1 \mod p$. Now consider the action of $G$ on $P$ by conjugation. The stabilizer under this action of any $x$ in $P$ is the centralizer $C(x)$ of $x$, which is the subgroup of $G$ consisting of all elements that commute with $x$. The orbit of $x$ then has size $[G:C(x)]$. But $G$ is a $p$-group, so $[G:C(x)]$ is a power of $p$. Hence $[G:C(x)]$ is either 1 or a multiple of $p$. Since $|P|$ is not a multiple of $p$, it follows that at least one of the orbits is a singleton. Then $C(x)=G$, which is to say that $x$ commutes with every element of $G$. We have thus found a nontrivial element $x$ of the center of $G$.
\end{proof}

\nl

\begin{cor}
Let $P$ be a Sylow $p$-subgroup of $G$. Then the following are equivalent:
\begin{enumerate}
\item $P$ is the unique Sylow $p$-subgroup of $G$, i.e., $n_p = 1$
\item $P$ normal in $G$
\item $P$ is characteristic in $G$
\item All subgroups generated by elements of $p$-power order are $p$-groups, i.e., if $X$ is any subset of $G$ such that $|x|$ is a power of $p$ for all $x\in X$, then $\langle X\rangle$ is a $p$-subgroup.
\end{enumerate}
\end{cor}


%################################################################################

\section{Direct and Semidirect Products and Abelian Groups}

\setcounter{thm}{0}

\begin{prop}
Let $G_1, G_2, \ldots, G_n$ be groups and let $G = G_1\times G_2\times \cdots\times G_n$ be their direct product.
\begin{enumerate}
\item For each fixed $i$ the set of elements of $G$ which have the identity of $G_j$ in the $j^{th}$ position for all $j\neq i$ and arbitrary elements of $G_i$ in position $i$ is a subgroup of $G$ isomorphic to $G_i$:
\[G_i \cong \{(1,1,\ldots,1,g_i,1,\ldots,1)\ |\ g_i\in G_i\}.\]
If we identify $G_i$ with this subgroup, then \hl{$G_i \unlhd G$} and 
\[G/G_i\cong G_1\times\cdots\times G_{i-1}\times G_{i+1}\times \cdots\times G_n.\]
\item for each fixed $i$ define $\pi_i:G\ra G_i$ by
\[\pi_i((g_1,g_2,\ldots,g_n)) = g_i.\]
Then $\pi_i$ is a surjective homomorphism with
\begin{align*}
\ker(\pi_i) &= \{(g_1,\ldots,g_{i-1},1,g_{i+1},\ldots,1)\ |\ g_j\in G_j\text{ for all } j\neq i\}\\
&\cong G_1\times\cdots\times G_{i-1}\times G_{i+1}\times \cdots\times G_n.
\end{align*}
\item Under the identifications in part (1), if $x\in G_i$ and $y\in G_j$ then $xy = yx$.
\end{enumerate}
\end{prop}

\nl

\begin{defn}\nl
\begin{enumerate}
\item A group $G$ is \textit{finitely generated} if there is a finite subset $A$ of $G$ such that $G = \langle A\rangle.$
\item For each $r\in \Z$ with $r\geq 0$, let $\Z^r = \Z\times\Z\times\cdots\times\Z$ be the direct product of $r$ copies of the group $\Z$, where $\Z^0 = 1$. The group $\Z^r$ is called the \textit{free abelian group of rank r}.
\end{enumerate}
\end{defn}

\nl

\begin{thm}\hl{\textit{(Fundamental Theorem of Finitely Generated Abelian Groups)}}
Let $G$ be a finitely generated abelian group. Then
\begin{enumerate}
\item 
\[G\cong \Z^r\times Z_{n_1}\times Z_{n_2}\times\cdots\times Z_{n_s}\]
for some integers $r,n_1,n_2,\ldots,n_s$ satisfying the following conditions:
\begin{enumerate}
\item $r\geq 0$ and $n_j \geq 2$ for all $j$, and
\item $n_{i+1}\ |\ n_i$ for $1\leq i\leq s-1$.
\end{enumerate}
\item the expression in (1) is unique.
\end{enumerate}
\end{thm}

\nl

\begin{defn}
The integer $r$ in the previous theorem is called the \hl{\textit{free rank} or \textit{Betti number}} of $G$ and the integers $n_1, n_2,\ldots,n_s$ are called the \textit{invariant factors} of $G$. The description 
\[G\cong \Z^r\times Z_{n_1}\times Z_{n_2}\times\cdots\times Z_{n_s}\]
is called the \textit{invariant factor decomposition} of $G$.
\end{defn}

\nl

\begin{cor}
If $n$ is the product of distinct primes, then up to isomorphism the only abelian group of order $n$ is the cyclic group of order $n$, \hl{$\Z/n\Z = Z_n$}.
\end{cor}

\nl

\begin{thm}
Let $G$ be an abelian group of order $n > 1$ and let the unique factorization of $n$ distinct prime powers be
\[n = p_1^{\al_1}p_2^{\al_2}\cdots p_k^{\al_k}.\]
Then
\begin{enumerate}
\item $G\cong A_1\times A_2\times\cdots\times A_k$, where $|A_i| = p_i^{\al_i}$
\item for each $A\in \{A_1,A_2,\ldots,A_k\}$ with $|A| = p^\al$,
\[A\cong  Z_{p^{\be_1}}\times Z_{p^{\be_2}}\times\cdots\times Z_{p^{\be_t}}\]
with $\be_1\geq\be_2\geq\cdots\geq\be_t\geq 1$ and $\be_1+\be_2+\cdots+\be_t = \al$
\item the decomposition in (1) and (2) is unique.
\end{enumerate}
\end{thm}

\nl

\begin{defn}
The integers $p^{\be_j}$ described in the preceding theorem are called the \textit{elementary divisors} of $G$. The description of $G$ given in the first two parts of the previous theorem is called the \textit{elementary divisor decomposition} of $G$.
\end{defn}

\nl

\begin{prop}
Let $m,n\in \Z^+$
\begin{enumerate}
\item $Z_m\times Z_n\cong Z_{mn}$ if and only if $\gcd(m,n) = 1$.
\item If $n = p_1^{\al_1}p_2^{\al_2}\cdots p_k^{\al_k}$ then $Z_n \cong  Z_{p_1^{\al_1}}\times Z_{p_2^{\al_2}}\times\cdots\times Z_{p_k^{\al_k}}$
\end{enumerate}
\end{prop}

\nl

\begin{defn}
Let $G$ be a group, let $x,y\in G$ and let $A,B$ be nonempty subsets of $G$.
\begin{enumerate}
\item Define $[x,y] = x\inv y\inv xy$, called the \textit{commutator} of $x$ and $y$.
\item Define $[A,B] = \langle [a,b]\ |\ a\in A,\ b\in B\rangle$, the group generated by commutator of elements from $A$ and $B$.
\item Define $G^\p = \langle [x,y]\ |\ x,y\in G\rangle$, the subgroup of $G$ generated by the commutators of elements from $G$, called the \textit{commutator subgroup} of $G$.
\end{enumerate}
\end{defn}

\nl

\begin{prop}
\hl{Let $G$ be a group, let $x,y\in G$ and let $H\leq G$. Then}
\begin{enumerate}
\item $xy = xy[x,y]$.
\item $H\unlhd G$ if and only if $[H,G]\leq H$.
\item $\sigma([x,y]) = [\sigma(x), \sigma(y)]$ for any $\sigma \in \Aut(G)$, $G^\p$ char $G$, and $G/G^\p$ is abelian.
\item $G/G^\p$ is the largest abelian quotient of $G$ in the sense that if $H\unlhd G$ and $G/H$ is abelian, then $G^\p\leq H$. Conversely, if $G^\p\leq H$ and $H\unlhd G$, then $G/H$ is abelian.
\item If $\vphi:G\ra A$ is any homomorphism of $G$ into an abelian group $A$, then $\vphi$ factors through $G^\p$ i.e. $G^\p\leq\ker(\vphi)$ and the following diagram commutes
\end{enumerate}
\begin{center}
\begin{tikzcd}[column sep = 3em]
G\arrow[r]\arrow[rd, swap, "\vphi"] & G/H\arrow[d]\\
& A
\end{tikzcd}
\end{center}
\end{prop}

\nl

\begin{prop}
Let $H$ and $K$ be subgroup of the group $G$. The number of distinct ways of writing each element of the set $HK$ in the form $hk$, for some $h\in H$ and $k\in K$ is $|H\cap K|$. In particular, if $H\cap K = 1$, the each element of $HK$ can be written uniquely as a product $hk$, for some $h\in H$ and $k\in K$.
\end{prop}

\nl

\begin{thm}\hl{\textit{(Product Recognition)}} Suppose $G$ is a group with subgroups $H$ and $K$ such that
\begin{enumerate}
\item $H$ and $K$ are normal in $G$, and
\item $H\cap K = 1$.
\end{enumerate}
Then $HK\cong H\times K$.
\end{thm}

\nl

\begin{defn}
If $G$ is a group and $H$ and $K$ are normal subgroups of $G$ with $H\cap K = 1$ then we call $HK$ the \textit{internal direct product} of $H$ and $K$. We shall call $H\times K$ the \textit{external direct product} of $H$ and $K$ \hl{(Note: This difference purely determines the notation of the elements of the group as these two are isomorphic by the recognition theorem).}
\end{defn}

\nl

\begin{thm}\label{con. semi}
Let $H$ and $K$ be groups and let $\vphi$ be a homomorphism from $K$ into $\Aut(H)$. Let $\cdot$ denote the (left) action of $K$ on $H$ determined by $\vphi$. Let $G$ be the set of ordered pairs $(h,k)$ with $h\in H$ and $k\in K$ and define the following multiplication on $G$:
\[(h_1,k_1)(h_2,k_2) = (h_1(k_1\cdot h_2), k_1k_2).\]
\begin{enumerate}
\item This multiplication makes $G$ into a group of order $|H||K|$.
\item The sets $\{(h, 1)\ |\ h\in H\}$ and $\{(1,k)\ |\ k\in K\}$ are subgroups of $G$ and the maps $h\mapsto (h,1)$ for $h\in H$ and $k\mapsto (1,k)$ for $k\in K$ are isomorphisms of these subgroups with the groups $H$ and $K$ respectively:
\[H\cong \{(h, 1)\ |\ h\in H\}\quad\text{and}\quad K\cong\{(1,k)\ |\ k\in K\}.\]
\item $\wh H = \{(h, 1)\ |\ h\in H\}\unlhd G$
\item $\wh H\cap \wh K = 1$
\item for all $h\in \wh H$ and $k\in\wh K$, $khk\inv = k\cdot h = \vphi(k)(h)$.

\end{enumerate}
\end{thm}


\nl

\begin{defn}
Let $H$ and $K$ be groups and let $\vphi$ be a homomorphism from $K$ into $\Aut(H)$. The group described in \autoref{con. semi} is called the \textit{semidirect product} of $H$ and $K$ with respect to $\vphi$ and will be denoted $H\rtimes_\vphi K$ (or simply $H\rtimes K$).
\end{defn}

\nl

\begin{prop}
Let $H$ and $K$ be groups and let $\vphi:K\ra \Aut(H)$ be a homomorphism. Then the following are equivalent:
\begin{enumerate}
\item the identity (set) map between $H\rtimes K$ and $H\times K$ is a group homomorphism
\item $\vphi$ is the trivial homomorphism from $K$ into $\Aut(H)$
\item $K\unlhd H\rtimes K$.
\end{enumerate}
\end{prop}

\nl

\begin{thm}
Suppose $G$ is a group with subgroups $H$ and $K$ such that
\begin{enumerate}
\item $H$ and $K$ are normal in $G$, and
\item $H\cap K = 1$.
\end{enumerate}
Let $\vphi:K\ra \Aut(H)$ be the homomorphism defined by mapping $k\in K$ to the automorphism of left conjugation by $k$ on $H$. Then $HK\cong H\times K$. In particular, if $G=HK$ with $H$ and $K$ satisfying (1) and (2), then $G$ is the semidirect product of $H$ and $K$.
\end{thm}

\nl

\begin{defn}
Let $H$ be a subgroup of $G$. A subgroup $K$ is called a \textbf{\textit{compliment}} for $H$ in $G$ if $G = HK$ and $H\cap K = 1$.
\end{defn}

%################################################################################

\section{Futher Topics in Group Theory}
\setcounter{thm}{0}

\begin{defn}
A \textbf{\textit{maximal subgroup}} of a group $G$ is a proper subgroup $M$ of $G$ such that there are no subgroups $H$ of $G$ such that $M<H<G$.
\end{defn}

\nl

\begin{thm}
Let $p$ be a prime and let $P$ be a group of order $p^a$, $a\geq 1$. Then
\begin{enumerate}
\item The center of $P$ is nontrivial.
\item If $H$ is a nontrivial normal subgroup of $P$ then $H$ intersects the center non-trivially. In particular, every subgroup of order $p$ is contained in the center.
\item If $H$ is a normal subgroup of $P$ then $H$ contains a subgroup of order $p^b$ that is normal in $P$ for each divisor $p^b$ of $|H|$. \hl{In particular, $P$ has a normal subgroup of order $p^b$ for every $b\in \{1,2,\ldots,a\}$}.
\item Let $H< P$ then $H<N_P(H)$.
\item Every maximal subgroup of $P$ is of index $p$ and is normal in $P$.
\end{enumerate}
\end{thm}

\nl

\begin{defn}\nl
\begin{enumerate}
\item For any (finite or infinite) group $G$ define the following subgroups inductively
\[Z_0(G) = 1,\qquad Z_1(G)= Z(G)\]
and $Z_{i+1}(G)$ is the subgroup of $G$ containing $Z_i(G)$ such that
\[Z_{i+1}(G)/Z_i(G) = Z(G/Z_i(G))\]
(i.e. $Z_{i+1}(G)$ is the complete preimage in $G$ of the center of $G/Z_i(G)$ under the natural projection). The chain of subgroups
\[Z_0(G)\leq Z_1(G)\leq Z_2(G)\leq\cdots\]
is called the\textbf{ \textit{upper central series of $G$}}.
\item A group $G$ is called \textit{\textbf{nilpotent}} if $Z_c(G) = G$ for some $c\in \Z$. The smallest such $c$ is called the \textit{nilpotence class of G}.
\end{enumerate}
\end{defn}

\nl

\begin{prop}
Let $p$ be a prime and let $P$ be a group of order $p^a$. Then $P$ is nilpotent of nilpotence class at most $a-1$ for $a\geq 2$.
\end{prop}

\begin{proof}
For each $i\geq 0,\ P/Z_i(P)$ is a $p$-group, so if 
\[|P/Z_i(P)| > 1\text{ then } Z(P/Z_i(P)\neq 1\]
by \textcolor{red}{Theorem 6.1 (1)}. Thus if $Z_i(P)\neq P$ then we have that $|Z_{i + 1}(P) \geq p|Z_i(P)|$and so $|Z_{i+1}(P)\geq p^{i+1}$. In particular $|Z_a(P)|\geq p^a$, so $P = Z_a(P)$. The only way $P$ could be of nilpotence class exactly equal to $a$ would be if $|Z_i(P)| = p^i$ for all $i$. In this case, however, $Z_{a-2}$ would have index $p^2$ in $P$, so $P/Z_{a-2}(P)$ would be abelian by \textcolor{red}{Corollary 4.9}. But then $P/Z_{a-1}(P)$ would equal its center and so $Z_{a-1}(P)$ would equal $P$ $\lightning$. This proves that the class of $P$ is $\leq a-1$.
\end{proof}

\nl

\begin{thm}
Let $G$ be a finite group, let $p_1, p_2, \ldots,p_s$ be the distinct primes dividing the order, and let $P_i\in Syl_{p_i}(G),\ 1\leq i\leq s$. Then the following are equivalent:
\begin{enumerate}
\item \hl{$G$ is nilpotent}
\item if $H<G$ then $H<N_G(H)$
\item $P_i\unlhd G$ for $1\leq i\leq s$, \hl{i.e., every Sylow subgroup is normal in $G$}
\item $G\cong P_1\times P_2\times \cdots\times P_s$.
\end{enumerate}
\end{thm}

\nl

\begin{cor}
A finite abelian group is the direct product of its Sylow subgroups (all abelian groups are nilpotent of rank 1).
\end{cor}


\nl

\begin{prop}
If $G$ is a finite group such that for all positive integers $n$ dividing its order, $G$ contains at most $n$ elements $x$ satisfying $x^n = 1$, then $G$ is cyclic.
\end{prop}

\nl

\begin{prop}\textit{(Frattini's Argument)}
Let $G$ be a group, let $H$ be a normal subgroup of $G$, and let $P\in Syl_p(H)$. Then $G=HN_G(P)$ and $|G:H|$ divides $|N_G(P)|$.
\end{prop}

\nl

\begin{prop}
A finite group is nilpotent if and only if every maximal subgroup is normal.
\end{prop}

\nl

\begin{defn}
For any (finite or infinite) group $G$ define the following subgroups inductively:
\[G^0 = G,\qquad G^1 = [G,G],\quad\text{and}\quad G^{i+1} = [G,G^i].\]
The chain of groups 
\[G^0\geq G^1\geq G^2\geq\cdots\]
is called the \textbf{\textit{lower central series of $G$}}.
\end{defn}

\nl

\begin{thm}
A group $G$ is nilpotent if and only if $G^n = 1$ for some $n\geq 0$. More precisely, $G$ is nilpotent of class $c$ if and only if $c$ is the smallest nonnegative integer such that $G^c = 1$. If $G$ is nilpotent of class $c$ then
\[G^{c-1} \leq Z_i(G)\quad\text{for all } i\in \{0,1,\ldots,c\}.\]
\end{thm}

\nl

\begin{defn}
For any group $G$ define the following sequence of subgroups inductively:
\[G^{(0)} = G,\qquad G^{(1)} = [G,G],\quad\text{and}\quad G^{(i+1)} = [G^{(i)},G^{(i)}]\quad\text{for all }i\geq 1.\]
This series of subgroups is called the \textit{derived} or \textit{commutator} series of $G$.
\end{defn}

\nl

\begin{thm}
A group $G$ is solvable if and only if $G^{(n)} = 1$ for some $n\geq 0$.
\end{thm}

\begin{proof}
Assume that $G$ is solvable and so possesses a series 
\[1 = H_0\unlhd H_1\unlhd \cdots\unlhd H_s = G\]
such that each factor $H_{i+1}, H_i$ is abelian. We prove by induction that $G^{(i)}\leq H_{s-i}$. This is true for $i = 0$, so assume that $G^{(i)}\leq H_{s-i}$. Then
\[G^{(i+1)} = [G^{(i)},G^{(i)}] \leq [H_{s-i},H_{s-i}].\]
Since $G$ is solvable, we know that $H_{s-i}/H_{s-i-1}$ is abelian. Moreover, $[H_{s-i},H_{s-i}]$ is the commutator subgroup of $H_{s-1}$, so $H_{s-i}/[H_{s-i},H_{s-i}]$ is the largest abelian quotient of $H_{s-i}$ which gives us that $[H_{s-i},H_{s-i}]\leq H_{s-i-1}$. Thus $G^{(i+1)} [H_{s-i},H_{s-i}]\leq H_{s-i-1}$. Since $H_0 = 1$, we have that $G^{(s)} = 1$.

Conversely, if $G^{(n)} = 1$ for some $n\geq 0$ then if we take $H_i = G^{(n - i)}$ we have $H_i$ is the largest abelian quotient of $H_{i+1}$. Thus the commutator series satisfies the condition for solvability.
\end{proof}

\nl

\begin{prop}
Let $G$ and $K$ be groups, let $H$ be a subgroup of $G$, and let $\vphi:G\ra K$ be a surjective homomorphism.
\begin{enumerate}
\item $H^{(i)}\leq G^{(i)}$ for all $i\geq 0$. In particular, if $G$ is solvable, then so is $H$. 
\item $\vphi(G^{(i)}) = K^{(i)}$. In particular, homomorphic images and quotient groups of solvable groups are solvable.
\item If $N\unlhd G$ and both $N$ and $G/N$ are solvable then so is $G$.
\end{enumerate}
\end{prop}

\nl

\begin{thm}
Let $G$ be a finite group.
\begin{enumerate}
\item (Burnside) If $|G| = p^aq^b$ for some primes $p$ and $q$, then $G$ is solvable.
\item (Phillip Hall) If for every prime $p$ dividing $|G|$ we factor the order of $G$ as $|G| = p^a m$ where $\gcd(p,m) = 1$, and $G$ has a subgroup of order $m$, then $G$ is solvable.
\item (Feit-Thompson) If $|G|$is odd then $G$ is solvable.
\item (Thompson) If for every pair of elements $x,y\in G,$ $\langle x,y\rangle$ is a solvable group, then $G$ is solvable.
\end{enumerate}
\end{thm}

\newpage

\subsection{Free Groups}\nl

The basic idea behind a free group $F(S)$ generated by a set $S$ is that there are no relations satisfied by any of the elements of $S$ (in this sense $S$ can be considered "free" of relations). Now, if we let $S$ be an arbitrary set then a \textit{\textbf{word}} in $S$ is a finite sequence of elements of $S$. We can then define $F(S)$ to simply be the set of all words in $S$. We shall use this idea to carry out a formal construction of $F(S)$ for an arbitrary $S$ below.

One of the important properties that reflects the fact that there are no relations that must be satisfied by members of $S$ is that any \textit{map} from the set $S$ to a group $G$ can be \textit{\textbf{uniquely extended}} to a homomorphism from the group $F(S)$ to $G$. This is called the \textit{\textbf{universal property}} of the free group and is what characterizes the group $F(S)$.

\begin{center}
\begin{tikzcd}[column sep = 2cm, row sep = 0.8cm]
S\arrow[r, "inclusion"] \arrow[dr, swap, "\vphi"] & F(S)\arrow[d, "\Phi"]\\
& G
\end{tikzcd}
\end{center}

Now, the difficulty in the construction of $F(S)$ is the proof that the word concatenation operation is both well defined and associative. If we say that $S$ is given as a set of literals, then we can define a set $S\inv$ such that there is a bijection from the set $S$ to the set $S\inv$ as given by sending $s\in S$ to its corresponding $s\inv \in S\inv$. If we then take some singleton set that is not contained in either $S$ or $S\inv$ and call it $\{1\}$. If we then join these sets we can take any $x\in S\cup S\inv\cup\{1\}$ and declare that $x^1 = x$. This allows us to think of words of $S$ as finite products of members of $S$ and their inverses. A word $s = (s_1,s_2,s_3,\ldots)$ is then said to be \textit{reduced} if 
\begin{enumerate}
\item $s_{i + 1} \neq s_i\inv$ for all $i$ with $s_i\neq 1$
\item if $s_k = 1$ for some $k$, then $s_i = 1$ for all $i\geq k$
\end{enumerate}

The reduced word $(1,1,1,\ldots)$ is called the \textit{empty word} and is denoted by 1. If we let $F(S)$ be the set of reduced words on $S$ then we can embed $S$ into $F(S)$ by 
\[s\mapsto (s,1,1,1,\ldots ).\]
Under this set injection we identify $S$ with its image and henceforth consider $S$ as a subset of $F(S)$. We can then introduce a binary operation on the set $F(S)$ to the tune of word concatenation followed by reduction (this is pretty self-explanatory), and with the introduction of this operation we get our first theorem of this section.

\nl

\begin{thm}
$F(S)$ is a group under the binary operation given above.
\end{thm}

\nl

\begin{thm}
Let $G$ be a group, $S$ a set and $\vphi:S\ra G$ a set map. Then there is a unique group homomorphism $\Phi:F(S)\ra G$ such that the following diagram commutes:
\begin{center}
\begin{tikzcd}[column sep = 2cm, row sep = 0.8cm]
S\arrow[r, "inclusion"] \arrow[dr, swap, "\vphi"] & F(S)\arrow[d, "\Phi"]\\
& G
\end{tikzcd}
\end{center}
\end{thm}

\begin{proof}
If such a map were to exist, then $\Phi$ must satisfy $\Phi(s_1^{\varepsilon_1}s_2^{\varepsilon_2}\cdots s_n^{\varepsilon_n}) = \vphi(s_1)^{\varepsilon_1}\vphi(s_2)^{\varepsilon_2}\cdots \vphi(s_n)^{\varepsilon_n}$ if it is so be a homomorphism (which gives us uniqueness), and the fact that this actually is a homomorphism follows almost directly.
\end{proof}

\nl

\begin{defn}
The group $F(S)$ is called the \textit{free group} on the set $S$. A group $F$ is a \textit{free group} if there is some set $S$ such that $F= F(S)$ -- in this case we call $S$ the set of \textit{free generators} of $F$. The cardinality of $S$ is called the \textit{rank} of the free group.
\end{defn}

\nl

\begin{defn}
Let $S$ be a subset of a group $G$ such that $G = \langle S\rangle$.
\begin{enumerate}
\item A \textit{\textbf{presentation}} for $G$ is a pair $(S,R)$, where $R$ is a set of words in $F(S)$ such that the normal closure of $\langle R\rangle$ in $F(S)$ (the smallest normal subgroup containing $\langle R\rangle$) equals the kernel of the homomorphism $\pi:F(S)\ra G$ (where $\pi$ extends the identity map from $S$ to $S$). The elements of $S$ are called \textit{generators} and those of $R$ are called \textit{relations} of $G$.
\item We say $G$ is \textit{finitely generated} if there is a presentation $(S,R)$ such that $S$ is a finites set and we say $G$ is \textit{finitely presented} if there is a presentation $(S,R)$ with both $S$ and $R$ finite sets.
\end{enumerate}
\end{defn}


%################################################################################

\section{Introduction to Rings}

\setcounter{thm}{0}

\begin{defn}\nl
\begin{enumerate}
\item a ring $R$ is a set together with two binary operations $+$ and $\times$ satisfying the following axioms
\begin{enumerate}
\item $(R, +)$ is an abelian group 
\item $\times$ is associative 
\item the distributive laws hold in $R$
\end{enumerate}
\item The ring $R$ is commutative if $\times$ is commutative
\item The ring $R$ is said to have identity if there is an element $1\in R$.
\end{enumerate}

\end{defn}

\nl

\begin{defn}
A ring with identity $R$ is said to be a \textit{division ring} if very nonzero element has a multiplicative inverse. A commutative division ring is called a \textit{field}.
\end{defn}

\nl

\begin{defn}\nl
\begin{enumerate}
\item A nonzero element $a$ of $R$ is called a \textit{zero divisor} if there is a nonzero element $b\in R$ such that $ab = 0$ or $ba = 0$.
\item Assume that $R$ has identity $1\neq 0$. An element $u$ of $R$ is called a \textit{\textbf{unit}} in $R$ if there is some $v$ in $R$ such that $uv=vu=1$. The set of units is denoted $R^\times$.
\end{enumerate}
\end{defn}

\nl

\begin{defn}
\hl{A commutative ring with identity is called an \textit{\textbf{integral domain}} if it has no zero divisors.}
\end{defn}

\nl

\begin{prop}
Assume that $a,b,$ and $c$ are elements of any ring with $a$ not a zero divisor. If $ab=ac$ then either $a=0$ or $b=c$.
\end{prop}

\nl

\begin{cor}
Any finite integral domain is a field.
\end{cor}

\nl

\begin{defn}
A \textit{subring} of the ring $R$ is \hl{a subgroup of $R$ that is closed under multiplication.}
\end{defn}

\nl

\begin{prop}
Let $R$ be an integral domain and let $p(x), q(x)$ be nonzero elements of $R[x]$. Then
\begin{enumerate}
\item $deg(p(x)q(x)) = deg (p(x)) + deg(q(x))$,
\item the units of $R[x]$ are just the units of $R$,
\item $R[x]$ is an integral domain.
\end{enumerate}
\end{prop}

\nl

\begin{defn}
Let $R$ and $S$ be rings.
\begin{enumerate}
\item A \textit{ring homomorphism} is a map $\vphi: R\ra S$ satisfying
\begin{enumerate}
\item $\vphi(a+b) = \vphi(a)+\vphi(b)$ for all $a,b\in R$, and 
\item $\vphi(ab) = \vphi(a)\vphi(b)$ for all $a,b\in R$
\end{enumerate}
\item The \textit{kernel} of the ring homomorphism $\vphi$ is the set of elements that map to $0_S$.
\item A bijective ring homomorphism is called an isomorphism.
\end{enumerate}
\end{defn}

\nl

\begin{prop}
Let $R$ and $S$ be rings and let $\vphi:R\ra S$ be a homomorphism.
\begin{enumerate}
\item The image of $\vphi$ is a subring of $S$.
\item The kernel of $\vphi$ is a subring of $R$. Furthermore, \hl{if $\al\in\ker(\vphi)$ then $r\al$ and $\al r$ are in $\ker(\vphi)$ for every $r\in R$.}
\end{enumerate}
\end{prop}

\nl

\begin{defn}
Let $R$ be a ring, let $I$ be a subset of $R$ and let $r\in R$.
\begin{enumerate}
\item $rI = \{ra\ |\ a\in I\}$
\item A subset $I$ of $R$ is a \textit{\textbf{left ideal}} of $R$ if 
\begin{enumerate}
\item $I$ is a subring of $R$, and
\item $I$ is closed under left multiplication by elements from $R$, i.e., $rI\seq I$ for all $r\in R$.
\end{enumerate}
There is a similar definition for a right ideal.
\item A subset $I$ that is both a left ideal and a right ideal is called an ideal of $R$.
\end{enumerate}
\end{defn}

\nl

\begin{prop}
Let $R$ be a ring and let $I$ be an ideal of $R$. Then the (additive) quotient group $R/I$ is a ring under the binary operations:
\[(r+I)+(s+I) = (r+s)+I\qquad\text{and}\qquad(r+I)\times(s+I)=(rs)+I\]
for all $r,s\in R$. Conversely if $I$ is any subgroup such that the above operations are well defined, then $I$ is an ideal of $R$.
\end{prop}

\nl

\begin{defn}
When $I$ is an ideal of $R$ the ring $R/I$ with the operations in the previous proposition is called the \textit{\textbf{quotient ring}} of $R$ by $I$.
\end{defn}

\nl

\begin{thm}\nl
\begin{enumerate}
\item \textit{(The First Isomorphism Theorem for Ring)} If $\vphi:R\ra S$ is a homomorphism of rings, then the kernel of $\vphi$ is an ideal of $R$, the image of $\vphi$ is a subring of $S$, and $R/\ker(\vphi)$ is isomorphic as a ring to $\vphi(R)$.
\item If $I$ is any ideal of $R$, then the map 
\[R\ra R/I\qquad\text{defined by}\qquad r\mapsto r+I\]
is a surjective homomorphism with kernel $I$. Thus every ideal is the kernel of a ring homomorphism and vice versa.
\end{enumerate}
\end{thm}

\nl

\begin{thm}\nl
\begin{enumerate}
\item \textit{(The Second Isomorphism Theorem for Rings)} Let $A$ be a subring and let $B$ be an ideal of $R$. Then $A+B = \{a + b\ |\ a\in A, \ b\in B\}$ is a subring of $R$, $A\cap B$ is an ideal of $A$, and $(A+B)/B\cong A/(A\cap B)$.
\item \textit{(The Third Isomorphism Theorem for Rings)} Let $I$ and $J$ be ideals of $R$ with $I\seq J$. Then $J/I$ is an ideal of $R/I$ and $(R/I)/(J/I)\cong R/J$.
\item \textit{(The Fourth or Lattice Isomorphism Theorem for Rings)} Let $I$ be an ideal of $R$. The correspondence $A\leftrightarrow A/I$ is an inclusion preserving bijection between the set of subrings $A$ of $R$ that contain $I$ and the set of subrings of $R/I$ Furthermore, $A$ is an ideal of $R$ if and only if $A/I$ is an ideal of $R/I$.
\end{enumerate}
\end{thm}

\nl

\begin{defn}
Let $R$ be a ring. Then the \hl{\textit{\textbf{characteristic}}} of the ring $R$ is the smallest number $n$ such that $n1 = 1+1+1+\cdots+1 = 0$. If this never happens, then the characteristic of $R$ is said to be $0$.
\end{defn}

\nl

\begin{prop}
\hl{Let $R$ be an integral domain. Then $char(R)$ is either prime or 0.}
\end{prop}

\nl

\begin{defn}
Let $A$ be any subset of the ring $R$.
\begin{enumerate}
\item Let $(A)$ denote the smallest ideal of $R$ containing $A$, called \textbf{\textit{the ideal generated by $A$}}.
\item Let $RA$ denote the set of all finite sums of elements of the form $ra$ with $r\in R$ and $a\in A$.
\item \hl{An ideal generated by a single element is called a \textit{\textbf{principal ideal}}.}
\item An ideal generated by a finite set is called a \textbf{\textit{finitely generated ideal}}.
\end{enumerate}
\end{defn}

\nl

\begin{prop}
Let $I$ be an ideal of $R$.
\begin{enumerate}
\item \hl{$I = R$ if and only if $I$ contains a unit.}
\item Assume $R$ is commutative. Then $R$ is a field if and only if its only ideals are $0$ and $R$.
\end{enumerate}
\end{prop}

\nl

\begin{cor}
If $R$ is a field then any nonzero ring homomorphism from $R$ into another ring is an injection (the kernel of the ring homomorphism is an ideal).
\end{cor}

\nl

\begin{defn}
An ideal $M$ in an arbitrary ring $S$ is called a \textbf{\textit{maximal ideal}} if $M\neq S$ and the only ideals containing $M$ are $M$ and $S$.
\end{defn}

\nl

\begin{prop}
In a ring with identity every proper ideal is contained in a maximal ideal. [NB: This is important because this means ideals in a ring with identity satisfy the ascending chain condition. This becomes really important in the study of infinite rings like the power series ring $\Z\lbb x\rbb$.]
\end{prop}

\nl

\begin{prop}
\hl{Assume $R$ is commutative. The ideal $M$ is maximal if and only if the quotient ring $R/M$ is a field.}
\end{prop}

\nl

\begin{defn}
Assume $R$ is commutative. An ideal $P$ is called a \hl{\textit{\textbf{prime ideal}}} if $P\neq R$ and whenever the product $ab$ of two elements $a,b\in R$ is an element of $P$, then at least one of $a$ and $b$ is an element of $P$.
\end{defn}

\nl

\begin{prop}
\hl{Assume $R$ is commutative. Then the ideal $P$ is a prime ideal in $R$ if and only if the quotient ring $R/P$ is an integral domain.}
\end{prop}

\nl

\begin{cor}
Assume $R$ is commutative. Every maximal ideal of $R$ is a prime ideal.
\end{cor}

\nl

\begin{thm}
Let $R$ be a commutative ring. Let $D$ be any nonempty subset of $R$ that does not contain 0, does not contain any zero divisors, and is closed under multiplication. Then there is a commutative ring $Q$ with 1 such that $Q$ contains $R$ as a subring and every element of $D$ is a unit in $Q$. The ring $Q$ has the following additional properties:
\begin{enumerate}
\item every element of $Q$ is of the form $rd\inv$ for some $r\in R$ and $d\in D$. In particular, if $D = R\backslash\{0\}$ then $Q$ is a field.
\item (uniqueness of $Q$) The ring $Q$ is the \textit{"smallest"} ring containing $R$ in which all the elements of $D$ become units, in the following sense. Let $S$ be any commutative ring with identity and let $\vphi:R\ra S$ be any injective ring homomorphism such that $\vphi(d)$ is a unit in $S$ for every $d\in D$. Then there is an injective homomorphism $\Phi:Q\ra S$ such that $\Phi|_R = \vphi$. In other words, any ring containing an isomorphic copy of $R$ in which all the elements of $D$ become units must also contain an isomorphic copy of $Q$.
\end{enumerate}
\end{thm}

\nl

\begin{defn}
Let $R$, $D$, and $Q$ be as in the above theorem.
\begin{enumerate}
\item The ring $Q$ is called the \textit{\textbf{ring of fractions}} of $D$ with respect to $R$ and is denoted $D\inv R$.
\item If $R$ is an integral domain and $D = R\backslash\{0\}$, $Q$ is called the \textit{\textbf{field of fractions}} or \textit{quotient field} of $R$.
\end{enumerate}
\end{defn}

\nl

\begin{cor}
Let $R$ be an integral domain and let $Q$ be the field of fractions of $R$. If a field $F$ contains a subring $R^\p$ isomorphic to $R$ then the subfield of $F$ generated by $R^\p$ is isomorphic to $Q$.
\end{cor}

\nl

\begin{defn}
The ideals $A$ and $B$ of the ring $R$ are said to be \textbf{\textit{comaximal}} if $A + B = R$.
\end{defn}

\nl

\begin{thm}\hl{\textit{(Chinese Remainder Theorem)}} Let $A_1,A_2,\ldots,A_k$ be ideals in $R$. The map
\[R\ra R/A_1\times R/A_2\times\cdots\times R/A_k\quad\text{defined by}\quad r\mapsto(r+A_1,r+A_2,\ldots,r+A_k)\]
is a ring homomorphism with kernel $\cap A_i$. If for each $i,j\in\{1,2,\ldots,k\}$ with $i\neq j$ the ideals $A_i$ and $A_j$ are comaximal, then this map is surjective and $A_1\cap A_2\cap\cdots\cap A_k = A_1A_2\cdots A_k$, so
\[R/(A_1A_2\cdots A_k) = R/(A_1\cap A_2\cap\cdots\cap A_k) \cong R/A_1\times R/A_2\times\cdots\times R/A_k.\]
\end{thm}

\nl

\begin{cor}
Let $n$ be a positive integer and let $p_1^{\al_1}p_2^{\al_2}\cdots p_k^{\al_k}$ be its factorization into powers of distinct primes. Then
\[Z/n\Z \cong (\Z/p_1^{\al_1}\Z)\times(\Z/p_2^{\al_2}\Z)\times\cdots\times (\Z/p_k^{\al_k}\Z),\]
as rings, so in particular we have the following isomorphism of multiplicative groups:
\[(Z/n\Z)^\times \cong (\Z/p_1^{\al_1}\Z)^\times\times(\Z/p_2^{\al_2}\Z)^\times\times\cdots\times (\Z/p_k^{\al_k}\Z)^\times.\]
\end{cor}

\nl

\begin{cor}
Let $a,b\in \Z$ then
\[\Z/(m)\times \Z/(n) \cong \Z/(\gcd(m,n))\times \Z/(\text{lcm}(m,n))\]
\end{cor}

\begin{proof}(copied from math.stackexchange)
Fix $u,v\in\Bbb Z$ with $un+vm=d$ (Bezout). 
The map $$\Bbb Z_{\operatorname{lcm}(n,m)}\times\Bbb Z_{\gcd(n,m)} \to\Bbb Z_m\times\Bbb Z_n$$
$$ (a+\operatorname{lcm}(n,m)\Bbb Z,b+\gcd(n,m)\Bbb Z)\mapsto(ua+\tfrac mdb+m\Bbb Z,va-\tfrac ndb+n\Bbb Z)$$
is well-defined(!) and clearly a group homomorphism.
For the element on the left to be in the kernel, 
$ua+\tfrac mdb$ must be a multiple of $m$ and $va-\tfrac ndb$ a multiple of $n$.
But then
$$\frac nd\left(ua+\frac mdb\right)+\frac md\left(va-\frac ndb\right) 
=\frac{nu+vm}{d}a=a$$ 
is a multiple of $\frac{nm}d=\operatorname{lcm}(n,m)$, i.e., we may as well assume that $a=0$. Then $\frac mdb$ must be a multiple of $m$, i.e., $b$ a multiple of $d$, i.e. $b\equiv 0$. We conclude that the kernel is trivial and our homomorphism injective. As both groups are finite of same order, the homomoprhism must be an isomorphism.
\end{proof}

\section{Euclidean Domains, Principal Ideal Domains, and Unique Factorization Domains}

\hl{All rings in this section are commutative.}

\nl

\begin{defn}
Any function $N:R\ra \Z_{\geq 0}$ with $N(0) = 0$ is called a \textit{\textbf{norm}} on the integral domain $R$. If $N(a)> 0$ for all $a\neq 0$ define $N$ to be a \textit{positive norm}.
\end{defn}

\nl

\begin{defn}
The integral domain $R$ is said to be a \hl{\textit{\textbf{Euclidean Domain}}} if there is a norm $N$ on $R$ such that for any two elements $a$ and $b$ of $R$ with $b\neq 0$ there exist elements $q$ and $r$ in $R$ with 
\[a = qb + r\qquad \text{with } r = 0 \text{ or } N(r)<N(b).\]
\end{defn}

\nl

\begin{defn}
Let $R$ be a commutative ring and let $a,b\in R$ with $b\neq 0$.
\begin{enumerate}
\item $a$ is said to be a \textbf{\textit{multiple}} of $b$ if $a = bx$ for some $x\in R$. In this case $b$ is said to divide or be a divisor of $a$, written $b\ |\ a$.
\item A \textbf{\textit{greatest common divisor}} of $a$ and $b$ is a nonzero element $d$ such that 
\begin{enumerate}
\item $d\ |\ a$ and $d\ |\ b$, and 
\item if $d^\p\ |\ a$ and $d^\p\ |\ b$ then $d\ |\ d^\p$.
\end{enumerate}
A greatest common divisor of $a$ and $b$ will be denoted by $\gcd(a,b)$.
\end{enumerate}
\end{defn}

\nl

\begin{prop}
If $a$ and $b$ are nonzero elements in the commutative ring $R$ such that the ideal generated by $a$ and $b$ is a principal ideal $(d)$, then $d$ is a greatest common divisor of $a$ and $b$.
\end{prop}

\nl

\begin{prop}
Let $R$ be an integral domain. If two elements $d$ and $d^\p$ of $R$ generate the same principal ideal, then $d^\p = ud$ for some unit $u\in R$. In particular, if $d$ and $d^\p$ are both greatest common divisors of $a$ and $b$, then $d^\p = ud$ for some unit $u$.
\end{prop}

\nl

\begin{thm}
Let $R$ be a Euclidean Domain and let $a$ and $b$ be nonzero elements of $R$. Let $d = r_n$ be the last nonzero remainder in the Euclidean Algorithm for $a$ and $b$. Then
\begin{enumerate}
\item $d$ is a greatest common divisor of $a$ and $b$, and 
\item the principal ideal $(d)$ is the ideal generated by $a$ and $b$. In particular, $d$ can be written as an $R$\textbf{\textit{-linear combination}} of $a$ and $b$, i.e., there are elements $x$ and $y$ in $R$ such that 
\[d = ax + by.\]
\end{enumerate}
\end{thm}

\nl

\begin{defn}
A domain $R$ in which every ideal is principal is called a \textbf{\textit{Principal Ideal Domain}} (PID).
\end{defn}

\nl

\begin{prop}
Let $R$ be a PID and let $a$ and $b$ be nonzero elements of $R$. Let $d$ be a generator for the principal ideal generated by $a$ and $b$. Then
\begin{enumerate}
\item $d$ is a greatest common divisor of $a$ and $b$
\item $d$ can be written as an $R$-\textit{linear combination} of $a$ and $b$, i.e., there are elements $x$ and $y$ in $R$ with 
\[d = ax + by\]
\item $d$ is unique up to multiplication by a unit in $R$.
\end{enumerate}
\end{prop}

\nl

\begin{prop}
\hl{Every nonzero prime ideal in a PID is a maximal ideal.} 
\end{prop}

\nl

\begin{cor}
If $R$ is any commutative ring such that the polynomial ring $R[x]$ is a PID (or Euclidean Domain), then $R$ is necessarily a field.
\end{cor}

\nl

\begin{defn}
Let $R$ be an integral domain
\begin{enumerate}
\item Suppose $r\in R$ is nonzero and is not a unit. Then $r$ is called \textit{\textbf{irreducible}} if $R$ if whenever $r = ab$ with $a,b\in R$ at least one of $a$ or $b$ is a unit in $R$.
\item The nonzero element $p\in R$ is called \textbf{\textit{prime}} in $R$ it the ideal $(p)$ generated by $p$ is a prime ideal. In other words, for any $a,b\in R$ if $p\ |\ ab$ then either $p\ |\ a$ or $p\ |\ b$.
\item Two elements $a,b\in R$ differing by a unit are said to be \textit{\textbf{associate}} in $R$.
\end{enumerate}
\end{defn}

\nl

\begin{prop}
\hl{In an integral domain a prime element is always irreducible.}
\end{prop}

\nl

\begin{prop}
In a PID a nonzero element is prime if and only if it is irreducible.
\end{prop}

\nl

\begin{defn}
A \hl{\textit{\textbf{Unique Factorization Domain (UFD)}}} is an integral domain $R$ in which every nonzero element $r\in R$ which is not a unit has the following two properties:
\begin{enumerate}
\item $r$ can be written as the finite product of irreducibles $p_i$ of $R$: $r = p_1p_2\cdots p_n$ and
\item the decomposition given in (1) is unique up to associates. 
\end{enumerate}
\end{defn}

\nl

\begin{prop}
\hl{In a UFD a nonzero element is a prime if and only if it is irreducible.}
\end{prop}

\nl

\begin{prop}
Let $a$ and $b$ be two nonzero elements of the UFD $R$ and suppose
\[a = u\ p_1^{e_1}p_2^{e_2}p_3^{e_3}\cdots p_n^{e_n}\qquad\text{and}\qquad b = v\ p_1^{f_1}p_2^{f_2}p_3^{f_3}\cdots p_n^{f_n}\]
are prime factorizations for $a$ and $b$, where $u$ and $v$ are units, the primes $p_1,p_2,\ldots,p_n$ are \textit{distinct} and the exponents $e_i$ and $f_i$ are $\geq 0$. Then the element 
\[d = p_1^{\min(e_1,f_1)}p_2^{\min(e_2,f_2)}p_3^{\min(e_3,f_3)}\cdots p_n^{\min(e_n,f_n)}\]
is a greatest common divisor of $a$ and $b$. 
\end{prop}

\nl

\begin{thm}
Every PID is a UFD. In particular, every Euclidean Domain is a UFD. 
\end{thm}

\nl

\begin{lem}
The prime number $p\in \Z$ divides an integer of the form $n^2 + 1$ if and only if $p$ is either 2 or is an odd prime congruent to $1\mod 4$.
\end{lem}

\nl

\begin{prop}\nl
\begin{enumerate}
\item \textit{(Fermat's Theorem on sums of squares)} The prime $p$ is the sum of two integer squares, $p = a^2 + b^2$ if and only if $p = 2$ or $p\equiv 1\mod 4$. Except for the interchanging $a$ and $b$, the representation of $p$ as the sum of two squares is unique. 
\item The irreducible elements in the Gaussian integers $\Z[i]$ are as follows
\begin{enumerate}
\item $1 + i$
\item the primes $p\in \Z$ with $p\equiv 3\mod 4$
\item $a + bi,\ a-bi$, the distinct irreducible factors of $p = a^2 + b^2$ for the primes $p\in \Z$ with $p\equiv 1\mod 4$.
\end{enumerate}
\end{enumerate}
\end{prop}


%################################################################################
\section{Polynomial Rings}
\setcounter{thm}{0}

\begin{prop}
Let $I$ be an ideal of $R$ and let $(I) = I[x]$ denote the ideal of $R[x]$ generated by $I$. Then 
\[R[x]/(I) \cong (R/I)[x].\]
In particular, if $I$ is a prime ideal of $R$ then (I) is a prime ideal of $R[x]$
\end{prop}

\nl

\begin{defn}
The \textit{polynomial ring in the variables} $x_1,x_2,\ldots,x_n$ \textit{with coefficients in $R$}, denoted $R[x_1,x_2,\ldots,x_n]$, is defined inductively by
\[R[x_1,x_2,\ldots,x_n] = R[x_1,x_2,\ldots,x_{n-1}][x_n]\]
\end{defn}

\nl

\begin{thm}
Let $F$ be a field. The polynomial ring $F[x]$ is a Euclidean Domain. Specifically, if $a(x)$ and $b(x)$ are two polynomials in $F[x]$ with $b(x)$ nonzero, the there are \textit{unique} $q(x)$ and $r(x)$ in $F[x]$ such that
\[a(x) = q(x)b(x) + r(x)\qquad\text{with } r(x) = 0\text{ or } deg(r(x))<deg(b(x)).\]
\end{thm}

\nl

\begin{prop}\hl{\textit{(Gauss' Lemma)}} Let $R$ be a UFD with field of fractions $F$ and let $p(x)\in R[x]$. If $p(x)$ is reducible in $F[x]$ then $p(x)$ is reducible in $R[x]$. More precisely, if $p(x) = A(x)B(x)$ for some nonconstant polynomials $A(x),B(x)\in F[x]$, then there are some nonzero elements $r,s\in F$ such that $rA(x) = a(x)$ and $sB(x) = b(x)$ both lie in $R[x]$ and $p(x) = a(x)b(x)$ is a factorization in $R[x]$.
\end{prop}

\nl

\begin{cor}
Let $R$ be a UFD, let $F$ be its field of fractions and let $p(x)\in R[x]$. Suppose the gcd of the coefficients of $p(x)$ is 1. Then $p(x)$ is irreducible in $R[x]$ if and only if it is irreducible in $F[x]$. In particular, if $p(x)$ is a monic polynomial that is irreducible in $R[x]$, then $p(x)$ is irreducible in $F[x]$.
\end{cor}

\nl

\begin{thm}
$R$ is a UFD if and only if $R[x]$ is a UFD.
\end{thm}

\nl

\begin{cor}
If $R$ is a UFD, then a polynomial ring in an arbitrary number of variables with coefficients in $R$ is also a UFD.
\end{cor}

\nl

\begin{prop}
Let $F$ be a field and let $p(x)\in F[x]$. Then $p(x)$ has a factor of degree one if and only if $p(x)$ has a root in $F$.
\end{prop}

\nl

\begin{prop}
A polynomial of degree two or three is reducible over a field $F$ if and only if it has a root in $F$.
\end{prop}

\nl

\begin{prop}
Let $p(x) = a_nx^n+ a_{n-1}x^{n-1} + \cdots +a_0$ be a polynomial with integer coefficients. If $r/s\in \Q$ is in lowest terms and $r/s$ is a root of $p(x)$, then $r$ divides the constant term and $s$ divides the leading coefficient of $p(x)$. In particular, if $p(x)$ is a monic polynomial with integer coefficients and $p(d) \neq 0$ for all integers dividing the constant term of $p(x)$, then $p(x)$ has no roots in $\Q$.
\end{prop}

\nl

\begin{prop}
\hl{Let $I$ be a proper ideal in the integral domain $R$ and let $p(x)$ be a nonconstant monic polynomial in $R[x]$. If the image of $p(x)$ in $(R/I)[x]$ cannot be factored in $(R/I)[x]$ into two polynomials of smaller degree, then $p(x)$ is irreducible in $R[x]$.} \underline{\textit{(Use this with $\Z$ AND $\Z/p\Z$ to prove irreducibility.)}}
\end{prop} 

\nl

\begin{prop}\hl{\textit{(Eisenstein's Criterion)}}
Let $P$ be a prime ideal of the integral domain $R$ and let $f(x) = x^n + a_{n-1}x^{n-1}+\cdots + a_0$ be a polynomial in $R[x]$ where $n\geq 1$. Suppose $a_{n-1}, \ldots a_0$ are all elements of $P$ and suppose $a_0$ is not an element of $P^2$. Then $f(x)$ is irreducible in $R[x]$.
\end{prop}

\nl

\begin{prop}
The maximal ideals in $F[x]$ are the ideals $(f(x))$ generated by irreducible polynomials $f(x)$. In particular $F[x]/ (f(x))$ is a field if and only if $f(x)$ is irreducible.
\end{prop}

\nl

\begin{prop}
Let $g(x)$ be a nonconstant monic element of $F[x]$ and let
\[g(x) = f_1(x)^{n_1}f_2(x)^{n_2}\cdots f_k(x)^{n_k}\]
be its factorization into irreducible, where the $f_i(x)$ are distinct. Then we have the following isomorphism of rings:
\[F[x]/(g(x)) \cong F[x]/ (f_1(x)^{n_1}) \times F[x]/ (f_2(x)^{n_2}) \times \cdots F[x]/ (f_k(x)^{n_k}).\]
\end{prop}

\nl

\begin{prop}
If the polynomial $f(x)$ has roots $\al_1,\al_2,\ldots,\al_k$ in $F$, then $f(x)$ has $(x-\al_1)\cdots (x-\al_k)$ as a factor. In particular, a polynomial of degree $n$ in one variable has at most $n$ roots in $F$, even counted with multiplicity.
\end{prop}

\nl

\begin{prop}
A finite subgroup of the multiplicative group of a field is cyclic. In particular, if $F$ is a finite field, the the multiplicative group $F^\times$ of nonzero elements of $F$ is a cyclic group.
\end{prop}

\nl

\begin{cor}
Let $n\geq 2$ be an integer with factorization $n = p_1^{\al_1}p_2^{\al_2}\cdots p_r^{\al_r}$ in $\Z$, where $p_1,p_2,\ldots,p_r$ are distinct primes. We have the following isomorphism of (multiplicative) groups:
\begin{enumerate}
\item $(\Z/n\Z)\cong (\Z/p_1^{\al_1}\Z)^\times \times (\Z/p_2^{\al_2}\Z)^\times \times \cdots \times(\Z/p_r^{\al_r}\Z)^\times$
\item $(\Z/2^\al \Z)^\times$ is the direct product of a cyclic group of order 2 and a cyclic group of order $2^{\al -2}$, for all $\al \geq 2$
\item $(\Z/p^\al \Z)^\times$ is a cyclic group of order $p^{\al -1}(p-1)$, for all odd primes $p$.
\end{enumerate}
\end{cor}


%################################################################################
\section{Introduction to Module Theory}
\setcounter{thm}{0}

\begin{defn}
Let $R$ be a ring (not necessarily commutative nor with 1). A \textit{\textbf{left $R$-module}} or a \textit{left module over $R$} is a set $M$ together with
\begin{enumerate}
\item a binary operation $+$ on $M$ under which $M$ is an abelian group, and
\item an action of $R$ on $M$ (that is, a map $R\times M\ra M$) denoted by $rm$, for all $r\in R$ and for all $m\in M$ which satisfies
\begin{enumerate}
\item $(r + s)m = rm + sm$, \ \ \ for all $r,s\in R,\ m\in M$
\item $(rs)m = r(sm)$, \ \ \ for all $r,s\in R,\ m\in M$, and 
\item $r(m + n) = rm + rn$, \ \ \ for all $r,s\in R,\ m\in M$.
\end{enumerate}
If the ring $R$ has 1 we impose the additional axiom:
\begin{enumerate}
\item[(d)]  $1m = m$, \ \ \ for all $m\in M$.
\end{enumerate}
\end{enumerate}
\end{defn}

\nl

\begin{defn}
Let $R$ be a ring and let $M$ be an $R$-module. An $R$-\textit{\textbf{submodule}} of $M$ is a subgroup $N$ of $M$ which is closed under the action of ring elements. 
\end{defn}

\nl

\begin{prop}
\hl{\textit{(The Submodule Criterion)}} Let $R$ be a ring and let $M$ be an $R$-module. A subset $N$ of $M$ is a submodule of $M$ if and only if
\begin{enumerate}
\item $N\neq \es$, and
\item $x + ry\in N$ for all $r\in R$ and for all $x,y\in M$.
\end{enumerate}
\end{prop}

\nl

\begin{defn}
Let $R$ be a ring and let $M$ and $N$ be $R$-modules.
\begin{enumerate}
\item A map $\vphi: M\ra N$ is an $R$\textbf{\textit{-module homomorphism}} if it respects the $R$-module structures of $M$ and $N$, i.e.,
\begin{enumerate}
\item $\vphi(x + y) = \vphi(x) + \vphi(y)$, \ \ \ for all $x,y\in M$ and
\item $\vphi(rx) = r\vphi(x)$, \ \ \ for all $r\in R$, $x\in M$.
\end{enumerate}
\item An $R$-module homomorphism is an \textbf{\textit{isomorphism}} if it is both injective and surjective. The modules $M$ and $N$ are said to be \textbf{\textit{isomorphic}}, denoted $M\cong N$ if there is some $R$-module isomorphism $\vphi: M\ra N$.
\item If $\vphi:M\ra N$ is an $R$-module homomorphism, let $\ker(\vphi) = \{m\in M\ |\ \vphi(m) = 0\}$ and let $\vphi(M) = \{n\in N\ |\ n = \vphi(m)\text{ for some } m\in M\}$.
\item Let $M$ and $N$ be $R$-modules and define $\Hom_R(M,N)$ to be the set of $R$-module homomorphisms from $M$ to $N$.
\end{enumerate}
\end{defn}

\nl

\begin{prop}
Let $M$, $N$, and $L$ be $R$-modules
\begin{enumerate}
\item A map $\vphi: M\ra N$ is an $R$-module homomorphism if and only if $\vphi(rx + y) = r\vphi(x) + \vphi(y)$ for all $c,y\in M$ and $r\in R$.
\item Let $\vphi,\ \psi$ be elements of $\Hom_R(M,N)$. Define $\vphi + \psi $ by
\[(\vphi + \psi)(m) = \vphi(m) + \psi(m)\qquad\text{for all } m\in M.\]
Then $\vphi + \psi\in \Hom_R(M,N)$ and with this operation $\Hom_R(M,N)$ is an abelian group. If $R$ is a commutative ring the for $r\in R$ define $r\vphi$ by 
\[(r\vphi)(m) = r(\vphi(m))\qquad\text{for all } m\in M.\]
Then $r\vphi\in\Hom_R(M.N)$ and with this action of the commutative ring $R$ the abelian group $\Hom_R(M,N)$ is an $R$-module.
\item If $\vphi\in\Hom_R(L,M)$ and $\psi\in\Hom_R(M,N)$ then $\psi\circ\vphi\in\Hom_R(L,N)$.
\item With addition as above and multiplication defined as function composition, $\Hom_R(M,M)$ is an $R$-algebra.
\end{enumerate}
\end{prop}

\nl

\begin{defn}
The ring $\Hom_R(M,M)$ is called the \textbf{\textit{endomorphism ring of $M$}} and will often be denoted by $\End_R(M)$. Elements of $\End(M)$ are called \textbf{\textit{endomorphisms}}.
\end{defn}

\nl

\begin{prop}
Let $R$ be a ring, let $M$ be an $R$-module, and let $N$ be a submodule of $M$. The quotient group $M/N$ can be made into an $R$-module by defining an action of elements of $R$ by
\[r(x + N) = (rx) + N),\qquad\text{ for all }r\in R,\ x + N \in M/N.\]
The natural projection map $\pi:M\ra M/N$ is an $R$-module homomorphism with kernel $N$.
\end{prop}

\nl

\begin{defn}
Let $A$, $B$ be submodules of the $R$-module $M$. The \textit{sum} of $A$ and $B$ is the set 
\[A + B = \{a + b\ |\ a\in A,\ b\in B\}.\]
\end{defn}

\nl

\begin{defn}
Let $M$ be an $R$-module and let $N_1,\ldots,N_n$ be submodules of $M$.
\begin{enumerate}
\item The \textbf{\textit{sum}} of $N_1,\ldots,N_n$ is the set of all finite sums of elements form the sets $N_i:\ \{a_1+\cdots+a_n\ |\ a_i\in N_i\}$. Denote this sum by $N_1+\cdots +N_n$.
\item For any subset $A$ of $M$ let
\[RA = \{r_1a_1+\cdots+r_ma_m\ |\ a_i\in A,\ r_i \in R,\ m\in\Z^+\}.\]
If $A$ is finite we may write $Ra_1 + Ra_2+\cdots +Ra_m$. Call $RA$ th \textit{\textbf{submodule of $M$ generated by $A$}}. If $N$ is a submodule of $M$ and $N = RA$ for some subset $A$ of $M$, we call $A$ a set of generators or a generating set for $N$,. and we say that $N$ is generated by $A$.
\item A submodule $N$ of $M$ is \textbf{\textit{finitely generated}} if there is some finites subset $A$ of $M$ such that $N = RA$.
\item A submodule $N$ of $M$ is \textit{\textbf{cyclic}} if there exists an element $a\in M$ such that $N = Ra$, that is, if $N$ is generated by one element.
\end{enumerate}
\end{defn}

\nl

\begin{prop}
Let $N_1,N_2,\ldots,N_k$ be submodules of the $R$-module $M$. Then the following are equivalent
\begin{enumerate}
\item The map $\pi:N_1\times N_2\times \cdots \times N_k\ra N_1 + N_2 + \cdots + N_k$ defined by 
\[\pi(a_1,a_2,\ldots,a_k) = a_1 + a_2+\cdots + a_k\]
is an isomorphism (of $R$-modules)
\item $N_j\cap N_1+\cdots N_{j-1} + N_{j + 1} +\cdots +N_k = 0$ for all $j\in \{1,2,\ldots, k\}$.
\item Every $x\in N_1+\cdots + N_k$ can be written \textit{uniquely} in the form $a_1 + a_2 + \cdots + a_k$ for $a_i \in N_i$.
\end{enumerate}
\end{prop}

\nl

\begin{defn}
If an $R$-module $M = N_1 + N_2 + \cdots + N_k$ is the sum of submodules $N_1,N_2,\ldots,N_k$ of $M$ satisfying the equivalent conditions in the above proposition, then $M$ is said to be the \textit{\textbf{(internal) direct sum}} of $N_1, N_2, \ldots, N_k$ written
\[M = N_1 \oplus N_2 \oplus \cdots \oplus N_k.\]
\end{defn}

\nl

\begin{defn}
And $R$-module $F$ is said to be \textbf{\textit{free}} on the subset $A$ of $F$ if for every nonzero element $x$ of $F$, there exist unique nonzero elements $r_1,r_2,\ldots, r_n$ of $R$ and unique $a_1,a_2,\ldots, a_n$ in $A$ such that $x = r_1a_1 + r_2 a_2 + \cdots + r_n a_n$, for some $n\in \Z^+$. In this situation we say $A$ is a \textbf{\textit{basis}} or \textbf{\textit{set of free generators}} for $F$. If $R$ is a commutative ring the cardinality of $A$ is called the \textbf{\textit{rank}} of $F$.
\end{defn}

\nl

\begin{thm}
For any set $A$ there is a free $R$-module $F(A)$ on the set $A$ and $F(A)$ satisfies the following \textbf{\textit{universal property}}: if $M$ is any $R$-module and $\vphi: A\ra M$ is any map of sets, then there is a unique $R$-module homomorphism $\Phi:F(A) \ra M$ such that $\Phi(a) = \vphi(a)$, for all $a\in A$, that is, the following diagram commutes.

\begin{center}
\begin{tikzcd}[row sep = 2em, column sep = 3em]
A\arrow[r, "\iota"]\arrow[dr, swap, "\vphi"] & F(A)\arrow[d, "\Phi"]\\
 & M
\end{tikzcd}
\end{center}

\end{thm}

\nl

\begin{cor}\nl
\begin{enumerate}
\item If $F_1$ and $F_2$ are free modules on the same set $A$, there is a unique isomorphism between $F_1$ and $F_2$ which is the identity map on $A$.
\item If $F$ is any free $R$-module with basis $A$, then $F\cong F(A)$. In particular, $F$ enjoys the same universal property with respect to $A$ as $F(A)$ does in the previous theorem.
\end{enumerate}
\end{cor}



%################################################################################
\section{Vector Spaces}
\setcounter{thm}{0}

\begin{defn}
If $F$ is an field and $V$ is an $F$-module, then $V$ is called a \textit{vector space over $F$}.
\end{defn}

\nl

\begin{defn}\nl
\begin{enumerate}
\item A subset $S$ of $V$ is called a set of \textbf{\textit{linearly independent}} vectors if an equation $\al_1v_1+\cdots +\al_nv_n = 0$ with $\al_1,\ldots \al_n\in F$ and $v_1,\ldots, v_n\in S$ implies $\al_1 = \al_2 = \cdots = \al_n = 0$. \hl{(Note: an infinite set is linearly independent if this condition holds for any finite subset.)}
\item A \textit{basis} of a vector space $V$ is an \textbf{\textit{ordered set}} of linearly independent vectors which span $V$. In particular, two bases sill be considered different even if one is simply a rearrangement of the other. This is sometimes referred to as an \textit{ordered basis}.
\end{enumerate}
\end{defn}

\nl

\begin{prop}
Assume that $\A = \{v_1,v_2,\ldots, v_n\}$ spans the vector space $V$ but no proper subset of $\A$ spans $V$. Then $\A$ is a basis of $V$. \hl{In particular, any finitely generated vector space over $F$ is a free $F$-module.}
\end{prop}

\nl

\begin{thm}\hl{\textit{(A Replacement Theorem)}}
Assume $\A = \{a_1,a_2,\ldots,a_n\}$ is a basis for $V$ containing $n$ elements and $\{b_1,b_2,\ldots, b_m\}$ is a set of linearly independent vectors in $V$. Then there is an ordering $a_1,a_2,\ldots,a_n$ such that for each $k\i n\{1,2,\ldots,m\}$ the set $\{b_1,\ldots, b_k, a_{k+1}, \ldots, a_n\}$ is a basis of $V$. In other words, the elements of $b_1,b_2,\ldots, b_m$ can be used to successively replace the elements of the basis $\A$, still retaining a basis. In particular $n\geq m$
\end{thm}

\nl

\begin{cor}\nl
\begin{enumerate}
\item Suppose $V$ has a finite basis with $n$ elements. Any set of linearly independent vectors has $\leq n$ elements. Any spanning set has $\geq n$ elements.
\item If $V$ has some finite basis, then any two bases of $V$ have the same cardinality.
\end{enumerate}
\end{cor}

\nl

\begin{defn}
If $V$ is a finitely generated $F$-module the cardinality of any basis is called the \textit{dimension} of $V$ and is denoted $\dim_F(V)$, or just $\dim(V)$ when $F$ is clear from the context, and $V$ is said to be \textit{finite dimensional over $F$}. If $V$ is not finitely generated, $V$ is said to be infinite dimensional.
\end{defn}

\nl

\begin{cor}
If $A$ is a set of linearly independent vectors in the finite dimensional vector space $V$, then there exists a basis of $V$ containing $A$
\end{cor}

\nl

\begin{thm}
If $V$ is an $n$ dimensional vector space over $F$, the $V\cong F^n$. In particular, any two finite dimensional vector spaces over $F$ of the same dimension are isomorphic.
\end{thm}

\begin{proof}
Let $v_1,v_2,\ldots,v_n$ be a basis for $V$. Define the map 
\[\vphi:F^n\ra V:(\al_1,\al_2,\ldots,\al_n)\mapsto\al_1v_1 + \al_2v_2+\cdots+\al_n v_n.\]
The map $\vphi$ is clearly $F$-linear, is surjective since the $v_i$ span $V$, and is injective since the $v_i$ are linearly independent, hence is an isomorphism.
\end{proof}

\nl

\begin{thm}
Let $V$ be a vector space over $F$ and let $W$ be a subspace of $V$. Then $V/W$is a vector space with $\dim(V) = \dim(W) + \dim(V/W)$.
\end{thm}

\nl

\begin{cor}
Let $\vphi:V\ra U$ be a linear transformation of vector spaces over $F$. Then $\ker(\vphi)$ is a subspace of $V$, $\vphi(V)$ is a subspace of $U$, and $\dim(V) = \dim(\ker(\vphi)) + \dim(\vphi(V))$.
\end{cor}

\nl

\begin{cor}
Let $\vphi:V\ra U$ be a linear transformation of vector spaces of the same finite dimension. Then the following are equivalent
\begin{enumerate}
\item $\vphi$ is an isomorphism
\item $\vphi$ is injective, i.e., $\ker(\vphi) = 0$
\item $\vphi$ is surjective
\item $\vphi$ sends a basis of $V$ to a basis of $W$.
\end{enumerate}
\end{cor}

\nl

\begin{defn}
If $\vphi:V\ra U$ is a linear transformation of vector spaces over $F$, $\ker(\vphi)$ is sometimes called the \textit{\textbf{null space}} of $\vphi$. and the dimension of $\ker(\vphi)$ is called the \textit{\textbf{nullity}} of $\vphi$. The dimension of $\vphi(V)$ is called the \textit{\textbf{rank}} of $\vphi$. If $\ker(\vphi) = 0$, then the transformation is said to be \textit{\textbf{nonsingular}}.
\end{defn}

\nl

\begin{defn}
The $m \times m$ matrix $A = (a_{ij})$ associated to the linear transformation $\vphi$ is said to \textit{represent} the linear transformation $\vphi$ with respect to the bases $\BB, \EE$. Similarly, $\vphi$ is the linear transformation represented by $A$ with respect to the bases $\BB, \EE$.
\end{defn}

\nl

\begin{thm}
Let $B$ be a vector space over $F$ of dimension $n$ and let $W$ be a vector space over $F$ of dimension $m$, with bases $\BB, \EE$ respectively. Then the map $\Hom_F(V, W)\ra M_{m\times n}(F)$ from the space of linear transformations from $v$ to $W$ to the space of $m\times n$ matrices with coefficients in $F$ defined by $\vphi\mapsto M_\BB^\EE(\vphi)$ is a vector space isomorphism. In particular, there is a bijective correspondence between linear transformations and their associated matrices with respect to a fixed choice of bases.
\end{thm}

\nl

\begin{cor}
The dimension of $\Hom_F(V,W)$ is $(\dim(V))(\dim(W))$.
\end{cor}

\nl

\begin{defn}
An $m\times n$ matrix $A$ is called \textit{\textbf{nonsingular}} if $Ax = 0$ with $x\in F^n$ implies $x = 0$.
\end{defn}

\nl

\begin{thm}
With notation as above $M_\BB^\EE(\vphi\circ \psi) = M_\BB^\EE(\vphi) M_\BB^\EE(\psi)$.
\end{thm}

\nl

\begin{cor}
Matrix multiplication is associative and distributive. An $n\times n$ matrix $A$ is nonsingular if and only if it is invertible.
\end{cor}

\nl

\begin{cor}\nl
\begin{enumerate}
\item If $\BB$ is a basis of the $n$-dimensional space $V$, the map $\vphi\mapsto M_\BB^\BB(\vphi)$ is a ring and a vector space isomorphism of $\Hom_F(V,V)$ onto the space $M_n(F)$ of $n\times n$ matrices with coefficients in $F$.
\item $GL(V)\cong GL_n(F)$ where $\dim(V) = n$. 
\end{enumerate}
\end{cor}

\nl

\begin{defn}
If $A$ is any $m\times n$ matrix with entries of $F$, the \textit{\textbf{row rank}} of $A$ is the maximal number of linearly independent rows of $A$.
\end{defn}

\nl

\begin{defn}
Two $n\times n$ matrices $A$ and $B$ are said to be \textit{\textbf{similar}} if the is an invertible $n\times n$ matrix $P$ such that $P\inv A P = B$. Two linear transformations $\vphi$ and $\psi$ from a vector space $V$ to itself are said to be \textit{similar} if the is a nonsingular linear transformation $\xi$
\end{defn}

\nl

\begin{defn}\nl
\begin{enumerate}
\item For $V$ any vector space over $F$ let $V^* = \Hom_F(V,F)$ be the space of linear transformations from $V$ to $F$, called the \textit{\textbf{dual space}} of $V$. Elements of $V^*$ are called \textit{\textbf{linear functionals}}.
\item If $\BB  = \{v_1,v_2,\ldots,v_n\}$ is a basis of the finite dimensional space $V$, define $v_i^* \in V^*$ for each $i = 1..n$ by its action on the basis $\BB$:
\[v_i^*(v_j) = \begin{cases}1, & \text{if } i = j\\
0, & \text{if } i \neq j\end{cases}\qquad 1\leq j\leq n.\]
\end{enumerate}
\end{defn}

\nl

\begin{prop}
With notations as above, $\{v_1^*,v_2^*,\ldots,v_n^*\}$ is a basis of $V^*$. In particular, if $V$ is finite dimensional then $V^*$ has the same dimension as $V$.

\begin{proof}
(Copied from D\&F) Observe that since $V$ is finite dimensional, $\dim(V^*) = \dim(\Hom_F(V,F)) = \dim(V) = n$ (\textcolor{red}{Corollary 11.11}), so since there are $n$ of the $v_i^*$'s it suffices to prove that they are linearly independent. If
\[\al_1v_1^* + \al_2v_2^* + \cdots + \al_nv^n = 0\quad \text{in } \Hom_F(V,F),\]
then applying this element to $v_i$ and using th equation above gives us that $\al_i = 0$. Since $i$ is arbitrary these elements are linearly independent.
\end{proof}
\end{prop}

\nl

\begin{defn}
The basis $\{v_1^*,v_2^*,\ldots,v_n^*\}$ of $V^*$ is called the \textit{\textbf{dual basis}} to  $\{v_1,v_2,\ldots,v_n\}$.
\end{defn}

\nl

\begin{thm}
There is a natural injective linear transformation from $V$ to $V^{**}$. If $V$ is finite dimensional then this linear transformation is an isomorphism. 

\textit{Sketch of proof.} Let $v\in V$ and define the evaluation map $E_v:V^*\ra F:f\mapsto f(v)$. This is a linear transformation from $V^*$ to $F$, and so is an element of $\Hom_F(V^*, F) = V^{**}$. This defines a natural map $\vphi: V\ra V^{**}:v\mapsto E_v$. This map is injective for all $V$ and $\vphi$ is an isomorphism if $V$ is finite dimensional.
\end{thm}

\nl

\begin{thm}
Let $V,W$ be finite dimensional vector spaces over $F$ with bases $\BB, \EE$, respectively and let $\BB^*,\EE^*$ be the dual bases . Fix some $\vphi\in \Hom(V,W)$. Then for each $f\in W^*$, the composite $f\circ \vphi$ is a linear transformation from $V$ to $F$, that is $f\circ \vphi\in V^*$. Thus, we can define a map $\vphi^*:W^* \ra V^*:f\mapsto f\circ \vphi$ (called the \textit{\textbf{pullback}} of $f$) and the matrix $M_{\EE^*}^{\BB^*}(\vphi^*)$ is the transpose of th matrix $M_\BB^\EE(\vphi)$.
\end{thm}

\nl

\begin{cor}
For any matrix $A$, the row rank of $A$ equals the column rank of $A$.
\end{cor}

\nl

\begin{defn}\nl
\begin{enumerate}
\item A map $\vphi:V_1\times V_2\times \cdots \times V_n\ra W$ is called \textit{\textbf{multilinear}} if for each fixed $i$ and fixed $i$ and fixed elements $v_j\in V_j$, $j\neq i$, the map
\[V_i \ra W\qquad \text{defined by}\qquad x\mapsto \vphi(v_1,\ldots,v_{i-1},x,v_{i+1},\ldots, v_n)\]
is an $R$-module homomorphism. If $V_i = V$, $i = 1,2,\ldots, n$, then $\vphi$ is called an $n$\textit{-multilinear function on $V$}, and if in addition $W = R$, $\vphi$ is called an \textit{$n$-multilinear form on $V$}.

\item An $n$-multilinear function $\vphi$ on $V$ is called \textit{alternating} if $\vphi(v_1, v_2,\ldots, v_n) = 0$ whenever $v_i = v_{i + 1}$ for some $i\in \{1,2,\ldots, n-1\}$. The function $\vphi$ is called \textit{symmetric} if interchanging $v_i$ and $v_j$ for any $i$ and $j$ in $(V_1,v_2,\ldots, v_n)$ does not alter the value of $\vphi$ on this $n$-tuple.
\end{enumerate}
\end{defn}

\nl

\begin{prop}
Let $\vphi$ be an $n$-multilinear alternating function on $V$. Then 
\begin{enumerate}
\item $\vphi(v_1,\ldots, v_{i-1}, v_{i + 1}, v_i, v_{i + 2}, \ldots, v_n) = -\vphi(v_1, v_2, \ldots, v_n)$ for any $i\in \{1,2,\ldots, n-1\}$.

\item For each $\sigma\in S_n$, $\vphi(v_{\sigma(1)}, v_{\sigma(2)},\ldots, v_{\sigma(n)}) = sgn(\sigma)\vphi(v_1, v_2, \ldots, v_n)$.

\item If $v_i = v_j$ for any pair of distinct $i,j\in \{1,2,\ldots, v_n\}$ then $\vphi(v_1, v_2, \ldots, v_n) = 0$.

\item If $v_i$ is replaced by $v_i + \al v_j$ in $(v_1,v_2,\ldots,v_n)$ for any $j\neq i$ and any $\al \in R$, the value of$\vphi$ on this $n$-tuple is not changed.
\end{enumerate}
\end{prop}

\nl

\begin{prop}
Assume $\vphi$ is an $n$-multilinear alternating function on $V$ and that for some $v_1,v_2,\ldots,v_n$ and $w_1,w_2,\ldots,w_n\in V$ and some $\al_{ij}\in R$ we have
\begin{align*}
w_1 &= \al_{11}v_1 + \al_{21}v_2 + \cdots + \al_{n1}v_n\\
w_2 &= \al_{12}v_1 + \al_{22}v_2 + \cdots + \al_{n2}v_n\\
\vdots\\
w_n &= \al_{1n}v_1 + \al_{2n}v_2 + \cdots + \al_{nn}v_n.\\
\end{align*}
Then 
\[\vphi(w_1,w_2,\ldots,w_n) = \sum_{\sigma\in S_n} sgn(\sigma)\al_{\sigma(1)1}\al_{\sigma(2)2}\cdots\al_{\sigma(n)n}\vphi(v_1,v_2,\ldots,v_n).\]
\end{prop}

\nl

\begin{defn}
An $n\times n$ \textit{\textbf{determinant function}} on $R$ is any function 
\[\det:M_{n\times n}(R)\ra R\]
that satisfies the following two axioms:
\begin{enumerate}
\item $\det$ is an $n$-multilinear alternating form on $R^n( = V)$, where the $n$-tuples are the $n$ columns of the matrices in $M_{n\times n}(R)$.
\item $\det(I) = 1$.
\end{enumerate}
\end{defn}

\nl

\begin{thm}
There is a unique $n\times n$ determinant function on $R$ and it can be computed for any $n\times n$ matrix $(\al_{ij})$ by the formula:
\[det(\al_{ij}) = \sum_{\sigma\in S_n} sgn(\sigma) \al_{\sigma(1)1} \al_{\sigma(2)2} \cdots \al_{\sigma(n)n}\]
\end{thm}

\nl

\begin{cor}
The determinant is an $n$-multilinear function of the rows of $M_{n\times n}(R)$ and for any $n\times n $ matrix $A$, $\det(A) = \det(A^t)$.
\end{cor}

\nl

\begin{thm}\textit{(Cramer's Rule)}
If $A_)1,A_23,\ldots, A_n$ are the columns of an $n\times n$ matrix $A$ and $B = \be_1A_1 + \be_2A_2 + \cdots + \be_nA_n$, for some $\be_1,\ldots, \be_n\in R$, then
\[\be_i\det(A) = \det(A_1,\ldots,A_{i-1}, B, A_{i+1},\ldots, A_n).\]
\end{thm}

\nl

\begin{cor}
If $R$ is an integral domain, then $\det(R) = 0$ for $A\in M_n(R)$ if and only if the columns of $A$ are $R$-linearly dependent as elements of the free $R$-module of rank $n$. Also $\det(A) = 0$ if and only if the rows of $A$ are $R$-linearly dependent.
\end{cor}

\nl

\begin{thm}
For matrices $A, B\in M_{n\times n}(R)$, $\det(A,B) = \det(A)\det(B)$.
\end{thm}

\nl

\begin{defn}
Let $A = (\al_{ij})$ be an $n\times n$ matrix. For each $i,\ j$, let $A_{ij}$ be the $n-1\times n-1$ matrix obtained from $A$ by deleting its $i^{th}$ row and $j^{th}$ column. Then $(-1)^{i + j}\det(A_{ij})$ is called the \textit{\textbf{$ij$ cofactor of $A$.}}
\end{defn}

\nl

\begin{thm}\textit{(The Cofactor Expansion Formula along the $i^{th}$ row)}
If $A = (\al_{ij})$ is an $n\times n$ matrix, then for each fixed $i\in \{1,2,\ldots n\}$ the determinant of $A$ can be computed from the formula
\[\det(A) = (-1)^{i+1}\al_{i1}\det(A_{i1}) + (-1)^{i+2}\al_{i2}\det(A_{i2}) + \cdots + (-1)^{i+n}\al_{in}\det(A_{in}).\]
\end{thm}

\nl

\begin{thm}\textit{(Cofactor Formula for the Inverse of a Matrix)} Let $A = (\al_{ij})$ be an $n\times n$ matrix and let $B$ be the transpose of is matrix of cofactors, i.e., $B = (\be_{ij})$, where $\be_{ij} = (-01)^{i + j}\det(A_{ji})$, $1\leq ii,j\leq n$. Then $AB = BA = \det(A)I$. Moreover, $\det(A)$ is a unit in $R$ if and only if $A$ is a unit in $M_{n\times n}(R)$; in this case the matrix $\frac{1}{\det(A)} B$ is the inverse of $A$.
\end{thm}

%################################################################################
\section{Modules over Principal Ideal Domains}
\setcounter{thm}{0}

\begin{defn}\nl
\begin{enumerate}
\item The left $R$ module $M$ is said to be a \textit{\textbf{Noetherian $R$-module}} or to satisfy the \textit{\textbf{ascending chain condition on submodules}} if there are no infinite increasing chains of submodules (any increasing chain stabilizes).
\item The ring $R$ is said to be \textit{Noetherian} if it is Noetherian as a left module over itself.
\end{enumerate}
\end{defn}

\nl

\begin{thm}
Let $R$ be a ring and let $M$ be a lift $R$-module. Then the following are equivalent:
\begin{enumerate}
\item \hl{$M$ is a Noetherian $R$-module.}
\item Every nonempty set of submodules of $M$ contains a maximal element under inclusion.
\item \hl{Every submodule of $M$ is finitely generated.}
\end{enumerate}
\end{thm}

\nl

\begin{cor}
If $R$ is a $PID$ then every nonempty set of ideal of $R$ has a maximal element and $R$ is a Noetherian ring.
\end{cor}

\nl

\begin{prop}
Let $R$ be an integral domain and let $M$ be a free $R$-module of rank $n<\infty$. Then any $n + 1$ elements of $M$ are $R$-linearly dependent.
\end{prop}

\nl

\begin{defn}
For any integral domain $R$ the \textit{\textbf{rank}} of an $R$-module $M$ is the maximum number of $R$-linearly independent elements of $M$.
\end{defn}

\nl

\begin{thm}
Let $R$ be a PID, let $M$ be a free $R$-module of finite rank $n$ and let $N$ be a submodule of $M$. Then
\begin{enumerate}
\item $N$ is free of rank $m$, $M\leq n$
\item there exists a basis $y_1,y_2,\ldots,y_n$ of $M$ so that $a_1y_1,\ldots,a_my_m$ is a basis of $N$ where $a_1, a_2, \ldots, a_m$ are nonzero elements of $R$ with the divisibility relations
\[a_1\ |\ a_2\ |\ \cdots\ |\ a_m.\]
\end{enumerate}
\end{thm}

\nl

\begin{thm}\textit{\hl{(Fundamental Theorem, Existence: Invariant Factor Form)}}
Let $R$ be a PID and let $M$ be a finitely generated $R$-module.
\begin{enumerate}
\item Then $M$ is isomorphic to the direct sum of finitely many cyclic modules. More precisely,
\[M\cong R^r\oplus R/(a_1) \oplus R/(a_2) \oplus \cdots \oplus R/(a_m)\]
for some integer $r\geq 0$ and nonzero elements $a_1, a_2,\ldots, a_m$ of $R$ which are not units in $R$ an which satisfy the divisibility relations 
\[a_1\ |\ a_2\ |\ \cdots\ |\ a_m.\]

\item $M$ is torsion free if and only if $M$ is free.
\item In the decomposition in (1) the set of torsion elements,
\[\Tor(M) \cong R/(a_1) \oplus R/(a_2) \oplus \cdots \oplus R/(a_m)\]
(Recall: $\Tor(M) = \{m\in M\ |\ rm = 0\text{ for some nonzero }r\in R\}$). In particular, $M$ is a torsion module if and only if $r = 0$ and in this case the the annihilator of $M$ is the ideal $(a_m)$.
\end{enumerate}
\end{thm}

\nl

\begin{defn}
The integer $r$ in the previous theorem is called the \textit{free rank} or the \textit{Betti number} of $M$ and the elements $a_1, a_2, \ldots, a_m\in R$ are called the \textit{\textbf{invariant factors}} of $M$.
\end{defn}

\nl

\begin{thm}\textit{\hl{(Fundamental Theorem, Existence: Elementary Divisor Form)}} 
Let $R$ be a PID and let $M$ be a finitely generated $R$-module. Then $M$ is the direct sum of a finite number of cyclic module whose annihilators are either $(0)$ or generated by powers of the primes in $R$, i.e.,
\[M\cong R^r \oplus R/(p_1^{\al_1})\oplus R/(p_2^{\al_2}) \oplus \cdots \oplus R/(p_t^{\al_t})\]
where $r\geq 0$ is an integer and $p_1^{\al_1},\ldots, p_t^{\al_t}$ are positive powers of (not necessarily distinct) primes in $R$.
\end{thm}

\nl

\begin{defn}
Let $R$ be a PID and let $M$ be a finitely generated $R$-module as in the previous theorem. The prime powers $p_1^{\al_1},\ldots, p_t^{\al_t}$ are called the \textit{\textbf{elementary divisors}} of $M$.
\end{defn}

\nl

\begin{thm}\hl{\textit{(The Primary Decomposition Theorem)}} Let $R$ be a PID and let $M$ be a nonzero torsion $R$-module with nonzero annihilator $a$. Suppose the factorization of $A$ into distinct prime powers in $R$ is 
\[a = up_1^{\al_1}p_2^{\al_2}\cdots p_n^{\al_n}\]
and let $N_i = \{x\in M\ |\ p_i^{\al_i}x = 0\}$. $1 \leq i\leq n$. Then $N_i$ is a submodule of $M$ with annihilator $p_i^{\al_i}$ and is the submodule of $M$ of all the elements annihilated by some power of $p_i$. We have
\[m\cong N_1\oplus N_2\oplus \oplus \cdots \oplus N_n.\]
If $M$ is finitely generated then each $N_i$ is the direct sum of finitely many cyclic module whose annihilators are divisors of $p_i^{\al_i}$.
\end{thm}

\nl

\begin{defn}
The submodule $N_i$ given in the previous theorem is called the $p_i$\textit{-primary component} of $M$.
\end{defn}

\nl

\begin{lem}
Let $R$ be a PID and let $p$ be a prime in $R$. Let $F$ denote the field $R/(p)$.
\begin{enumerate}
\item Let $M = R^r$. Then $M/pM \cong F^r$.
\item Let $M = R/(a)$ where $a$ is a nonzero element of $R$. Then 
\[M/pM \cong \begin{cases}F & \text{if $p$ divides $a$ in $R$}\\ 0 &\text{if $p$ does not divide $a$ in $R$.}\end{cases}\]
\item Let $M\cong R/(a_1)\oplus R/(a_2)\oplus \cdots \oplus R/(a_k)$ where each $a_i$ is divisible by $p$. Then $M/pM \cong F^k$.
\end{enumerate}
\end{lem}

\nl

\begin{thm}\hl{\textit{(Fundamental Theorem, Uniqueness)}}
Let $R$ be a PID.
\begin{enumerate}
\item Two finitely generated $R$-modules $M_1$ and $M_2$ are isomorphic if and only if they have the same free rank and list of invariant factors.
\item Two finitely generated $R$-modules $M_1$ and $M_2$ are isomorphic if and only if they have the same free rank and the same list of elementary divisors.
\end{enumerate}
\end{thm}

\nl

\begin{cor}
Let $R$ be a PID and let $M$ be a finitely generated $R$-module.
\begin{enumerate}
\item The elementary divisors of $M$ are the prime power factors of the invariant factor of $M$.
\item The largest invariant factor of $M$ is the product of the larges of the distinct prime powers among the elementary divisors of $M$, the next largest invariant factor is the product of the largest of the distinct prime powers among the remaining elementary divisors of $M$, and so on.
\end{enumerate}
\end{cor}

\nl

\begin{cor}\textit{(The Fundamental Theorem of Finitely Generated Abelian Groups)}
See \textcolor{red}{Theorem 5.3} and \textcolor{red}{Theorem 5.5}.
\end{cor}

\nl

\begin{defn}\nl
\begin{enumerate}
\item An element $\lambda$ of $F$ is called an \textit{\textbf{eigenvalue}} of a linear transformation $T$ if there is a nonzero vector $v\in V$ such that $T(v) = \lambda v$. In this situation $v$ is called an \textit{\textbf{eigenvector}} of $T$ with corresponding eigenvalue $\lambda$.

\item If $A$ is an $n\times n$ matrix with coefficients in $F$, and element $\lambda$ is called an \textit{eigenvalue} of $A$ with corresponding eigenvector $v$ is $V$ is a nonzero $n\times 1$ column vector such that $Av\ = \lambda v$.

\item If $\lambda$ is an eigenvalue of the linear transformation $T$, the set $\{v\in V\ |\ T(v) = \lambda v\}$ is called the \textit{\textbf{eigenspace}} of $T$ corresponding to the eigenvalue $\lambda$. Similarly, if $\lambda$ is an eigenvalue of the $n\times n$ matrix $A$, the set of $n\times 1$ matrices $v$ with $Av = \lambda v$ is called the \textit{eigenspace} of $A$ corresponding to the eigenvalue $\lambda$.
\end{enumerate}
\end{defn}

\nl

\begin{defn}
The determinant of a linear transformation from $V$ to $V$ is the determinant of any matrix representing the linear transformation.
\end{defn}

\nl

\begin{prop}
The following are equivalent:
\begin{enumerate}
\item $\lambda$ is an eigenvalue of $T$.
\item $\lambda I - T$ is a singular linear transformation.
\item $\det(\lambda I - T) = 0$.
\end{enumerate}
\end{prop}

\nl

\begin{defn}
Let $x$ be an indeterminate over $F$. The polynomial $\det(xI - T)$ is called the \textit{\textbf{characteristic polynomial}}of $T$ and will be denoted $c_T(x)$. If $A$ is an $n\times n$ matrix with coefficients in $F$, $\det(xI- A)$ is called the \textit{characteristic polynomial of $A$} and will be denoted $c_A(x)$.
\end{defn}

\nl

\begin{defn}
The unique monic polynomial which generates the ideal $Ann(V)$ in $F[x]$ is called the \textit{\textbf{minimal polynomial}} of $T$ and will be denoted $m_T(x)$. The unique monic polynomial of smallest degree which when evaluated at the matrix $A$ is the zero matrix is called the \textit{minimal polynomial} of $A$  and will be denoted $m_A(x)$.
\end{defn}
\nl\\
\textbf{Note:} Since $V$ is finite dimensional, we know that $V$ is a finitely generated module over $F$. So $V$ is torsion over $F[x]$ and we have that
\[V\cong F[x]/(a_1(x)) \oplus F[x]/(a_2(x))\oplus \cdots \oplus F[x]/(a_m(x))\]
where the $a_i(x)$ are subject to the divisibility relations
\[a_1(x)\ |\  a_2(x)\ |\ \cdots\ |\  a_m(x).\]
These $a_i(x)$ are called the invariant factors of $V$.

\nl

\begin{prop}
The minimal polynomial $m_T(x)$ is the largest invariant factor of $V$. All the invariant factors ov $V$ divide $m_T(x)$.
\end{prop}

\nl

\begin{defn}
Let $a(x) = x^k + b_{k - 1}x^{k - 1} + \cdots + b_1 x + b_0$ be any monic polynomial in $F[x]$. The \textit{\textbf{companion matrix}} of $a(x)$ is the $k\times k$ matrix with 1's down the first subdiagonal, $-b_0, -b_1, \ldots, -b_{k - 1}$ down the last column and zeros elsewhere. The companion matrix of $a(x)$ will be denoted $\CC_{a(x)}$.
\[\CC_{a(x)} = \begin{pmatrix}
0 & 0 & \cdots & \cdots & \cdots & -b_0\\
1 & 0 & \cdots & \cdots & \cdots & -b_1\\
0 & 1 & \cdots & \cdots & \cdots & -b_2\\
0 & 0 & \ddots &  &  & \vdots\\
\vdots & \vdots & & \ddots  &  & \vdots\\
0 & 0 & \cdots & \cdots & 1 & -b_{k - 1}\\
\end{pmatrix}\]
\end{defn}



\nl

\begin{defn}\nl
\begin{enumerate}
\item A matrix is said to be in \textit{\textbf{rational canonical form}} if it is the direct sum of companion matrices for monic polynomials $a_1(x), \ldots, a_m(x)$ of degree at least one with $a_1(x)\ |\  a_2(x)\ |\ \cdots\ |\  a_m(x).$ The polynomials $a_i(x)$ are called the \textit{invariant factors} of the matrix. Such a matrix is also said to be a \textit{\textbf{block diagonal}} matrix with block of the companion matrices for the $a_i(x)$.
\[\begin{pmatrix}
\CC_{a_1(x)} & & & \\
& \CC_{a_1(x)} & & \\
& & \ddots & \\
& & & \CC_{a_m(x)}
\end{pmatrix}\]
\item A \textit{\textbf{rational canonical form}} for a linear transformation $T$ is a matrix representing $T$ which is in rational canonical form.
\end{enumerate}
\end{defn}

\nl

\begin{thm}\hl{\textit{(Rational Canonical Form for Linear Transformations)}}
Let $V$ be a finite dimensional vector space over the field $F$ and let $T$ be a linear transformation of $V$.
\begin{enumerate}
\item There is a basis for $V$ with respect to which the matrix for $T$ is in rational canonical form.
\item The rational canonical form is unique
\end{enumerate}
\end{thm}

\nl

\begin{thm}
Let $S$ and $T$ be linear transformations of $V$. Then the following are equivalent:
\begin{enumerate}
\item $S$ and $T$ are similar linear transformations
\item the $F[x]$-modules obtained from $V$ via $S$ and via $T$ are isomorphic $F[x]$-modules
\item $S$ and $T$ have the same rational canonical form.
\end{enumerate}
\end{thm}

\nl

\begin{thm}\hl{\textit{(Rational Canonical Form for Linear Transformations)}}
Let $A$ be a $n\times n$ matrix over a filed $F$.
\begin{enumerate}
\item The matrix $A$ is similar to a matrix in rational canonical form.
\item The rational canonical form of $A$ is unique.
\end{enumerate}
\end{thm}

\nl

\begin{defn}
The \textit{invariant factors} of an $n\times n$ matrix over a field $F$ are the invariant factors of its rational canonical form.
\end{defn}

\nl

\begin{thm}
Let $A$ and $B$ be $n\times n$ matrices over a field $F$. Then $A$ and $B$ are similar if and only if $A$ and $B$ have the same rational canonical form.
\end{thm}

\nl

\begin{cor}
Let $A$ and $B$ be two $n\times n$ matrices over a field $F$ and suppose $F$ is a subfield of the field $K$.
\begin{enumerate}
\item The rational canonical form of $A$ is the same whether it is computed over $K$ or over $F$. The minimal and characteristic polynomials and the invariant factors of $A$ are the same whether $A$ is considered as a matrix over $F$ or as a matrix over $K$.
\item The matrices $A$ and $B$ are similar over $K$ if and only if they are similar over $F$.
\end{enumerate}
\end{cor}

\nl

\begin{lem}
Let $a(x)\in F[x]$ be any monic polynomial.
\begin{enumerate}
\item The characteristic polynomial of the companion matrix of $a(x)$ is $a(x)$.
\item If $M$ is the block diagonal matrix
\[\begin{pmatrix}
A_1 & 0 & \cdots & 0\\
0 & A_2 & \cdots & 0\\
\vdots & \vdots & \ddots & \vdots\\
0 & 0 & \cdots & A_k
\end{pmatrix}\]
given by the direct sum of matrices $A_1, A_2, \ldots, A_k$ then the characteristic polynomial of $M$ is the product of the characteristic polynomials of $A_1, A_2, \ldots, A_k$.
\end{enumerate}
\end{lem}

\nl

\begin{prop}
Let $A$ be an $n\times n$ matrix over the field $F$.
\begin{enumerate}
\item The characteristic polynomial of $A$ is the product of all the invariant factors of $A$.
\item \hl{\textit{(The Cayley-Hamilton Theorem)}} The minimal polynomial of $A$ divides the characteristic polynomial of $A$.
\item The characteristic polynomial of $A$ divides some power of the minimal polynomial of $A$. In particular these polynomials have the same roots, not counting multiplicities.
\end{enumerate}
\end{prop}

\nl

\begin{thm}
Let $A$ be an $n\times n$ matrix over the field $F$. Using the three elementary rows and column operations, the $n\times n$ matrix $xI - A$ with entries from $F[x]$ can be put into the diagonal \textit{\textbf{Smith Normal Form}} given by 
\[\begin{pmatrix}
1 & & & & & & \\
& \ddots & & & & & \\
& & 1 & & & & \\
& & & a_1(x) & & & \\
& & & & a_2(x) & & \\
& & & & & \ddots & \\
& & & & & & a_m(x)
\end{pmatrix}\]
\end{thm}

\nl

\begin{defn}
The $k\times k$ matrix with $\lambda$ along the main diagonal and 1 along the first superdiagonal is called the $k\times k$ \textit{elementary Jordan matrix with eigenvalue $\lambda$} or the \textit{Jordan block of size $k$ with eigenvalue $\lambda$}.
\end{defn}

\[\begin{pmatrix}
\lambda & 1 & & & \\
 & \lambda & \ddots & &\\
 & & \ddots & 1 & \\
 & & & \lambda & 1\\
 & & & & \lambda
\end{pmatrix}\]

\begin{defn}\nl
\begin{enumerate}
\item A matrix is said to be in \textit{\textbf{Jordan canonical form}} if it is a block diagonal matrix with Jordan blocks along the diagonal.

\item A \textbf{\textit{Jordan canonical form}} for a linear transformation $T$ is a matrix representing $T$ which is in Jordan canonical form.
\end{enumerate}
\end{defn}

\nl

\begin{thm}\hl{\textit{(Jordan Canonical Form for Linear Transformations)}} Let $V$ be a finite dimensional vector space over the field $F$ and let $T$ be a linear transformation of $V$. Assume that $F$ contains all the eigenvalues of $T$.
\begin{enumerate}
\item There is a basis for $V$ with respect to which the matrix for $T$ is in Jordan canonical form.
\item The Jordan canonical for for $T$ is unique up to a permutation of the Jordan blocks along the diagonal.
\end{enumerate}
\end{thm}

\nl

\begin{thm}\hl{\textit{(Jordan Canonical Form for Matrices)}} Let $A$ be a $n\times n$ matrix over the field $F$ and assume that $F$ contains all the eigenvalues of $A$.
\begin{enumerate}
\item The matrix $A$ is similar to a matrix in Jordan canonical form.
\item The Jordan canonical for for $A$ is unique up to a permutation of the Jordan blocks along the diagonal.
\end{enumerate}
\end{thm}

\nl

\begin{cor}\nl
\begin{enumerate}
\item If a matrix $A$ is similar to a diagonal matrix $D$, then $D$ is the Jordan canonical form of $A$.
\item Two diagonal matrices are similar if and only if their diagonal entries are the same up to a permutation.
\end{enumerate}
\end{cor}

\nl

\begin{cor}
If $A$ is an $n\times n$ matrix with entries from $F$ and $F$ contains all the eigenvalues of $A$, then $A$ is similar to a diagonal matrix over $F$ if and only if the minimal polynomial of $A$ has no repeated roots.
\end{cor}

%################################################################################
\section{Field Theory}
\setcounter{thm}{0}

\begin{defn}
The \textit{characteristic} of a field $F$, denoted $ch(F)$, is defined to be the smallest positive integer $p$ such that $p\cdots 1_F = 0$ if such a $p$ is defined to be 0 otherwise. 
\end{defn}

\nl

\begin{prop}
The characteristic of a field $F$, $ch(F)$ is either 0 or a prime $p$. If $ch(F) = p$ then for any $\al \in F$,
\[p\cdot \al = \underbrace{\al + \al + \cdots + \al}_{\text{$p$ times}} = 0.\]
\end{prop}

\nl

\begin{defn}
The \textit{\textbf{prime subfield}} of a field $F$ is the subfield of $F$ generated by the multiplicative identity $1_F$ of $F$. It is (isomorphic to) either $\Q$ or $\F_p$.
\end{defn}

\nl

\begin{defn}
If $K$ is a field containing the subfield $F$, then $K$ is said to be an \textit{\textbf{extension field}} of $F$, denoted $K/F$ or by the digram 
\begin{center}
\begin{tikzcd}[column sep = 1em, row sep = 1.5em]
K \arrow[dash, d]\\
F
\end{tikzcd}
\end{center}
In particular, every field $F$ is an extension of its prime subfield. The field $F$ is sometimes called the \textit{\textbf{base field}} of the extension.
\end{defn}

\nl

\begin{defn}
The \textit{\textbf{degree}} (or \textit{\textbf{relative degree}} or \textit{\textbf{index}}) of a field extension $K/F$, denoted $[K:F]$, is the dimension of $K$ as a vector space over $F$. The extension is said to be \textit{\textbf{finite}} if the degree of $K$ is finite and infinite otherwise.
\end{defn}

\nl

\begin{prop}
Let $\vphi: F\ra F^\p$ be a homomorphism of fields. Then $\vphi$ is either identically 0 or is injective, so that the image of $\vphi$ is either 0 or isomorphic to $F$.
\end{prop}

\nl

\begin{thm}
Let $F$ be a field and let $p(x)\in F[x]$ be an irreducible polynomial. Then there exists a field $K$ containing an isomorphic copy of $F$ in which $p(x)$ has a root. Identifying $F$ with this isomorphic copy show that there exists an extension of $F$ in which $p(x)$ has a root.
\end{thm}

\nl

\begin{thm}
Let $p(x)\in F[x]$ be an irreducible polynomial of degree $n$ over the field $F$ and let $K$ be the field $F[x]/(p(x))$. Let $\theta = x\mod(p(x))\in K$. Then the elements 
\[1, \theta, \theta^2,\ldots, \theta^{n-1}\]
are a basis for $K$ as a vector space over $F$, so the degree of the extension is $n$, i.e., $[K:F] = n$. Hence
\[K = \{a_0 + a_1\theta + a_2\theta^2 + \cdots + a_{n - 1}\theta^{n - 1}\ |\ a_0, a_1, \ldots, a_{n - 1}\in F\}\]
consists of all polynomials of degree $<n$ in $\theta$.
\end{thm}

\nl

\begin{cor}
Let $K$ be as in the previous theorem, and let $a(\theta), b(\theta)\in K$. be two polynomials of degree $<n$ in $\theta$. Then addition in $K$ is defined simply by the usual polynomial addition and multiplication in $K$ is defined by
\[a(\theta)b(\theta) = r(\theta)\]
where $r(x)$ is the remainder obtained after dividing the polynomial $a(x)b(x)$ by $p(x)$ in $F[x]$.
\end{cor}

\nl

\begin{defn}
Let $K$ be an extension of the field $F$ and let $\al,\be,\ldots\in K$ be a collection of elements of $K$. Then the smallest subfield of $K$ containing both $F$ and the elements of $\al,\be,\ldots$ denoted $F(\al,\be,\ldots)$ is called the field \textbf{\textit{generated by $\al,\be,\ldots$ over $F$}}.
\end{defn}

\nl

\begin{defn}
If the field $K$ is generated by a single element $\al$ over $F$, $K = F(\al)$, then $K$ is said to be a \textit{\textbf{simple}} extension of $F$ and the element $\al$ is called a \textit{\textbf{primitive element}} for the extension.
\end{defn}

\nl

\begin{thm}
Let $F$ be a field and let $p(x) \in F[x]$ be an irreducible polynomial. Suppose $K$ is an extension field of $F$ containing a root $\al$ of $p(x)$: $p(\al) = 0$. Let $F(\al)$ denote the subfield of $K$ generated over $F$ by $\al$. Then
\[F(\al) \cong F[x]/(p(x)).\]
\end{thm}

\nl

\begin{cor}
Suppose in the previous theorem that $p(x)$ is of degree $n$. Then
\[F(\al) = \{a_0 + a_1\al + a_2\al^2 + \cdots + a_{n - 1}\al^{n - 1} \ |\ a_0, a_1, \ldots a_{n - 1}\}\seq K.\]
\end{cor}

\nl

\begin{thm}
Let $\vphi: F\overset{\sim}{\ra} F^\p$ be an isomorphism of fields. Let $p(x)\in F[x]$ be an irreducible polynomial and let $p^\p(x)\in F^\p[x]$ be the irreducible polynomial obtained by applying the map $\vphi$ to the coefficients of $p(x)$. Let $\al$ be a root of $p(x)$ and let $\be$ be a root of $p^\p(x)$. Then there is an isomorphism 
\[\sigma:F(\al)\overset{\sim}\longrightarrow F^\p(\be)\]
\[\al\longmapsto \be\]
\end{thm}

\nl

\begin{defn}
The element $\al\in K$ is said to be \textit{\textbf{algebraic}} over $F$ if $\al$ is a root of some nonzero polynomial $f(x)\in F[x]$. If $\al$ is not algebraic over $F$ then $\al$ is said to be \textit{\textbf{transcendental}} over $F$. The extension $K/F$ is said to be \textit{algebraic} if every element of $K$ is algebraic over $F$.
\end{defn}

\nl

\hl{\textbf{Note:}} If $K$ is algebraic then it is not necessarily true that $K$ is finite. Consider the set $A$ of all algebraic numbers over $\Q$. Then $\Q(A)$ is algebraic but is certainly not finite.

\nl

\begin{prop}
Let $\al$ be algebraic over $F$. Then there is a unique, monic, irreducible polynomial $m_{\al,F}(x)\in F[x]$ which has $\al$ as a root. A polynomial $f(x)\in F[x]$ has $\al$ as a root if and only if $m_{\al, F}(x)$ divides $f(x)$ in $F[x]$. 
\end{prop}

\nl

\begin{cor}
If $L/F$ is an extension of fields and $\al$ is algebraic over both $F$ and $L$, then $m_{\al,L}(x)$ divides $m_{\al,F}(x)$ in $L[x]$.
\end{cor}

\nl

\begin{defn}
The polynomial $m_{\al, F}(x)$ is called the \textit{\textbf{minimal polynomial}} for $\al$ over $F$. The \textit{degree} of $m_\al(x)$ is called the \textit{degree} of $\al$.
\end{defn}

\nl

\begin{prop}
Let $\al$ be algebraic over the field $F$ and let $F(\al)$ be the field generated by $\al$ over $F$. Then
\[F(\al) \cong F[x]/(m_\al(x))\]
so that in particular
\[[F(\al):F] = \deg(m_\al(x)) = \deg9\al),\]
i.e., the degree of $\al$ over $F$ is the degree of the extension it generates over $F$.
\end{prop}

\nl

\begin{prop}
The element $\al$ is algebraic over $F$ if and only if the simple extension $F(\al)/F$ is finite.
\end{prop}

\nl

\begin{cor}
If the extension $K/F$ is finite, then it is algebraic.
\end{cor}

\nl

\begin{thm}
Let $F\seq K\seq L$ be fields. Then 
\[[L:F] = [L:K][K:F].\]
\end{thm}

\nl

\begin{cor}
Suppose $L/F$ is a finite extension and let $K$ be any subfield of $L$ containing $F$, $F\seq K\seq L$. Then $[K:F]$ divides $[L:F]$.
\end{cor}

\nl

\begin{defn}
Am extension $K/F$ is \textbf{\textit{finitely generated}} if there are elements $\al_1,\al_2,\ldots,\al_k$ in $K$ such that $K = F(\al_1,\al_2,\ldots,\al_k)$.
\end{defn}

\nl

\begin{lem}
$F(\al,\be)= (F(\al))(\be)$.
\end{lem}

\nl

\begin{thm}
The extension $K/F$ is finite if an only if $K$ is generated by a finite number of algebraic elements over $F$. More precisely, a field generated over $F$ by a finite number of algebraic elements of degrees $n_1,n_2,\ldots, n_k$ is algebraic of degree $\leq n_1n_2\cdots n_k$.
\end{thm}

\nl

\begin{cor}
Suppose $\al$ and $\be$ are algebraic over $F$. Then $\al\pm\be,\ \al\be,\ \al/\be$ (for $\be\neq 0$), are all algebraic.
\end{cor}

\nl

\begin{cor}
Let $L/F$ be an arbitrary extension. Then the collection of elements of $L$ that are algebraic over $F$ form a subfield $K$ of $L$.
\end{cor}

\nl

\begin{thm}
If $K$ is algebraic over $F$ and $L$ is algebraic over $K$, then $L$ is algebraic over $F$.
\end{thm}

\nl

\begin{defn}
Let $K_1$ and $K_2$ be two subfields of a field $K$. Then the \hl{\textit{\textbf{composite field}}} of $K_1$ and $K_2$, denoted $K_1K_2$, is the smallest subfield of $K$ containing both $K_1$ and $K_2$. Similarly, the composite of any collection of subfields of $K$ is the smallest subfield containing all the subfields.
\end{defn}

\nl

\begin{prop}
Let $K_1$ and $K_2$ be two finite extensions of a field $F$ contained in $K$. Then
\[[K_1K_2:F]\leq[K_1:F][K_2:F]\]
with equality if and only if an $F$-basis for one of the fields remains linearly independent over the other field. If $\al_1,\al_2,\ldots,\al_n$ and $\be_1,\be_2,\ldots,\be_m$ are bases for $K_1$ and $K_2$ over $F$, respectively, then the elements $\al_i\be_j$ for $i = 1..n$ and $j = 1..m$ span $K_1K_2$ over $F$.
\end{prop}

By this proposition, we have the following diagram

\begin{center}
\begin{tikzcd}[column sep = 2em, row sep = 2em]
 & K_1K_2\arrow[dash, dr, "\leq n"]\arrow[dash, dl, "\leq m", swap] & \\
K_1\arrow[dash, dr, "n", swap] & & K_2\arrow[dash, dl, "m"]\\
 & F &
\end{tikzcd}
\end{center}

\nl

\begin{cor}
Suppose that $[K_1:F] = n,\ [K_2: F] = m$ in the previous proposition, where $m$ and $n$ are relatively prime. Then $[K_1K_2 : F] = [K_1:F][K_2:F]$.
\end{cor}

\nl

\begin{prop}
If the element $\al\in \R$ is obtained from a field $F\subset \R$ by a series of straightedge and compass constructions then $[F(\al) : F] = 2^k$ for some integer $k\geq 0$.
\end{prop}

\nl

\begin{defn}
The extension field $K$ of $F$ is called a \textit{\textbf{splitting field}} for the polynomial $f(x)\in F[x]$ if $f(x)$ factors completely in $K[x]$ and $f(x)$ does not factor completely into linear factors over any proper subfield of $K$ containing $F$.
\end{defn}

\nl

\begin{thm}
For any field $F$, if $f(x)\in F[x]$ then there exists an extension $K$ of $F$ which is a splitting filed for $f(x)$.
\end{thm}

\nl

\begin{defn}
If $K$ is an algebraic extension of $F$ which is the splitting field over $F$ for a collection of polynomials $f(x)\in F[x]$ then $K$ is called a \textit{\textbf{normal extension}} of $F$.
\end{defn}

\nl

\begin{prop}
A splitting field of a polynomial of degree $n$ over $F$ is of degree at most $n!$ over $F$.
\end{prop}

\nl

\begin{defn}
A generator of the cyclic group of all $n^{th}$ roots of unity is called a \textit{\textbf{primitive}} $n^{th}$ root of unity.
\end{defn}

\nl

\begin{defn}
The field $\Q(\zeta_n)$ is called the \textit{\textbf{cyclotomic field of n$^{th}$ roots of unity}}.
\end{defn}

\nl

\begin{thm}
Let $\vphi: F\overset{\sim}{\ra} F^\p$ be an isomorphism of fields. Let $f(x)\in F[x]$ be a polynomial and let $f^\p(x)\in F^\p[x]$ be the polynomial obtained by applying $\vphi$ to the coefficients of $f(x)$. Let $E$ be a splitting field for $f(x)$ over $F$ and let $E^\p$ be a splitting field for $f^\p(x)$ over $F^\p$. Then the isomorphism $\vphi$ extends to an isomorphism $\sigma: E\overset{\sim}{\ra} E^\p$, i.e., $\sigma$ restricted to $F$ is the isomorphism $\vphi$:
\begin{center}
\begin{tikzcd}
\sigma:\ \  E\arrow[r, "\sim"]\arrow[dash, start anchor = {[xshift = 2.5ex]}, end anchor = {[xshift = 2.5ex]}, d] & E^\p\arrow[dash, d]\\
\vphi:\ \  F\arrow[r, "\sim"] & F^\p
\end{tikzcd}
\end{center}
\end{thm}

\nl

\begin{cor}\textit{(Uniqueness of Splitting Fields)}
Any two splitting fields for a polynomial $f(x)\in F[x]$ over a field $F$ are isomorphic.
\end{cor}

\nl

\begin{defn}
The field $\ol F$ is called an \textit{\textbf{algebraic closure}} of $F$ if $\ol F$ is algebraic over $F$ and if every polynomial $f(x)\in F[x]$ splits completely over $\ol F$.
\end{defn}

\nl

\begin{defn}
A filed $K$ is said to be \textit{\textbf{algebraically closed}} if every polynomial with coefficients in $K$ has root in $K$.
\end{defn}

\nl

\begin{prop}
Let $\ol F$ be an algebraic closure of $F$. Then $\ol F$ is algebraically closed.
\end{prop}

\nl

\begin{prop}
For any filed $F$ there exists an algebraically closed field $K$ containing $F$.
\end{prop}

\nl

\begin{prop}
Let $K$ be an algebraically closed field and let $F$ be a subfield of $K$. Then the collection of elements $\ol F$ of $K$ that are algebraic over $F$ is an algebraic closure of $F$. An algebraic closure is unique up to isomorphism.
\end{prop}

\nl

\begin{defn}
A polynomial over $F$ is called \textit{\textbf{separable}} if has no multiple roots. A polynomial which is not separable is called \textit{\textbf{inseparable}}.
\end{defn}

\nl

\begin{defn}
The \textit{\textbf{derivative}} of the polynomial
\[f(x) = a_nx^n + a_{n - 1}x^{n - 1} + \cdots + a_1x + a_0\in F[x]\]
is defined to be the polynomial
\[D_xf(x) = na_n x^{n - 1} + (n - 1)a_{n - 1}x^{n - 2} + \cdots + 2a_2x + a_1 \in F[x].\]
\end{defn}

\nl

\begin{prop}
\hl{A polynomial $f(x)$ has a multiple root $\al$ if and only if $\al$ is also a root of $D_xf(x)$.} In particular, $f(x)$ is separable if and only if it is relatively prime to its derivative $\gcd(f(x), D_xf(x)) = 1$.
\end{prop}

\nl

\begin{cor}
Every \textit{irreducible} polynomial over a field of characteristic 0 is separable. A polynomial over such a field is separable if and only if it is the product of distinct irreducible polynomials.
\end{cor}

\nl

\begin{prop}
Let $F$ be a field of characteristic $o$. Then for any $a,b\in F$,
\[(a + b)^p = a^p + b^p,\quad\text{and}\quad(ab)^p = a^pb^p.\]
Put another way, the $p^{th}$-power map defined by $\vphi(a) = a^p$ is an injective field homomorphism from $F$ to $F$. This map is called the \textit{\textbf{Frobenius endomorphism}} of $F$.
\end{prop}

\nl

\begin{cor}
Suppose that $\F$ is a finite field of characteristic $p$. Then every element of $\F$ is a $p^{th}$ power in $\F$ (notationally $\F = \F^p$).
\end{cor}

\nl

\begin{prop}
Every irreducible polynomial over a finite field $\F$ is separable. A polynomial in $\F[x]$ is separable if an only if it is the product of distinct irreducible polynomials in $\F[x]$.
\end{prop}

\nl

\begin{defn}
A filed $K$ of characteristic $p$ is called \textit{\textbf{perfect}} if every element of $K$ is a $p^{th}$ power in $K$, i.e., $K = K^p$. Any field of characteristic 0 is also called perfect.
\end{defn}

\nl

\begin{prop}
Let $p(x)$ be an irreducible polynomial over a field $F$ of characteristic $p$. Then there is a unique integer $k\geq 0$ ans a unique irreducible, separable polynomial $p_{sep}(x)\in F[x]$ such that
\[p(x) = p_{sep}(x^{p^k}). \]
\end{prop}

\nl

\begin{defn}
Let $p(x)$ be an irreducible polynomial over field of characteristic $p$. The degree $p_{sep}(x)$ in the last proposition is called the \textit{\textbf{inseparable degree}} of $p(x)$, denoted $deg_i(p(x))$.
\end{defn}

\nl

\begin{defn}
The field $K$ is said to \textit{\textbf{separable}} over $F$ if every element of $K$ is the root of a separable polynomial over $F$. A field which is not separable is \textit{\textbf{inseparable}}.
\end{defn}

\nl

\begin{cor}
Every finite extension of a perfect field is separable. In particular, every finite extension of either $\Q$ or a finite field is separable.
\end{cor}

\nl

\begin{defn}
Let $\mu_n$ denote that \textit{\textbf{group of n$^{\textbf{\textit{th}}}$ roots of unity over $\Q$}}.
\end{defn}

\nl

\begin{defn}
Define the $n^{th}$ \textit{\textbf{cyclotomic polynomial}} $\Phi_n(x)$ to be the polynomial whose roots are primitive $n^{th}$ roots of unity:
\[\Phi_n(x) = \prod_{\zeta\text{ primitive }\in\ \mu_n}(x - \zeta) = \prod_{\substack{1 \leq a \leq n \\ (a,n) = 1}} (x - \zeta_n^a)\]
(which is of degree $\vphi(n)$ for the Euler $\vphi$).
\end{defn}

\nl

\begin{lem}
The cyclotomic polynomial $\Phi_n(x)$ is a monic polynomial in $\Z[x]$ of degree $\vphi(n)$.
\end{lem}

\nl

\begin{thm}
The cyclotomic polynomial $\Phi_n(x)$ is an irreducible, monic polynomial in $\Z[x]$ of degree $\vphi(n)$.
\end{thm}

\nl

\begin{cor}
The degree over $\Q$ of the cyclotomic field of $n^{th}$ roots of unity is $\vphi(n)$:
\[[\Q(\zeta_n): \Q] = \vphi(n).\]
\end{cor}


%################################################################################

\section{Galois Theory}
\setcounter{thm}{0}

\begin{defn}\nl
\begin{enumerate}
\item An isomorphism $\sig$ of $K$ with itself is called an \textit{\textbf{automorphism}} of $K$. The collection of automorphisms of $K$ is denoted $\Aut(K)$. If $\al\in K$ we shall write $\sig\al$ for $\sig(\al)$.
\item An automorphism $\sig\in \Aut(K)$ is said to \textit{\textbf{fix}} an element $\al\in K$ if $\sig\al = \al$. If $F$ is a subset of $K$, then an automorphism $\sig$ is said to \textit{\textbf{fix}} $F$ if it fixes all the elements of $F$. 
\end{enumerate}
\end{defn}

\nl

\begin{defn}
Let $K/F$ be an extension of fields. Let $\Aut(K/F)$ be the collections of automorphisms of $K$ which fix $F$.
\end{defn}

\nl

\begin{prop}
$\Aut(K)$ is a group under composition and $\Aut(K/F)$ is a subgroup.
\end{prop}

\nl

\begin{prop}
Let $K/F$ be a field extension and let $\al \in K$ be an algebraic over $F$. Then for any $\sig\in \Aut(K/F)$, $\sig\al$ is a root of the minimal polynomial for $\al$ over $F$, i.e., $\Aut(K/F)$ permutes the roots of irreducible polynomials. Equivalently, any polynomial with coefficients in $F$ having $\al$ as a root also has $\sig\al$ as a root.
\end{prop}

\nl

\begin{prop}
Let $H\leq \Aut(K)$ be a subgroup of the group of automorphisms of $K$. Then the collection $F$ of elements of $K$ fixed by all elements of $H$ is a subfield of $K$.
\end{prop}

\nl

\begin{defn}
If $H$ is a subgroup of the group of automorphisms of $K$, the subfield of $K$ fixed by all elements of $H$ is called the \textit{\textbf{fixed field}} of $H$.
\end{defn}

\nl

\begin{prop}
The association of groups to fields and fields to groups defined above is inclusion reversing, namely
\begin{enumerate}
\item if $F_1\seq F_2\seq K$ are two subfields of $K$ then $\Aut(K/F_2)\leq \Aut(K/F_1)$, and
\item if $H_1\leq H_2\leq \Aut(K)$ are two subgroups of automorphisms with associated fixed fields $F_1$ and $F_2$, respectively, then $F_2\seq F_1$.
\end{enumerate}
\end{prop}

\nl

\begin{prop}
Let $E$ be the splitting field over $F$ of the polynomial $f(x)\in F[x]$. Then 
\[|\Aut(E/F)|\leq [E:F]\]
with equality if $f(x)$ is separable over $F$.
\end{prop}

\nl

\begin{defn}
Let $K/F$ be a finite extension. Then $K$ is said to be \textit{\textbf{Galois}} over $F$ and $K/F$ is a \textit{\textbf{Galois extension}} if $|\Aut(K/F)| = [K:F]$. If $K/F$ is Galois the group of automorphisms $\Aut(K/F)$ is called the \textit{\textbf{Galois group}} of $K/F$, denoted $\Gal(K/F)$.
\end{defn}

\nl

\begin{cor}
If $K$ is the splitting field over $F$ of a separable polynomial $f(x)$ then $K/F$ is Galois.
\end{cor}

\nl

\begin{defn}
If $f(x)$ is a separable polynomial over $F$, then the \textit{\textbf{Galois group of $f(x)$ over $F$}} is the Galois group of the splitting field of $f(x)$ over $F$.
\end{defn}

\nl

\begin{defn}
A \textit{\textbf{character}} $\chi$ of a group $G$ with values in a field $L$ is a homomorphism from $G$ to the multiplicative group of $L$:
\[\chi: G\ra L^\times\]
i.e., $\chi(g_1g_2) = \chi(g_1)\chi(g_2)$ for all $g_1,g_2\in G$ and $\chi(g)$ is a nonzero element of $L$ for all $g\in G$.
\end{defn}

\nl

\begin{defn}
The characters $\chi_1, \chi_2, \cdots, \chi_n$ of $G$ are said to be \textit{\textbf{linearly independent}} over $L$ if they are linearly independent as functions on $G$, i.e., if there is no nontrivial relation
\[a_1\chi_1 + a_2\chi_2 + \cdots + a_n\chi_n = 0\]
as a function on $G$ (that is, $a_1\chi_1 + a_2\chi_2 + \cdots + a_n\chi_n = 0$ for all $g\in G$).
\end{defn}

\nl

\begin{thm}\textit{(Linear Independence of Characters)}
If $\chi_1, \chi_2, \cdots, \chi_n$ are distinct characters of $G$ with values in $L$ then they are linearly independent over $L$.
\end{thm}

\nl

\begin{cor}
If $\sig_1, \sig_2, \ldots, \sig_n$ are distinct embeddings (injective homomorphisms) of a field $K$ into a field $L$, then they are linearly independent as functions on $K$. In particular distinct automorphisms of a field $K$ are linearly independent as functions on $K$.
\end{cor}

\nl

\begin{thm}
Let $G = \{\sig_1 = 1, \sig_2, \ldots, \sig_n\}$ be a subgorup of the automorphisms of a field $K$ and let $F$ be the fixed field. Then
\[[K:F] = n = |G|.\]
\end{thm}

\nl

\begin{cor}
Let $K/F$ be any finite extension. Then
\[|\Aut(K/F)|\leq [K:F]\]
with equality if and only if $F$ is the fixed field of $\Aut(K/F)$. Put another way, $K/F$ is Galois if and only if $F$ is the fixed field of $\Aut(K/F)$.
\end{cor}

\begin{proof}
Let $F_1$ be the fixed field of $\Aut(K/F)$. Then since every $\sigma \in \Aut(K/F)$ fixes $F$, we have that 
\[F\seq F_1\seq K.\]
By \textcolor{red}{Theorem 14.9} we then get that $[K:F_1] = |\Aut(K/F)|$. Hence $[K:F] = |\Aut(K/F)|[F_1:F]$.
\end{proof}

\nl

\begin{cor}
Let $G$ be a finite subgroup of automorphisms of a field $K$ and let $F$ be the fixed field. Then every automorphism of $K$ fixing $F$ is contained in $G$, i.e, $\Aut(K/F) = G$, so that $K/F$ is Galois, with Galois group $G$.
\end{cor}

\nl

\begin{cor}
If $G_1\neq G_2$ are distinct finite subgroups of automorphisms of a field $K$ then their fixed fields are also distinct.
\end{cor}

\nl

\begin{thm}
\hl{The extension $K/F$ is Galois if and only if $K$ is the splitting field of some separable polynomial over $F$. Furthermore, if this is the case then every irreducible polynomial with coefficients in $F$ which has a root in $K$ is separable and has all its roots in $K$ (so in particular $K/F$ is a separable extension).}
\end{thm}

\nl

\begin{defn}
Let $K/F$ be a Galois extension. If $\al\in K$ the elements $\sig \al$ for $\sig$ in $\Gal(K/F)$ are called \textit{\textbf{conjugates}} (or \textit{\textbf{Galois conjugates}}) of $\al$ over $F$. If $E$ is a subfield of $K$ containing $F$, the field $\sig(E)$ is called the \textit{\textbf{conjugate field}} of $E$ over $F$.
\end{defn}

\nl

\textbf{Note.} We now have 4 characterizations of Galois extensions $K/F$:
\begin{enumerate}
\item splitting fields of separable polynomials over $F$
\item fields where $F$ is precisely the set of elements fixed by $\Aut(K/F)$
\item fields with $[K:F] = |\Aut(K/F)|$
\item finite, normal, separable extensions.
\end{enumerate}

\nl

\begin{thm}\underline{\hl{\textbf{\textit{(Fundamental Theorem of Galois Theory)}}}}
Let $K/F$ be a Galois extension and set $G = \Gal(K/F)$. Then there is a bijection
\[ 
\left \{
\begin{tabular}{cc}
& K\\
\text{subfields } E & |\\
\text{of } K & E\\
\text{containing } F & |\\
& F
\end{tabular}
\right \}\quad \longleftrightarrow \quad 
\left \{\begin{tabular}{cc}
& 1\\
\text{subgroups } H & |\\
\text{of } G & H\\
& |\\
& G
\end{tabular}\right \}
\]
given by the correspondences
\begin{align*}
E\qquad &\longrightarrow \quad\left \{\begin{tabular}{c}
\text{the elements of } G\\
\text{fixing } E
\end{tabular}\right \}\\
\left \{\begin{tabular}{c}
\text{the fixed field}\\
\text{of } H
\end{tabular}\right \}\quad &\longleftarrow\qquad H
\end{align*}
which are inverse to each other. Under this correspondence,
\begin{enumerate}
\item (inclusion reversing) If $E_1, E_2$ correspond to $H_1, H_2$, respectively, then $E_1\seq E_2$ if and only if $H_2\leq H_1$
\item $[K:E] = |H|$ and $[E:F] = |G:H|$, the index of $H$ in $G$:
\[\begin{tabular}{ccc}
K & & \\
| & \} & |H|\\
E & & \\
| & \} & |G\ :\ H|\\
F & & 
\end{tabular}\]
\item $K/E$ is always Galois, with Galois group $\Gal(K/E) = H$:
\[\begin{tabular}{cc}
K & \\
| & H\\
E & 
\end{tabular}\]
\item $E$ is Galois over $F$ if and only $H$ is a normal subgroup in $G$. If this is the case, then the Galois group is isomorphic to the quotient group 
\[\Gal(E/F)\cong G/H.\]
More generally, even if $H$ is not necessarily normal in $G$, the isomorphisms of $E$ which fix $F$ are in one to one correspondence with the cosets $\{\sig H\}$ of $H$ in $G$.
\item If $E_1, E_2$ correspond to $H_1, H_2$, respectively, then the intersection $E_1\cap E_2$ corresponds to the group $\langle H_1, H_2\rangle$ generated by $H_1$ and $H_2$ and the composite field $E_1E_2$ corresponds to the intersection $H_1\cap H_2$. Hence the lattice of subfields of $K$ containing $F$ and the lattice of subgroups of $F$ are "dual" (the lattice diagram for one is the lattice diagram for the other turned upside down).
\end{enumerate}
\end{thm}

\nl

\begin{prop}
Any finite field is isomorphic to $\F^{p^n}$ for some prime $p$ and some integer $n\geq 1$. The field $\F_{p^n}$ is the splitting field over $\F_p$ of the polynomial $x^{p^n} - x$, with cyclic Galois group of order $n$ generated by the Frobenius automorphism $\sig_p$. The subfields of $\F_{p^n}$ are all Galois over $\F-p$ and are in one to one correspondence with the divisors $d$ of $n$. They are the fields $\F_{p^d}$, the fixed fields of $\sig_p^d$.
\end{prop}

\nl

\begin{cor}
The irreducible polynomial $x^4 + 1\in \Z[x]$ is reducible modulo every prime p.
\end{cor}

\nl

\begin{prop}
The finite field $\F_{p^n}$ is a simple extension of $\F_p$. In particular, there exists an irreducible polynomial of degree $n$ over $\F_p$ for every $n\geq 1$.
\end{prop}

\nl

\begin{prop}
The polynomial $x^{p^n} - x$ is precisely the product of all the distinct irreducible polynomials in $\F_p[x]$ of degree $d$ where $d$ runs through all the divisors of $n$.
\end{prop}

\nl

\begin{prop}
Suppose $K/F$ is a Galois extension and $F^\p/ F$ is any extension. Then $KF^\p/F^\p$ is a Galois extension, with Galois group
\[\Gal(KF^\p/F^\p)\cong \Gal(K/K\cap F^\p)\]
isomorphic to a subgroup of $\Gal(K/F)$. Pictorially,
\[
\begin{tikzcd}
 & KF^\p & \\
K\arrow[ur, dash] & & F^\p\arrow[ul, dash, "//"{anchor = center, sloped}]\\
 & K\cap F^\p\arrow[ur, dash]\arrow[ul, dash, "//"{anchor = center, sloped}] & \\
 & F\arrow[u, dash] &
\end{tikzcd}
\]
\end{prop}

\nl

\begin{cor}
Suppose $K/F$ is a Galois extension and $F^\p/F$ is any finite extension. Then 
\[[KF^\p:F] = \frac{[K:F][F^\p:F]}{[K\cap F^\p:F]}.\]
\end{cor}

\nl

\begin{prop}
Let $K_1$ and $K_2$ be Galois extensions of a field $F$. Then
\begin{enumerate}
\item The intersection $K_1\cap K_2$ is Galois over $F$.
\item The composite $K_1K_2$ is Galois over $F$. The Galois group is isomorphic to the subgroup
\[H = \{\langle \sig, \tau\rangle\ |\ \sig |_{K_1\cap K_2} = \tau|_{K_1\cap K_2}\}\]
of the direct product $\Gal(K_1/ F)\times \Gal(K_2 / F)$ consisting of elements whose restrictions to the intersection $K_1\cap K_2$ are equal.
\end{enumerate}
\[
\begin{tikzcd}
 & K_1K_2 & \\
K_1\arrow[ur, dash] & & K_2\arrow[ul, dash]\\
 & K_1\cap K_2\arrow[ur, dash]\arrow[ul, dash] & \\
 & F\arrow[u, dash] &
\end{tikzcd}
\]
\end{prop}

\nl

\begin{cor}
Let $K_1$ and $K_2$ be Galois extensions of a field $F$ with $K_1\cap K_2 = F$. Then
\[\Gal(K_1K_2/ F)\cong \Gal(K_1/ F)\times \Gal(K_2 / F).\]
Conversely, if $K$ is Galois over $F$ and $G = \Gal(K/F) = G_1\times G_2$ is the direct product of two subgroups $G_1$ and $G_2$, then $K$ is the composite of two Galois extensions $K_1$ and $K_2$ of $F$ with $K_1\cap K_2 = F$.
\end{cor}

\nl

\begin{cor}
Let $E/F$ be any finite, separable extension. Then $E$ is contained in an extension $K$ which is Galois over $F$ and is minimal in the sense that in a fixed algebraic closure of $K$ any other Galois extension of $F$ containing $E$ contains $K$.
\end{cor}

\nl

\begin{defn}
The Galois extension $K$ of $F$ containing $E$ in the previous corollary is called the \textit{\textbf{Galois closure}} of $E$ over $F$.
\end{defn}

\nl

\begin{prop}
Let $K/F$ be a finite extension. Then $K = F(\tht)$ if and only if there exist finitely many subfields of $K$ containing $F$.
\end{prop}

\nl

\begin{thm}\hl{\textit{(The Primitive Element Theorem)}} If $K/F$ is finite and separable, then $K/F$ is simple. In particular, any finite extension of fields of characteristic 0 is simple.
\end{thm}

\nl

\begin{thm}
The Galois group of the cyclotomic field $\Q(\zeta_n)$ of $n^{th}$ roots of unity s isomorphic to the multiplicative group $(\Z/n\Z)^\times$. The isomorphism is give explicitly by the map
\begin{align*}
(\Z/n\Z)^\times\  &\overset{\sim}{\longrightarrow}\ \Gal(\Q(\zeta_n)/\Q)\\
a\mod n\  &\longmapsto\  \sig_a
\end{align*}
where $\sig_a$ is the automorphism defined by 
\[\sig_a(\zeta_n) = \zeta_n^a.\]
\end{thm}

\nl

\begin{cor}
Let $n = p_1^{a_1}p_2^{a_2}\cdots p_k^{a_k}$ be the decomposition of the positive integer $n$ into distinct prime powers. The the cyclotomic fields $\Q(\zeta_{p_i^{a_i}})$, $i = 1..k$ intersect only in the field $\Q$ and their composite is the cyclotomic field $\Q(\zeta_n)$. We have
\[\Gal(\Q(\zeta_n)/\Q)\cong \Gal(\Q(\zeta_{p_1^{a_1}})/\Q) \times \Gal(\Q(\zeta_{p_2^{a_2}})/\Q) \times \cdots \times \Gal(\Q(\zeta_{p_k^{a_k}})/\Q)\]
which under the isomorphism given in the previous theorem is the Chinese Remainder Theorem
\[(Z/n\Z)^\times \cong (\Z/p_1^{\al_1}\Z)^\times\times(\Z/p_2^{\al_2}\Z)^\times\times\cdots\times (\Z/p_k^{\al_k}\Z)^\times.\]
\end{cor}

\nl

\begin{defn}
The extension $K/F$ is called an \textit{\textbf{abelian}} extension if $K/F$ is Galois and $\Gal(K/F)$ is an abelian group.
\end{defn}

\nl

\begin{cor}
Let $G$ be any finite abelian group. Then there is a subfield $K$ of a cyclotomic field with $\Gal(K/\Q) \cong G$.
\end{cor}

\nl

\begin{thm}\textit{(Kronecker-Weber)}
Let $K$ be a finite abelian extension of $\Q$. Then $K$ is contained in a cyclotomic extension of $\Q$.
\end{thm}

\nl

\begin{prop}
\hl{The regular $n$-gon can be constructed by straightedge and compass if and only if $n = 2^kp_1\cdots p_r$ is the porduct of a power of 2 and distinct Fermat primes.}
\end{prop}

\nl

\textbf{Note:} Fermat primes are primes of the form $2^{2^n}+1$.














\end{document}
