\documentclass[onesided]{ccg-pset}
\usepackage{fourier}
\usepackage[T1]{fontenc}

\course{Math 6140}
\psnum{12}
\author{Colton Grainger}
\date{\today}

\begin{document}
\maketitle

\begin{enumerate}

\item \label{14.2.1} 14.2.1.
The minimal polynomial over $\Q$ for the element $\sqrt[]{ 2 } + \sqrt[]{ 5 }$ is:



\item \label{14.2.4} 14.2.4. 
Let $p$ be a prime. The elements of the Galois group of $x^{p}-2$ over $\Q$ are:



\item \label{14.2.5} 14.2.5. 
The Galois group of $x^p-2$ (as in problem~\ref{14.2.4}) is isomorphic to the matrix group
\[
    H = \set{\mqty[a & b\\ 0 & 1] \qgiven a,b \in \Fp \qand a \neq 0}
.\]


\item \label{14.2.8} 14.2.8. 
Suppose $K$ is a Galois extension of $F$ of degree $p^n$ for some prime $p$ and some $n \ge 1$. 
There are Galois extensions of $F$ contained in $K$ of degrees $p$ and $p^{n-1}$.



\item \label{14.2.11} 14.2.11.
Suppose $f(x) \in \Z[x]$ is an irreducible quartic whose splitting field has Galois group $S_4$ over $\Q$. 
Let $\theta$ be a root of $f(x)$ and set $K = \Q(\theta)$. 
Then $K$ is an extension of $Q$ of degree $4$ which has no proper subfields.
We determine if there are any Galois extensions of $\Q$ of degree $4$ with no proper subfields.



\item \label{14.2.13} 14.2.13.
If the Galois group of the splitting field of a cubic over $\Q$ is the cyclic group of order $3$, then all the roots of the cubic are real.

\item \label{14.3.1} 14.3.1.
The factors of $x^8 - x$ as irreducibles in $\Z[x]$ and $\F_2[x]$, respectively, are:


\item \label{14.3.3} 14.3.3.
An algebraically closed field is infinite.


\item \label{14.3.7} 14.3.7.
\begin{enumerate}
    \item One of $2$, $3$, or $6$ is a square in $\Fp$ for every prime $p$.
    \item Therefore, for every prime $p$, the polynomial 
        \begin{equation}
            \label{allprimes}
        x^6 - 11x^{4} + 36x^{2} - 36 = \paren{x^{2} -2} \paren{x^{2} -3} \paren{x^{2} -6} 
        \end{equation}
        has a root modulo $p$.
    \item However, the polynomial~\eqref{allprimes} is irreducible over $\Z$.
\end{enumerate}


\item \label{14.3.8} 14.3.8. We exhibit an \term{Artin--Schreier extension}.
\begin{enumerate}
    \item The splitting field $E$ of the polynomial $x^p - x -a$ over $\Fp$, where $a \neq 0$ and $a \in \Fp$, is:
    \item For a root $\alpha$ of $x^p - x -a$, the map $\alpha \mapsto \alpha + 1$ induces an automorphism of $E$ fixing $\Fp$.
    \item Therefore, the Galois group of $x^p - x -a$ over $\Fp$ is cyclic.
\end{enumerate}
 
\item \label{14.3.9} 14.3.9.
Let $q = p^m$ be a power of the prime $p$ and let $\Fq = \F_{p^m}$ be the finite field with $q$ elements. Then let $\sigma_q = \sigma_p^m$ be the $m$th power of the Froebenius automorphism $\sigma_p$, called the \term{$q$-Froebenius automorphism}.

\begin{enumerate}
    \item The $q$-Froebenius automorphism $\sigma_q$ fixes $\Fq$.
    \item Every finite extension of $\Fq$ of degree $n$ is the splitting field $K$ of $x^{q^n} -x$ over $\Fq$, hence unique.
    \item For $K/\Fq$ the unique degree $n$ extension of $\Fq$, we have 
    \[
        \Gal(K/\Fq) = \ang{\sigma_q}
    .\]
    \item Hence, there's a bijective correspondence
    \[
        \set{\mqty{\text{subfields $E$}\\
                    K \ge E \ge F}} 
            \leftrightsquigarrow 
        \set{\mqty{\text{divisors $d$}\\
                   1 \mid d \mid n}}
    .\]
\end{enumerate}
 
\item \label{14.3.10} 14.3.10.
Let $\phi$ be the Euler totient function, $p$ a prime, and $n$ a natural number. Then
\[
    n \qq{divides} \phi(p^n -1)
.\]
\begin{proof}
    Observe that $\phi(p^n -1)$ is the order of the group of automorphisms of a cyclic group of order $p^n -1$.
\end{proof}

\end{enumerate}

\end{document}
