\documentclass[onesided]{ccg-pset}

\course{Math 6140}
\psnum{12}
\author{Colton Grainger}
\date{\today}

\DeclareMathOperator{\ord}{ord}
\newcommand{\Fp}{\mathbb{F}_p} % finite fields
\newcommand{\Fpn}{\mathbb{F}_{p^n}} % finite fields
\newcommand{\Fq}{\mathbb{F}_q} % finite fields
\newcommand{\Fqn}{\mathbb{F}_{q^n}} % finite fields

\begin{document}
\maketitle

\begin{enumerate}

\item \label{14.2.1} 14.2.1.
The minimal polynomial over $\Q$ for the element $\sqrt[]{ 2 } + \sqrt[]{ 5 }$ is $x^4 - 14x^2 +9$.

\begin{proof}
    Noting that $\Q(\sqrt{ 2 } + \sqrt{ 5 })$ is a subfield of the Galois extension $\Q(\sqrt{ 2 } , \sqrt{ 5 })$, it suffices to find the monic polynomial $m(x) \in \Q[x]$ of minimal degree such that $\Q[x]/\ang{m(x)}$ contains the four elements $\sqrt{ 2 } \pm \sqrt{ 5 }$ and $\pm \sqrt{ 2 } + \sqrt{ 5 }$. (These elements are the \emph{distinct} conjugate pairs of $\sqrt{ 2 } + \sqrt{ 5 }$, and the Galois group $\Gal\set{\Q(\sqrt{ 2 } + \sqrt{ 5 })}$ necessarily permutes the conjugate elements.) Consider then the product
    \begin{align}
        \prod\limits_{\text{conjugates}}\paren{x - \paren{\pm\sqrt{ 2 } \pm \sqrt{ 5 }}} 
        \label{congprod}
            & = \paren{ x^2 -\paren{ 2 + 2\sqrt{ 10 }+5  }  } \paren{ x^2 - \paren{ 2 - 2\sqrt{ 10 } + 5 }  } \\
            & = x^4 - 14x^2 -9.
    \end{align}
    We conclude $m(x) = x^4 - 14x^2 -9$ is minimal, because any product of $3$ or fewer linear factors in~\eqref{congprod} is not a polynomial in $\Q[x]$.
\end{proof}

\item \label{14.2.4} 14.2.4. 
    Let $p$ be an odd\footnote{The case $p=2$ with $x^2 -2$ has Galois group $C_2$ and automorphisms $\sqrt{ 2 } \mapsto \pm \sqrt{ 2 }$.} prime. Then the Galois group of $x^{p}-2$ over $\Q$ is the semidirect product of cyclic groups
    \[
        \Gal\paren{ \Q\paren{ \zeta, \sqrt[p]{ 2 } } / \Q }  \cong C_p \rtimes_\psi C_{p-1}
    \]
    where $\zeta$ is a primitive $p$th root of unity, $\sqrt[p]{ 2 }$ is the real $p$th root of $2$, and $\psi \colon C_{p-1} \xrightarrow{\cong} \Aut C_p$ is an isomorphism from $C_{p-1}$ to the $p-1$ automorphisms of $C_p$.
\begin{proof}
    Because $\chr\Q =0$, the polynomial $x^p -2$ is separable, with roots
    \begin{equation*}
        \set{ \zeta^k \sqrt[p]{ 2 } \qq{such that} 0 \le k \le p-1} \subset \C.
    \end{equation*}
    We see the splitting field of $x^p-2$ is generated by $\zeta, \sqrt[p]{ 2 } \in S$. 
    Hence the extension $F = \Q\paren{ \zeta, \sqrt[p]{ 2 } } / \Q$ is the splitting field of a separable polynomial, and so $F$ is Galois. 
    To see that $[F:\Q] = (p-1)p$, note $p$ and $p-1$ are coprime, and recall the (partial) Hasse diagram:
    \[
        \xymatrix{
            & \Q(\sqrt[p]{ 2 }, \zeta) \ar@{-}[dl]_{p-1} \ar@{-}[dr]^p  &\\
        \Q(\sqrt[p]{ 2 }) \ar@{-}[dr]_p & & \Q(\zeta) \ar@{-}[dl]^{p-1}\\
                                        & \Q &}
    \]

    Now, the Galois group is determined by its action on the (fixed) generators $\sqrt[p]{ 2 }, \zeta$, which gives the possibilities
    \begin{align}
        \label{automs}
        \zeta & \mapsto \zeta^k, & k=1, \ldots, p-1,\\
        \sqrt[p]{ 2 } & \mapsto \zeta^\ell\sqrt[p]{ 2 },  & \ell = 0, \ldots, p-1.
    \end{align}
    By order considerations, as the degree of $F/\Q$ is $p(p-1)$, the automorphisms in~\eqref{automs} form a complete list.
    Now let $m$ be an integer $1 < m < p-1$ that is coprime to $p-1$ (we need a generator for the cyclic group of order $p-1$). Then define $\sigma, \tau \in \Gal(F/\Q)$ by
    \[
        \sigma = \begin{cases}
            \zeta \mapsto \zeta^{m}\\
            \sqrt[p]{ 2 } \mapsto \sqrt[p]{ 2 }
        \end{cases}
        \qand
        \tau = \begin{cases}
            \zeta \mapsto \zeta\\
            \sqrt[p]{ 2 } \mapsto \zeta\sqrt[p]{ 2 }
        \end{cases}
    .\]
    Because $p$ is prime and $\ord\zeta = p$, it is visible that $\ord \tau =p$. We deduce that $\ord \sigma = p-1$ from both $(m, p-1) = 1$ and the fact that ``$\sigma \in \Aut{\ang{ \zeta } } \cong C_{p-1}$'' is a cyclic group.

    Moreover, as $\sigma$ and $\tau$ have coprime orders, by Lagrange's theorem, they generate the Galois group. We conclude that, as a set, $\Gal( F/\Q ) = \ang{\tau} \times \ang{\sigma} = C_{p} \times C_{p-1}$. The group structure $\Gal(F / \Q) \cong C_p \rtimes_\psi C_{p-1}$ follows from the isomorphism $\psi \colon \ang{ \sigma } \to \ang{ \tau }$ defined by
    \[
        \psi(\sigma).\tau = \sigma^{-1}\tau\sigma
    .\]
 \end{proof}

\item \label{14.2.5} 14.2.5. 
    The Galois group $\Gal(F/\Q)$ of $x^p-2$ (as in problem~\ref{14.2.4}) is isomorphic to the matrix group
\[
    H = \set{\mqty[a & b\\ 0 & 1] \qgiven a,b \in \Fp \qand a \neq 0}
.\]

\begin{proof}
    Fix $a$, a generator of the group of units $\Fp^\times$, and define a group homomorphism $f\colon \Gal(F/\Q) \to H$ by
    \[
        \sigma \underset{f}{\mapsto} \mqty[ a & 0\\ 0 & 1 ] \qand
        \tau \underset{f}{\mapsto} \mqty[ 1 & 1\\ 0 & 1 ]
    .\]
    Because $H$ has order $p(p-1)$ and $f$ is surjective (by choice of generator $a$), we have a surjective homomorphism between finite groups, and thus an isomorphism.
\end{proof}


\item \label{14.2.8} 14.2.8. 
Suppose $K$ is a Galois extension of $F$ of degree $p^n$ for some prime $p$ and some $n \ge 1$. 
There are Galois extensions of $F$ contained in $K$ of degrees $p$ and $p^{n-1}$.



\item \label{14.2.11} 14.2.11.
Suppose $f(x) \in \Z[x]$ is an irreducible quartic whose splitting field has Galois group $S_4$ over $\Q$. 
Let $\theta$ be a root of $f(x)$ and set $K = \Q(\theta)$. 
Then $K$ is an extension of $Q$ of degree $4$ which has no proper subfields.
We determine if there are any Galois extensions of $\Q$ of degree $4$ with no proper subfields.



\item \label{14.2.13} 14.2.13.
If the Galois group of the splitting field of a cubic over $\Q$ is the cyclic group of order $3$, then all the roots of the cubic are real.

\item \label{14.3.1} 14.3.1.
The factors of $x^8 - x$ as irreducibles in $\Z[x]$ and $\F_2[x]$, respectively, are:


\item \label{14.3.3} 14.3.3.
An algebraically closed field is infinite.


\item \label{14.3.7} 14.3.7.
\begin{enumerate}
    \item One of $2$, $3$, or $6$ is a square in $\Fp$ for every prime $p$.
    \item Therefore, for every prime $p$, the polynomial 
        \begin{equation}
            \label{allprimes}
        x^6 - 11x^{4} + 36x^{2} - 36 = \paren{x^{2} -2} \paren{x^{2} -3} \paren{x^{2} -6} 
        \end{equation}
        has a root modulo $p$.
    \item However, the polynomial~\eqref{allprimes} is irreducible over $\Z$.
\end{enumerate}


\item \label{14.3.8} 14.3.8. We exhibit an \term{Artin--Schreier extension}.
\begin{enumerate}
    \item The splitting field $E$ of the polynomial $x^p - x -a$ over $\Fp$, where $a \neq 0$ and $a \in \Fp$, is:
    \item For a root $\alpha$ of $x^p - x -a$, the map $\alpha \mapsto \alpha + 1$ induces an automorphism of $E$ fixing $\Fp$.
    \item Therefore, the Galois group of $x^p - x -a$ over $\Fp$ is cyclic.
\end{enumerate}
 
\item \label{14.3.9} 14.3.9.
Let $q = p^m$ be a power of the prime $p$ and let $\Fq = \F_{p^m}$ be the finite field with $q$ elements. Then let $\sigma_q = \sigma_p^m$ be the $m$th power of the Froebenius automorphism $\sigma_p$, called the \term{$q$-Froebenius automorphism}.

\begin{enumerate}
    \item The $q$-Froebenius automorphism $\sigma_q$ fixes $\Fq$.
    \item Every finite extension of $\Fq$ of degree $n$ is the splitting field $K$ of $x^{q^n} -x$ over $\Fq$, hence unique.
    \item For $K/\Fq$ the unique degree $n$ extension of $\Fq$, we have 
    \[
        \Gal(K/\Fq) = \ang{\sigma_q}
    .\]
    \item Hence, there's a bijective correspondence
    \[
        \set{\mqty{\text{subfields $E$}\\
                    K \ge E \ge F}} 
            \leftrightsquigarrow 
        \set{\mqty{\text{divisors $d$}\\
                   1 \mid d \mid n}}
    .\]
\end{enumerate}
 
\item \label{14.3.10} 14.3.10.
Let $\phi$ be the Euler totient function, $p$ a prime, and $n$ a natural number. Then
\[
    n \qq{divides} \phi(p^n -1)
.\]
\begin{proof}
    Observe that $\phi(p^n -1)$ is the order of the group of automorphisms of a cyclic group of order $p^n -1$.
\end{proof}

\end{enumerate}

\end{document}
